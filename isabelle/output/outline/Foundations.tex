%
\begin{isabellebody}%
\setisabellecontext{Foundations}%
%
\isadelimtheory
%
\endisadelimtheory
%
\isatagtheory
%
\endisatagtheory
{\isafoldtheory}%
%
\isadelimtheory
%
\endisadelimtheory
%
\isadelimdocument
%
\endisadelimdocument
%
\isatagdocument
%
\isamarkupchapter{Foundations%
}
\isamarkuptrue%
%
\endisatagdocument
{\isafolddocument}%
%
\isadelimdocument
%
\endisadelimdocument
%
\begin{isamarkuptext}%
\label{chap:foundations}%
\end{isamarkuptext}\isamarkuptrue%
%
\begin{isamarkuptext}%
In this chapter, the concepts that are relevant for the main part of this thesis will be introduced in text, as well as formalised in Isabelle. The formalisations of \cref{sec:LTS,sec:strong_bisimilarity} are based on those done by Benjamin Bisping in \cite{bisping2019computing} (the code is available on GitHub%
\footnote{see \code{\href{https://coupledsim.bbisping.de/code/}%
{coupledsim.bbisping.de/code/}}}%
).
All other formalisations were done as part of this thesis.%
\end{isamarkuptext}\isamarkuptrue%
%
\isadelimtheory
%
\endisadelimtheory
%
\isatagtheory
%
\endisatagtheory
{\isafoldtheory}%
%
\isadelimtheory
%
\endisadelimtheory
%
\end{isabellebody}%
\endinput
%:%file=~/projects/Reducing-Reactive-to-Strong-Bisimilarity/isabelle/Foundations.thy%:%
%:%24=7%:%
%:%36=8%:%
%:%40=10%:%
%:%41=11%:%
%:%42=12%:%
%:%43=13%:%
%:%44=14%:%
%
\begin{isabellebody}%
\setisabellecontext{Labelled{\isacharunderscore}{\kern0pt}Transition{\isacharunderscore}{\kern0pt}Systems{\isacharunderscore}{\kern0pt}with{\isacharunderscore}{\kern0pt}TimeOuts}%
%
\isadelimtheory
%
\endisadelimtheory
%
\isatagtheory
%
\endisatagtheory
{\isafoldtheory}%
%
\isadelimtheory
%
\endisadelimtheory
%
\isadelimdocument
%
\endisadelimdocument
%
\isatagdocument
%
\isamarkupsection{Labelled Transition Systems with Time-Outs%
}
\isamarkuptrue%
%
\endisatagdocument
{\isafolddocument}%
%
\isadelimdocument
%
\endisadelimdocument
%
\begin{isamarkuptext}%
\label{sec:LTSt}%
\end{isamarkuptext}\isamarkuptrue%
%
\begin{isamarkuptext}%
In addition to the hidden action $\tau$, labelled transition systems with time-outs (\LTSt{}) \cite{vanglabbeek2021failure} include the \emph{time-out action} $t$ as another special action; $t$-transitions can only be performed when no other (non--time-out) transition is allowed in a given environment. The rationale is that, in this model, all transitions that are facilitated by or independent of the environment happen instantaneously, and only when no such transition is possible, time elapses and the system is idle.
However, since the passage of time is not quantified here, the system does not \emph{have} to take a time-out transition in such a case; instead, the environment can spontaneously change its set of allowed actions (corresponding to a time-out on the part of the environment). Thus, the resolution of an idling period is non-deterministic.

In most works on LTSs, the actions which the environment allows in any given moment are usually not modelled explicitly; an (often implicit) requirement for any property of the system is that it should hold for arbitrary (and arbitrarily changing) environments. The introduction of time-outs necessitates an explicit consideration of the environment, as the possibility of a transition not only depends on whether its labelling action is currently allowed, but potentially on the set of all actions currently allowed by the environment. This is why, in previous sections, I have put unusual emphasis on the actions that are allowed or blocked by the system's environment in a given moment. Henceforth, I will refer to \enquote{environments allowing \emph{exactly} the actions in $X$} simply as \enquote{environments~$X$}.
\pagebreak
\example{%
The process $p$ can perform an $a$-transition in environments allowing the action $a$ and a $t$-transition in environments blocking $a$. On the other hand, the process $q$ cannot perform a $t$-transition in any environment, since the $\tau$-transition will always be performed immediately.

\lts{
    \node[state]    (p0)                            {$p$};
    \node[state]    (p1) [below left of=p0]         {$p_1$};
    \node[state]    (p2) [below right of=p0]        {$p_2$};
    \node[state]    (q0) [right of=p0,xshift=3cm]   {$q$};
    \node[state]    (q1) [below left of=q0]         {$q_1$};
    \node[state]    (q2) [below right of=q0]        {$q_2$};
    
    \path   (p0) edge node[above left]  {$a$}   (p1)
                 edge node              {$t$}   (p2)
            (q0) edge node[above left]  {$\tau$}(q1)
                 edge node              {$t$}   (q2);
}}

Furthermore, since $t$-transitions (as well as $\tau$-transitions) are hidden, they cannot trigger a change of the environment, so some states may only ever be entered in certain environments.

\example{%
The process $p$ can perform a $t$-transition only in environments blocking $a$. Therefore, the subsequent state $p_2$ must be entered in such an environment. The $\tau$-transition is now the only possible transition and will always be performed immediately.

\lts{
    \node[state]    (p0)                            {$p$};
    \node[state]    (p1) [below left of=p0]         {$p_1$};
    \node[state]    (p2) [below right of=p0]        {$p_2$};
    \node[state]    (p3) [below left of=p2]         {$p_3$};
    \node[state]    (p4) [below right of=p2]        {$p_4$};
    
    \path   (p0) edge node[above left]  {$a$}   (p1)
                 edge node              {$t$}   (p2)
            (p2) edge node[above left]  {$\tau$}   (p3)
                 edge node              {$a$}(p4);
}}

On the other hand, transitions with labels other than $\tau$ and $t$ require an interaction with the environment, and therefore \emph{are} detectable and can trigger a change in the allowed actions of the environment.

\example{%
Only in environments blocking $a$, $p$ can make a $t$-transition to $p_2$. However (if $b$ is allowed), the performance of the $b$-transition into $p_3$ may trigger a change of the environment, so it is possible that $p_3$ could perform its $a$-transition.

\lts{
    \node[state]    (p0)                            {$p$};
    \node[state]    (p1) [below left of=p0]         {$p_1$};
    \node[state]    (p2) [below right of=p0]        {$p_2$};
    \node[state]    (p3) [right of=p2,xshift=1cm]              {$p_3$};
    \node[state]    (p4) [below left of=p3]         {$p_4$};
    \node[state]    (p5) [below right of=p3]        {$p_5$};
    
    \path   (p0) edge node[above left]  {$a$}   (p1)
                 edge node              {$t$}   (p2)
            (p2) edge node              {$b$}   (p3)
            (p3) edge node[above left]  {$\tau$}   (p4)
                 edge node              {$a$}(p5);
}}

These semantics of \LTSt{}s induce a novel notion of behavioural equivalence, which will be discussed in the next section.%
\end{isamarkuptext}\isamarkuptrue%
%
\isadelimdocument
%
\endisadelimdocument
%
\isatagdocument
%
\isamarkupsubsubsection{Note on Metavariable usage%
}
\isamarkuptrue%
%
\endisatagdocument
{\isafolddocument}%
%
\isadelimdocument
%
\endisadelimdocument
%
\begin{isamarkuptext}%
If not referenced directly by $\vartheta(p)$ or $\vartheta_X(p)$, arbitrary states of an \LTSt{} range over $P, Q, P', Q', \dots$, where $P$ and $P'$ are used for states connected by some transition (i.e.\@ $P \xrightarrow{\alpha}_\vartheta P'$), whereas $P$ and $Q$ are used for states possibly related by an equivalence, as will be discussed in the next section.%
\end{isamarkuptext}\isamarkuptrue%
%
\isadelimdocument
%
\endisadelimdocument
%
\isatagdocument
%
\isamarkupsubsubsection{Practical Example%
}
\isamarkuptrue%
%
\endisatagdocument
{\isafolddocument}%
%
\isadelimdocument
%
\endisadelimdocument
%
\begin{isamarkuptext}%
As in \cref{sec:LTS}, we shall consider a real-world machine that may be modelled as an \LTSt{}. Continuing with our example, let us imagine that our simple vending machine ejects the coin if no snack has been selected after a certain amount of time. We can attempt to model the machine with this extended behaviour as an LTS, where the coin ejection requires no interaction and is therefore also modelled as a $\tau$-transition.

\lts{
    \node[state]    (p0) {$p$};
    \node[state]    (p1) [below of=p0] {$p_1$};
    \node[state]    (p2) [below left of=p1,xshift=-10pt] {$p_2$};
    \node[state]    (p3) [below of=p1] {$p_3$};
    \node[state]    (p4) [below right of=p1,xshift=10pt] {$p_4$};
    
    \path   (p0) edge node[right] {coin} (p1)
            (p1) edge node[above left] {choc} (p2)
                 edge node[right] {nuts} (p3)
                 edge node[above right] {crisps} (p4);
                 
    \draw (p2) to[out=150, in=180, looseness=1, edge node={node [left] {$\tau$}}] (p0);
    \draw (p4) to[out=30, in=-30, looseness=1, edge node={node [right] {$\tau$}}] (p0);
    \draw (p3) to[out=-20, in=0, looseness=2.5, edge node={node [right] {$\tau$}}] (p0);
    \draw (p1) to[out=150, in=230, looseness=.8, edge node={node [left] {$\tau$}}] (p0);
}

However, this LTS also models a machine that randomly ejects coins right after insertion.
In order to distinguish these behaviours, we need a $t$-transition along with \LTSt{} semantics.

\lts{
    \node[state]    (p0) {$p$};
    \node[state]    (p1) [below of=p0] {$p_1$};
    \node[state]    (p11) [below left of=p0,xshift=-10pt] {$p_1'$};
    \node[state]    (p2) [below left of=p1,xshift=-10pt] {$p_2$};
    \node[state]    (p3) [below of=p1] {$p_3$};
    \node[state]    (p4) [below right of=p1,xshift=10pt] {$p_4$};
    
    \path   (p0) edge node[right] {coin} (p1)
            (p1) edge node[above left] {choc} (p2)
                 edge node[right] {nuts} (p3)
                 edge node[above right] {crisps} (p4)
            (p1) edge node[above] {$t$} (p11)
            (p11) edge node[left] {$\tau$} (p0);
                 
    \draw (p2) to[out=150, in=180, looseness=2, edge node={node [left] {$\tau$}}] (p0);
    \draw (p4) to[out=30, in=-30, looseness=1, edge node={node [right] {$\tau$}}] (p0);
    \draw (p3) to[out=-20, in=0, looseness=2.5, edge node={node [right] {$\tau$}}] (p0);
}%
\end{isamarkuptext}\isamarkuptrue%
%
\isadelimdocument
%
\endisadelimdocument
%
\isatagdocument
%
\isamarkupsubsection{Isabelle%
}
\isamarkuptrue%
%
\endisatagdocument
{\isafolddocument}%
%
\isadelimdocument
%
\endisadelimdocument
%
\begin{isamarkuptext}%
We extend LTSs with hidden actions (\isa{lts{\isacharunderscore}{\kern0pt}tau}) by the special action \isa{t}. We have to explicitly require (/assume) that \isa{{\isasymtau}\ {\isasymnoteq}\ t}; when instantiating the locale \isa{lts{\isacharunderscore}{\kern0pt}timeout} and specifying a concrete type for the type variable \isa{{\isacharprime}{\kern0pt}a}, this assumption must be (proved to be) satisfied.%
\end{isamarkuptext}\isamarkuptrue%
\isacommand{locale}\isamarkupfalse%
\ lts{\isacharunderscore}{\kern0pt}timeout\ {\isacharequal}{\kern0pt}\ lts{\isacharunderscore}{\kern0pt}tau\ tran\ {\isasymtau}\ \isanewline
\ \ \isakeyword{for}\ tran\ {\isacharcolon}{\kern0pt}{\isacharcolon}{\kern0pt}\ {\isachardoublequoteopen}{\isacharprime}{\kern0pt}s\ {\isasymRightarrow}\ {\isacharprime}{\kern0pt}a\ {\isasymRightarrow}\ {\isacharprime}{\kern0pt}s\ {\isasymRightarrow}\ bool{\isachardoublequoteclose}\ \isanewline
\ \ \ \ {\isacharparenleft}{\kern0pt}{\isachardoublequoteopen}{\isacharunderscore}{\kern0pt}\ {\isasymlongmapsto}{\isacharunderscore}{\kern0pt}\ {\isacharunderscore}{\kern0pt}{\isachardoublequoteclose}\ {\isacharbrackleft}{\kern0pt}{\isadigit{7}}{\isadigit{0}}{\isacharcomma}{\kern0pt}\ {\isadigit{7}}{\isadigit{0}}{\isacharcomma}{\kern0pt}\ {\isadigit{7}}{\isadigit{0}}{\isacharbrackright}{\kern0pt}\ {\isadigit{8}}{\isadigit{0}}{\isacharparenright}{\kern0pt}\isanewline
\ \ \ \ \isakeyword{and}\ {\isasymtau}\ {\isacharcolon}{\kern0pt}{\isacharcolon}{\kern0pt}\ {\isachardoublequoteopen}{\isacharprime}{\kern0pt}a{\isachardoublequoteclose}\ {\isacharplus}{\kern0pt}\isanewline
\ \ \isakeyword{fixes}\ t\ {\isacharcolon}{\kern0pt}{\isacharcolon}{\kern0pt}\ {\isachardoublequoteopen}{\isacharprime}{\kern0pt}a{\isachardoublequoteclose}\isanewline
\ \ \isakeyword{assumes}\ tau{\isacharunderscore}{\kern0pt}not{\isacharunderscore}{\kern0pt}t{\isacharcolon}{\kern0pt}\ {\isacartoucheopen}{\isasymtau}\ {\isasymnoteq}\ t{\isacartoucheclose}\isanewline
\isakeyword{begin}%
\begin{isamarkuptext}%
We define the set of (relevant) visible actions (usually denoted by $A \subseteq \Act$) as the set of all actions that are not hidden and that are labels of some transition in the given LTS.%
\end{isamarkuptext}\isamarkuptrue%
\isacommand{definition}\isamarkupfalse%
\ visible{\isacharunderscore}{\kern0pt}actions\ {\isacharcolon}{\kern0pt}{\isacharcolon}{\kern0pt}\ {\isacartoucheopen}{\isacharprime}{\kern0pt}a\ set{\isacartoucheclose}\isanewline
\ \ \isakeyword{where}\ {\isacartoucheopen}visible{\isacharunderscore}{\kern0pt}actions\ \isanewline
\ \ \ \ {\isasymequiv}\ {\isacharbraceleft}{\kern0pt}a{\isachardot}{\kern0pt}\ {\isacharparenleft}{\kern0pt}a\ {\isasymnoteq}\ {\isasymtau}{\isacharparenright}{\kern0pt}\ {\isasymand}\ {\isacharparenleft}{\kern0pt}a\ {\isasymnoteq}\ t{\isacharparenright}{\kern0pt}\ {\isasymand}\ {\isacharparenleft}{\kern0pt}{\isasymexists}\ p\ p{\isacharprime}{\kern0pt}{\isachardot}{\kern0pt}\ p\ {\isasymlongmapsto}a\ p{\isacharprime}{\kern0pt}{\isacharparenright}{\kern0pt}{\isacharbraceright}{\kern0pt}{\isacartoucheclose}%
\begin{isamarkuptext}%
The formalisations in upcoming sections will often involve the type \isa{{\isacharprime}{\kern0pt}a\ set\ option}, which has values of the form \isa{None} and \isa{Some\ X} for some \isa{X\ {\isacharcolon}{\kern0pt}{\isacharcolon}{\kern0pt}\ {\isacharprime}{\kern0pt}a\ set}. We will usually use the metavariable \isa{XoN} (for \enquote{\isa{X} or \isa{None}}). The following abbreviation will be useful in these situations.%
\end{isamarkuptext}\isamarkuptrue%
\isacommand{abbreviation}\isamarkupfalse%
\ some{\isacharunderscore}{\kern0pt}visible{\isacharunderscore}{\kern0pt}subset\ {\isacharcolon}{\kern0pt}{\isacharcolon}{\kern0pt}\ {\isacartoucheopen}{\isacharprime}{\kern0pt}a\ set\ option\ {\isasymRightarrow}\ bool{\isacartoucheclose}\isanewline
\ \ \isakeyword{where}\ {\isacartoucheopen}some{\isacharunderscore}{\kern0pt}visible{\isacharunderscore}{\kern0pt}subset\ XoN\ \isanewline
\ \ \ \ {\isasymequiv}\ {\isasymexists}\ X{\isachardot}{\kern0pt}\ XoN\ {\isacharequal}{\kern0pt}\ Some\ X\ {\isasymand}\ X\ {\isasymsubseteq}\ visible{\isacharunderscore}{\kern0pt}actions{\isacartoucheclose}%
\begin{isamarkuptext}%
The initial actions of a process ($\mathcal{I}(p)$ in \cite{rbs}) are the actions for which the process has a transition it can perform immediately (if facilitated by the environment), i.e.\@ it is not a $t$-transition.%
\end{isamarkuptext}\isamarkuptrue%
\isacommand{definition}\isamarkupfalse%
\ initial{\isacharunderscore}{\kern0pt}actions\ {\isacharcolon}{\kern0pt}{\isacharcolon}{\kern0pt}\ {\isacartoucheopen}{\isacharprime}{\kern0pt}s\ {\isasymRightarrow}\ {\isacharprime}{\kern0pt}a\ set{\isacartoucheclose}\isanewline
\ \ \isakeyword{where}\ {\isacartoucheopen}initial{\isacharunderscore}{\kern0pt}actions{\isacharparenleft}{\kern0pt}p{\isacharparenright}{\kern0pt}\ \isanewline
\ \ \ \ {\isasymequiv}\ {\isacharbraceleft}{\kern0pt}{\isasymalpha}{\isachardot}{\kern0pt}\ {\isacharparenleft}{\kern0pt}{\isasymalpha}\ {\isasymin}\ visible{\isacharunderscore}{\kern0pt}actions\ {\isasymor}\ {\isasymalpha}\ {\isacharequal}{\kern0pt}\ {\isasymtau}{\isacharparenright}{\kern0pt}\ {\isasymand}\ {\isacharparenleft}{\kern0pt}{\isasymexists}\ p{\isacharprime}{\kern0pt}{\isachardot}{\kern0pt}\ p\ {\isasymlongmapsto}{\isasymalpha}\ p{\isacharprime}{\kern0pt}{\isacharparenright}{\kern0pt}{\isacharbraceright}{\kern0pt}{\isacartoucheclose}%
\begin{isamarkuptext}%
In \cite{rbs}, the term $\mathcal{I}(p) \cap (X \cup \{\tau\}) = \emptyset$ is used a lot, which expresses that there are no immediate transitions the process $p$ can perform (i.e.\@ it is idle) in environments~$X$.%
\end{isamarkuptext}\isamarkuptrue%
\isacommand{abbreviation}\isamarkupfalse%
\ idle\ {\isacharcolon}{\kern0pt}{\isacharcolon}{\kern0pt}\ {\isacartoucheopen}{\isacharprime}{\kern0pt}s\ {\isasymRightarrow}\ {\isacharprime}{\kern0pt}a\ set\ {\isasymRightarrow}\ bool{\isacartoucheclose}\isanewline
\ \ \isakeyword{where}\ {\isacartoucheopen}idle\ p\ X\ {\isasymequiv}\ initial{\isacharunderscore}{\kern0pt}actions{\isacharparenleft}{\kern0pt}p{\isacharparenright}{\kern0pt}\ {\isasyminter}\ {\isacharparenleft}{\kern0pt}X{\isasymunion}{\isacharbraceleft}{\kern0pt}{\isasymtau}{\isacharbraceright}{\kern0pt}{\isacharparenright}{\kern0pt}\ {\isacharequal}{\kern0pt}\ {\isasymemptyset}{\isacartoucheclose}%
\begin{isamarkuptext}%
\pagebreak
The following corollary is an immediate consequence of this definition.%
\end{isamarkuptext}\isamarkuptrue%
\isacommand{corollary}\isamarkupfalse%
\ idle{\isacharunderscore}{\kern0pt}no{\isacharunderscore}{\kern0pt}derivatives{\isacharcolon}{\kern0pt}\isanewline
\ \ \isakeyword{assumes}\ \isanewline
\ \ \ \ {\isacartoucheopen}idle\ p\ X{\isacartoucheclose}\ \isanewline
\ \ \ \ {\isacartoucheopen}X\ {\isasymsubseteq}\ visible{\isacharunderscore}{\kern0pt}actions{\isacartoucheclose}\isanewline
\ \ \ \ {\isacartoucheopen}{\isasymalpha}\ {\isasymin}\ X{\isasymunion}{\isacharbraceleft}{\kern0pt}{\isasymtau}{\isacharbraceright}{\kern0pt}{\isacartoucheclose}\isanewline
\ \ \isakeyword{shows}\isanewline
\ \ \ \ {\isacartoucheopen}{\isasymnexists}\ p{\isacharprime}{\kern0pt}{\isachardot}{\kern0pt}\ p\ {\isasymlongmapsto}{\isasymalpha}\ p{\isacharprime}{\kern0pt}{\isacartoucheclose}\isanewline
%
\isadelimproof
%
\endisadelimproof
%
\isatagproof
\isacommand{proof}\isamarkupfalse%
\ {\isacharparenleft}{\kern0pt}rule\ ccontr{\isacharcomma}{\kern0pt}\ auto{\isacharparenright}{\kern0pt}\isanewline
\ \ \isacommand{fix}\isamarkupfalse%
\ p{\isacharprime}{\kern0pt}\isanewline
\ \ \isacommand{assume}\isamarkupfalse%
\ {\isacartoucheopen}p\ {\isasymlongmapsto}{\isasymalpha}\ p{\isacharprime}{\kern0pt}{\isacartoucheclose}\isanewline
\ \ \isacommand{with}\isamarkupfalse%
\ assms{\isacharparenleft}{\kern0pt}{\isadigit{2}}{\isacharcomma}{\kern0pt}{\isadigit{3}}{\isacharparenright}{\kern0pt}\ \isacommand{have}\isamarkupfalse%
\ {\isacartoucheopen}{\isasymalpha}\ {\isasymin}\ initial{\isacharunderscore}{\kern0pt}actions{\isacharparenleft}{\kern0pt}p{\isacharparenright}{\kern0pt}{\isacartoucheclose}\isanewline
\ \ \ \ \isacommand{unfolding}\isamarkupfalse%
\ initial{\isacharunderscore}{\kern0pt}actions{\isacharunderscore}{\kern0pt}def\ \isacommand{by}\isamarkupfalse%
\ auto\isanewline
\ \ \isacommand{with}\isamarkupfalse%
\ assms{\isacharparenleft}{\kern0pt}{\isadigit{3}}{\isacharparenright}{\kern0pt}\ \isacommand{have}\isamarkupfalse%
\ {\isacartoucheopen}{\isasymnot}\ idle\ p\ X{\isacartoucheclose}\ \isacommand{by}\isamarkupfalse%
\ blast\ \isanewline
\ \ \isacommand{with}\isamarkupfalse%
\ assms{\isacharparenleft}{\kern0pt}{\isadigit{1}}{\isacharparenright}{\kern0pt}\ \isacommand{show}\isamarkupfalse%
\ False\ \isacommand{by}\isamarkupfalse%
\ simp\isanewline
\isacommand{qed}\isamarkupfalse%
%
\endisatagproof
{\isafoldproof}%
%
\isadelimproof
\isanewline
%
\endisadelimproof
\isanewline
\isacommand{end}\isamarkupfalse%
\ %
\isamarkupcmt{of \isa{locale\ lts{\isacharunderscore}{\kern0pt}timeout}%
}%
\isadelimtheory
%
\endisadelimtheory
%
\isatagtheory
%
\endisatagtheory
{\isafoldtheory}%
%
\isadelimtheory
%
\endisadelimtheory
%
\end{isabellebody}%
\endinput
%:%file=~/projects/Reducing-Reactive-to-Strong-Bisimilarity/isabelle/Labelled_Transition_Systems_with_TimeOuts.thy%:%
%:%24=7%:%
%:%36=8%:%
%:%40=10%:%
%:%41=11%:%
%:%42=12%:%
%:%43=13%:%
%:%44=14%:%
%:%45=15%:%
%:%46=16%:%
%:%47=17%:%
%:%48=18%:%
%:%49=19%:%
%:%50=20%:%
%:%51=21%:%
%:%52=22%:%
%:%53=23%:%
%:%54=24%:%
%:%55=25%:%
%:%56=26%:%
%:%57=27%:%
%:%58=28%:%
%:%59=29%:%
%:%60=30%:%
%:%61=31%:%
%:%62=32%:%
%:%63=33%:%
%:%64=34%:%
%:%65=35%:%
%:%66=36%:%
%:%67=37%:%
%:%68=38%:%
%:%69=39%:%
%:%70=40%:%
%:%71=41%:%
%:%72=42%:%
%:%73=43%:%
%:%74=44%:%
%:%75=45%:%
%:%76=46%:%
%:%77=47%:%
%:%78=48%:%
%:%79=49%:%
%:%80=50%:%
%:%81=51%:%
%:%82=52%:%
%:%83=53%:%
%:%84=54%:%
%:%85=55%:%
%:%86=56%:%
%:%87=57%:%
%:%88=58%:%
%:%89=59%:%
%:%90=60%:%
%:%91=61%:%
%:%92=62%:%
%:%93=63%:%
%:%94=64%:%
%:%95=65%:%
%:%96=66%:%
%:%97=67%:%
%:%98=68%:%
%:%99=69%:%
%:%100=70%:%
%:%109=72%:%
%:%121=74%:%
%:%130=76%:%
%:%142=78%:%
%:%143=79%:%
%:%144=80%:%
%:%145=81%:%
%:%146=82%:%
%:%147=83%:%
%:%148=84%:%
%:%149=85%:%
%:%150=86%:%
%:%151=87%:%
%:%152=88%:%
%:%153=89%:%
%:%154=90%:%
%:%155=91%:%
%:%156=92%:%
%:%157=93%:%
%:%158=94%:%
%:%159=95%:%
%:%160=96%:%
%:%161=97%:%
%:%162=98%:%
%:%163=99%:%
%:%164=100%:%
%:%165=101%:%
%:%166=102%:%
%:%167=103%:%
%:%168=104%:%
%:%169=105%:%
%:%170=106%:%
%:%171=107%:%
%:%172=108%:%
%:%173=109%:%
%:%174=110%:%
%:%175=111%:%
%:%176=112%:%
%:%177=113%:%
%:%178=114%:%
%:%179=115%:%
%:%180=116%:%
%:%181=117%:%
%:%182=118%:%
%:%183=119%:%
%:%192=122%:%
%:%204=124%:%
%:%206=126%:%
%:%207=126%:%
%:%208=127%:%
%:%209=128%:%
%:%210=129%:%
%:%211=130%:%
%:%212=131%:%
%:%213=132%:%
%:%215=134%:%
%:%217=136%:%
%:%218=136%:%
%:%219=137%:%
%:%222=140%:%
%:%224=142%:%
%:%225=142%:%
%:%226=143%:%
%:%229=146%:%
%:%231=148%:%
%:%232=148%:%
%:%233=149%:%
%:%236=152%:%
%:%238=154%:%
%:%239=154%:%
%:%240=155%:%
%:%242=157%:%
%:%243=158%:%
%:%245=160%:%
%:%246=160%:%
%:%247=161%:%
%:%248=162%:%
%:%249=163%:%
%:%250=164%:%
%:%251=165%:%
%:%252=166%:%
%:%259=167%:%
%:%260=167%:%
%:%261=168%:%
%:%262=168%:%
%:%263=169%:%
%:%264=169%:%
%:%265=170%:%
%:%266=170%:%
%:%267=170%:%
%:%268=171%:%
%:%269=171%:%
%:%270=171%:%
%:%271=172%:%
%:%272=172%:%
%:%273=172%:%
%:%274=172%:%
%:%275=173%:%
%:%276=173%:%
%:%277=173%:%
%:%278=173%:%
%:%279=174%:%
%:%285=174%:%
%:%288=175%:%
%:%289=176%:%
%:%290=176%:%
%:%291=176%:%
%
\begin{isabellebody}%
\setisabellecontext{Mapping{\isacharunderscore}{\kern0pt}for{\isacharunderscore}{\kern0pt}Formulas}%
%
\isadelimtheory
%
\endisadelimtheory
%
\isatagtheory
%
\endisatagtheory
{\isafoldtheory}%
%
\isadelimtheory
%
\endisadelimtheory
%
\isadelimdocument
%
\endisadelimdocument
%
\isatagdocument
%
\isamarkupsection{A Mapping for Formulas%
}
\isamarkuptrue%
%
\endisatagdocument
{\isafolddocument}%
%
\isadelimdocument
%
\endisadelimdocument
%
\begin{isamarkuptext}%
\label{sec:mapping_formulas}%
\end{isamarkuptext}\isamarkuptrue%
%
\begin{isamarkuptext}%
I will now introduce a mapping $\sigma(\cdot)$ that maps formulas of \HMLt{} to formulas of HML, in the context of the process mapping from \cref{sec:mapping_lts}, such that $\vartheta(p)$ satisfies $\sigma(\varphi)$ iff $p$ satisfies $\varphi$.

Again, we have $\mathbb{T} = (\Proc, \Act, \rightarrow)$ and $\mathbb{T}_\vartheta = (\Proc_\vartheta, \Act_\vartheta, \rightarrow_\vartheta)$ as defined in \cref{sec:mapping_lts}, with $A = \Act \!\setminus\! \{\tau, t\}$, and we assume that $t_\varepsilon \notin \Act$ and $\forall X \subseteq A.\; \varepsilon_X \notin \Act$.

Let $\sigma : (\text{\HMLt{} formulas}) \longrightarrow (\text{HML formulas})$ be recursively defined by
\begin{align*}
    \sigma(\textstyle\bigwedge_{i \in I} \varphi_i) =\;& \textstyle\bigwedge_{i \in I} \sigma(\varphi_i) &\\
    \sigma(\neg\varphi) =\;& \neg\,\sigma(\varphi)\\
    \sigma(\langle\tau\rangle\varphi) =\;& \langle\tau\rangle\,\sigma(\varphi)\\
    \sigma(\langle\alpha\rangle\varphi) =\;& 
        \langle\alpha\rangle\,\sigma(\varphi)\;\vee\\
        &\langle\varepsilon_A\rangle\langle\alpha\rangle\,\sigma(\varphi)\;\vee\\ 
        &\langle{}t_\varepsilon\rangle\langle\varepsilon_A\rangle\langle\alpha\rangle\,\sigma(\varphi) && \text{if $\alpha \in A$}\\
    \sigma(\langle\alpha\rangle\varphi) =\;& f\!\!f && \text{if $\alpha \notin A \cup \{\tau\}$}\\
    \sigma(\langle{}X\rangle\varphi) =\;&
        \langle\varepsilon_X\rangle\langle{}t\rangle\,\sigma(\varphi)\;\vee\\
        &\langle{}t_\varepsilon\rangle\langle\varepsilon_X\rangle\langle{}t\rangle\,\sigma(\varphi) && \text{if $X \subseteq A$} \\
    \sigma(\langle{}X\rangle\varphi) =\;& f\!\!f && \text{if $X \not\subseteq A$}
\end{align*}

This mapping simply expresses the time-out semantics given by the satisfaction relations of \HMLt{} (\cref{sec:HMLt}) in terms of ordinary HML evaluated on our mapped LTS $\mathbb{T}_\vartheta$. The disjunctive clauses compensate for the additional environment transitions ($\varepsilon$-actions) that are not present in $\mathbb{T}$.
\pagebreak%
\end{isamarkuptext}\isamarkuptrue%
%
\isadelimdocument
%
\endisadelimdocument
%
\isatagdocument
%
\isamarkupsubsection{Isabelle%
}
\isamarkuptrue%
%
\endisatagdocument
{\isafolddocument}%
%
\isadelimdocument
%
\endisadelimdocument
%
\begin{isamarkuptext}%
The implementation of the mapping in Isabelle is rather straightforward, although some details might not be obvious: 

\isa{cimage\ {\isacharparenleft}{\kern0pt}{\isasymlambda}\ {\isasymphi}{\isachardot}{\kern0pt}\ {\isasymsigma}{\isacharparenleft}{\kern0pt}{\isasymphi}{\isacharparenright}{\kern0pt}{\isacharparenright}{\kern0pt}\ {\isasymPhi}} is the image of the countable set \isa{{\isasymPhi}} under the function \isa{{\isasymlambda}\ {\isasymphi}{\isachardot}{\kern0pt}\ {\isasymsigma}{\isacharparenleft}{\kern0pt}{\isasymphi}{\isacharparenright}{\kern0pt}}, so it corresponds to $\{ \sigma(\varphi) \mid \varphi \in \Phi \}$ for countable $\Phi$.

\isa{{\isasymalpha}\ {\isasymnoteq}\ {\isasymtau}\ {\isasymand}\ {\isasymalpha}\ {\isasymnoteq}\ t\ {\isasymand}\ {\isasymalpha}\ {\isasymnoteq}\ t{\isacharunderscore}{\kern0pt}{\isasymepsilon}\ {\isasymand}\ {\isacharparenleft}{\kern0pt}{\isasymforall}\ X{\isachardot}{\kern0pt}\ {\isasymalpha}\ {\isasymnoteq}\ {\isasymepsilon}{\isacharbrackleft}{\kern0pt}X{\isacharbrackright}{\kern0pt}{\isacharparenright}{\kern0pt}} corresponds to $\alpha \in A$ with our assumption about there being no $\varepsilon$-actions in $\Act$. Similarly, \linebreak \isa{{\isasymalpha}\ {\isacharequal}{\kern0pt}\ t\ {\isasymor}\ {\isasymalpha}\ {\isacharequal}{\kern0pt}\ t{\isacharunderscore}{\kern0pt}{\isasymepsilon}\ {\isasymor}\ {\isasymalpha}\ {\isacharequal}{\kern0pt}\ {\isasymepsilon}{\isacharbrackleft}{\kern0pt}X{\isacharbrackright}{\kern0pt}} corresponds to $\alpha \notin A \cup \{\tau\}$.%
\end{isamarkuptext}\isamarkuptrue%
\isacommand{context}\isamarkupfalse%
\ lts{\isacharunderscore}{\kern0pt}timeout{\isacharunderscore}{\kern0pt}mappable\ \isakeyword{begin}\isanewline
\isanewline
\isacommand{function}\isamarkupfalse%
\ HMt{\isacharunderscore}{\kern0pt}mapping\ {\isacharcolon}{\kern0pt}{\isacharcolon}{\kern0pt}\ {\isacartoucheopen}{\isacharparenleft}{\kern0pt}{\isacharprime}{\kern0pt}a{\isacharparenright}{\kern0pt}HMLt{\isacharunderscore}{\kern0pt}formula\ {\isasymRightarrow}\ {\isacharparenleft}{\kern0pt}{\isacharprime}{\kern0pt}a{\isacharparenright}{\kern0pt}HML{\isacharunderscore}{\kern0pt}formula{\isacartoucheclose}\ \isanewline
\ \ {\isacharparenleft}{\kern0pt}{\isacartoucheopen}{\isasymsigma}{\isacharprime}{\kern0pt}{\isacharparenleft}{\kern0pt}{\isacharunderscore}{\kern0pt}{\isacharprime}{\kern0pt}{\isacharparenright}{\kern0pt}{\isacartoucheclose}{\isacharparenright}{\kern0pt}\isanewline
\ \ \isakeyword{where}\isanewline
\ \ \ \ {\isacartoucheopen}{\isasymsigma}{\isacharparenleft}{\kern0pt}HMLt{\isacharunderscore}{\kern0pt}conj\ {\isasymPhi}{\isacharparenright}{\kern0pt}\ {\isacharequal}{\kern0pt}\ HML{\isacharunderscore}{\kern0pt}conj\ {\isacharparenleft}{\kern0pt}cimage\ {\isacharparenleft}{\kern0pt}{\isasymlambda}\ {\isasymphi}{\isachardot}{\kern0pt}\ {\isasymsigma}{\isacharparenleft}{\kern0pt}{\isasymphi}{\isacharparenright}{\kern0pt}{\isacharparenright}{\kern0pt}\ {\isasymPhi}{\isacharparenright}{\kern0pt}{\isacartoucheclose}\isanewline
\ \ {\isacharbar}{\kern0pt}\ {\isacartoucheopen}{\isasymsigma}{\isacharparenleft}{\kern0pt}HMLt{\isacharunderscore}{\kern0pt}neg\ {\isasymphi}{\isacharparenright}{\kern0pt}\ {\isacharequal}{\kern0pt}\ HML{\isacharunderscore}{\kern0pt}neg\ {\isasymsigma}{\isacharparenleft}{\kern0pt}{\isasymphi}{\isacharparenright}{\kern0pt}{\isacartoucheclose}\isanewline
\ \ {\isacharbar}{\kern0pt}\ {\isacartoucheopen}{\isasymalpha}\ {\isacharequal}{\kern0pt}\ {\isasymtau}\ {\isasymLongrightarrow}\isanewline
\ \ \ \ \ \ {\isasymsigma}{\isacharparenleft}{\kern0pt}HMLt{\isacharunderscore}{\kern0pt}poss\ {\isasymalpha}\ {\isasymphi}{\isacharparenright}{\kern0pt}\ {\isacharequal}{\kern0pt}\ HML{\isacharunderscore}{\kern0pt}poss\ {\isasymalpha}\ {\isasymsigma}{\isacharparenleft}{\kern0pt}{\isasymphi}{\isacharparenright}{\kern0pt}{\isacartoucheclose}\isanewline
\ \ {\isacharbar}{\kern0pt}\ {\isacartoucheopen}{\isasymalpha}\ {\isasymnoteq}\ {\isasymtau}\ {\isasymand}\ {\isasymalpha}\ {\isasymnoteq}\ t\ {\isasymand}\ {\isasymalpha}\ {\isasymnoteq}\ t{\isacharunderscore}{\kern0pt}{\isasymepsilon}\ {\isasymand}\ {\isacharparenleft}{\kern0pt}{\isasymforall}\ X{\isachardot}{\kern0pt}\ {\isasymalpha}\ {\isasymnoteq}\ {\isasymepsilon}{\isacharbrackleft}{\kern0pt}X{\isacharbrackright}{\kern0pt}{\isacharparenright}{\kern0pt}\ {\isasymLongrightarrow}\isanewline
\ \ \ \ \ \ {\isasymsigma}{\isacharparenleft}{\kern0pt}HMLt{\isacharunderscore}{\kern0pt}poss\ {\isasymalpha}\ {\isasymphi}{\isacharparenright}{\kern0pt}\ {\isacharequal}{\kern0pt}\ HML{\isacharunderscore}{\kern0pt}disj\ {\isacharparenleft}{\kern0pt}acset\ {\isacharbraceleft}{\kern0pt}\isanewline
\ \ \ \ \ \ \ \ HML{\isacharunderscore}{\kern0pt}poss\ {\isasymalpha}\ {\isasymsigma}{\isacharparenleft}{\kern0pt}{\isasymphi}{\isacharparenright}{\kern0pt}{\isacharcomma}{\kern0pt}\isanewline
\ \ \ \ \ \ \ \ HML{\isacharunderscore}{\kern0pt}poss\ {\isasymepsilon}{\isacharbrackleft}{\kern0pt}visible{\isacharunderscore}{\kern0pt}actions{\isacharbrackright}{\kern0pt}\ {\isacharparenleft}{\kern0pt}HML{\isacharunderscore}{\kern0pt}poss\ {\isasymalpha}\ {\isasymsigma}{\isacharparenleft}{\kern0pt}{\isasymphi}{\isacharparenright}{\kern0pt}{\isacharparenright}{\kern0pt}{\isacharcomma}{\kern0pt}\isanewline
\ \ \ \ \ \ \ \ HML{\isacharunderscore}{\kern0pt}poss\ t{\isacharunderscore}{\kern0pt}{\isasymepsilon}\ {\isacharparenleft}{\kern0pt}HML{\isacharunderscore}{\kern0pt}poss\ {\isasymepsilon}{\isacharbrackleft}{\kern0pt}visible{\isacharunderscore}{\kern0pt}actions{\isacharbrackright}{\kern0pt}\ {\isacharparenleft}{\kern0pt}HML{\isacharunderscore}{\kern0pt}poss\ {\isasymalpha}\ {\isasymsigma}{\isacharparenleft}{\kern0pt}{\isasymphi}{\isacharparenright}{\kern0pt}{\isacharparenright}{\kern0pt}{\isacharparenright}{\kern0pt}\isanewline
\ \ \ \ \ \ {\isacharbraceright}{\kern0pt}{\isacharparenright}{\kern0pt}{\isacartoucheclose}\isanewline
\ \ {\isacharbar}{\kern0pt}\ {\isacartoucheopen}{\isasymalpha}\ {\isacharequal}{\kern0pt}\ t\ {\isasymor}\ {\isasymalpha}\ {\isacharequal}{\kern0pt}\ t{\isacharunderscore}{\kern0pt}{\isasymepsilon}\ {\isasymor}\ {\isasymalpha}\ {\isacharequal}{\kern0pt}\ {\isasymepsilon}{\isacharbrackleft}{\kern0pt}X{\isacharbrackright}{\kern0pt}\ {\isasymLongrightarrow}\isanewline
\ \ \ \ \ \ {\isasymsigma}{\isacharparenleft}{\kern0pt}HMLt{\isacharunderscore}{\kern0pt}poss\ {\isasymalpha}\ {\isasymphi}{\isacharparenright}{\kern0pt}\ {\isacharequal}{\kern0pt}\ HML{\isacharunderscore}{\kern0pt}false{\isacartoucheclose}\isanewline
\ \ {\isacharbar}{\kern0pt}\ {\isacartoucheopen}X\ {\isasymsubseteq}\ visible{\isacharunderscore}{\kern0pt}actions\ {\isasymLongrightarrow}\isanewline
\ \ \ \ \ \ {\isasymsigma}{\isacharparenleft}{\kern0pt}HMLt{\isacharunderscore}{\kern0pt}time\ X\ {\isasymphi}{\isacharparenright}{\kern0pt}\ {\isacharequal}{\kern0pt}\ HML{\isacharunderscore}{\kern0pt}disj\ {\isacharparenleft}{\kern0pt}acset\ {\isacharbraceleft}{\kern0pt}\isanewline
\ \ \ \ \ \ \ \ HML{\isacharunderscore}{\kern0pt}poss\ {\isasymepsilon}{\isacharbrackleft}{\kern0pt}X{\isacharbrackright}{\kern0pt}\ {\isacharparenleft}{\kern0pt}HML{\isacharunderscore}{\kern0pt}poss\ t\ {\isasymsigma}{\isacharparenleft}{\kern0pt}{\isasymphi}{\isacharparenright}{\kern0pt}{\isacharparenright}{\kern0pt}{\isacharcomma}{\kern0pt}\isanewline
\ \ \ \ \ \ \ \ HML{\isacharunderscore}{\kern0pt}poss\ t{\isacharunderscore}{\kern0pt}{\isasymepsilon}\ {\isacharparenleft}{\kern0pt}HML{\isacharunderscore}{\kern0pt}poss\ {\isasymepsilon}{\isacharbrackleft}{\kern0pt}X{\isacharbrackright}{\kern0pt}\ {\isacharparenleft}{\kern0pt}HML{\isacharunderscore}{\kern0pt}poss\ t\ {\isasymsigma}{\isacharparenleft}{\kern0pt}{\isasymphi}{\isacharparenright}{\kern0pt}{\isacharparenright}{\kern0pt}{\isacharparenright}{\kern0pt}\isanewline
\ \ \ \ \ \ {\isacharbraceright}{\kern0pt}{\isacharparenright}{\kern0pt}{\isacartoucheclose}\isanewline
\ \ {\isacharbar}{\kern0pt}\ {\isacartoucheopen}{\isasymnot}\ X\ {\isasymsubseteq}\ visible{\isacharunderscore}{\kern0pt}actions\ {\isasymLongrightarrow}\isanewline
\ \ \ \ \ \ {\isasymsigma}{\isacharparenleft}{\kern0pt}HMLt{\isacharunderscore}{\kern0pt}time\ X\ {\isasymphi}{\isacharparenright}{\kern0pt}\ {\isacharequal}{\kern0pt}\ HML{\isacharunderscore}{\kern0pt}false{\isacartoucheclose}\ \ \isanewline
%
\isadelimproof
\ \ %
\endisadelimproof
%
\isatagproof
\isacommand{by}\isamarkupfalse%
\ {\isacharparenleft}{\kern0pt}metis\ HMLt{\isacharunderscore}{\kern0pt}formula{\isachardot}{\kern0pt}exhaust{\isacharcomma}{\kern0pt}\ auto{\isacharplus}{\kern0pt}{\isacharcomma}{\kern0pt}\ {\isacharparenleft}{\kern0pt}simp\ add{\isacharcolon}{\kern0pt}\ distinctness{\isacharunderscore}{\kern0pt}special{\isacharunderscore}{\kern0pt}actions{\isacharparenleft}{\kern0pt}{\isadigit{1}}{\isacharcomma}{\kern0pt}{\isadigit{2}}{\isacharparenright}{\kern0pt}{\isacharparenright}{\kern0pt}{\isacharplus}{\kern0pt}{\isacharcomma}{\kern0pt}\ metis\ distinctness{\isacharunderscore}{\kern0pt}special{\isacharunderscore}{\kern0pt}actions{\isacharparenleft}{\kern0pt}{\isadigit{4}}{\isacharparenright}{\kern0pt}{\isacharparenright}{\kern0pt}%
\endisatagproof
{\isafoldproof}%
%
\isadelimproof
%
\endisadelimproof
%
\begin{isamarkuptext}%
Again, we show that the function terminates using a well-founded relation.%
\end{isamarkuptext}\isamarkuptrue%
\isacommand{inductive{\isacharunderscore}{\kern0pt}set}\isamarkupfalse%
\ sigma{\isacharunderscore}{\kern0pt}wf{\isacharunderscore}{\kern0pt}rel\ {\isacharcolon}{\kern0pt}{\isacharcolon}{\kern0pt}\ {\isacartoucheopen}{\isacharparenleft}{\kern0pt}{\isacharparenleft}{\kern0pt}{\isacharprime}{\kern0pt}a{\isacharparenright}{\kern0pt}HMLt{\isacharunderscore}{\kern0pt}formula{\isacharparenright}{\kern0pt}\ rel{\isacartoucheclose}\ \isanewline
\ \ \isakeyword{where}\isanewline
\ \ \ \ {\isacartoucheopen}{\isasymphi}\ {\isasymin}\isactrlsub c\ {\isasymPhi}\ {\isasymLongrightarrow}\ {\isacharparenleft}{\kern0pt}{\isasymphi}{\isacharcomma}{\kern0pt}\ HMLt{\isacharunderscore}{\kern0pt}conj\ {\isasymPhi}{\isacharparenright}{\kern0pt}\ {\isasymin}\ sigma{\isacharunderscore}{\kern0pt}wf{\isacharunderscore}{\kern0pt}rel{\isacartoucheclose}\ \isanewline
\ \ {\isacharbar}{\kern0pt}\ {\isacartoucheopen}{\isacharparenleft}{\kern0pt}{\isasymphi}{\isacharcomma}{\kern0pt}\ HMLt{\isacharunderscore}{\kern0pt}neg\ {\isasymphi}{\isacharparenright}{\kern0pt}\ {\isasymin}\ sigma{\isacharunderscore}{\kern0pt}wf{\isacharunderscore}{\kern0pt}rel{\isacartoucheclose}\ \isanewline
\ \ {\isacharbar}{\kern0pt}\ {\isacartoucheopen}{\isacharparenleft}{\kern0pt}{\isasymphi}{\isacharcomma}{\kern0pt}\ HMLt{\isacharunderscore}{\kern0pt}poss\ {\isasymalpha}\ {\isasymphi}{\isacharparenright}{\kern0pt}\ {\isasymin}\ sigma{\isacharunderscore}{\kern0pt}wf{\isacharunderscore}{\kern0pt}rel{\isacartoucheclose}\ \isanewline
\ \ {\isacharbar}{\kern0pt}\ {\isacartoucheopen}{\isacharparenleft}{\kern0pt}{\isasymphi}{\isacharcomma}{\kern0pt}\ HMLt{\isacharunderscore}{\kern0pt}time\ X\ {\isasymphi}{\isacharparenright}{\kern0pt}\ {\isasymin}\ sigma{\isacharunderscore}{\kern0pt}wf{\isacharunderscore}{\kern0pt}rel{\isacartoucheclose}\isanewline
%
\isadelimunimportant
%
\endisadelimunimportant
%
\isatagunimportant
%
\endisatagunimportant
{\isafoldunimportant}%
%
\isadelimunimportant
\isanewline
%
\endisadelimunimportant
\isanewline
\isacommand{termination}\isamarkupfalse%
\ HMt{\isacharunderscore}{\kern0pt}mapping%
\isadelimproof
\ %
\endisadelimproof
%
\isatagproof
\isacommand{using}\isamarkupfalse%
\ sigma{\isacharunderscore}{\kern0pt}wf{\isacharunderscore}{\kern0pt}rel{\isacharunderscore}{\kern0pt}is{\isacharunderscore}{\kern0pt}wf\ \isacommand{by}\isamarkupfalse%
\ {\isacharparenleft}{\kern0pt}standard{\isacharcomma}{\kern0pt}\ {\isacharparenleft}{\kern0pt}simp\ add{\isacharcolon}{\kern0pt}\ cin{\isachardot}{\kern0pt}rep{\isacharunderscore}{\kern0pt}eq\ sigma{\isacharunderscore}{\kern0pt}wf{\isacharunderscore}{\kern0pt}rel{\isachardot}{\kern0pt}intros{\isacharparenright}{\kern0pt}{\isacharplus}{\kern0pt}{\isacharparenright}{\kern0pt}%
\endisatagproof
{\isafoldproof}%
%
\isadelimproof
%
\endisadelimproof
\isanewline
\isanewline
\isacommand{end}\isamarkupfalse%
\ %
\isamarkupcmt{of \isa{context\ lts{\isacharunderscore}{\kern0pt}timeout{\isacharunderscore}{\kern0pt}mappable}%
}%
\isadelimtheory
%
\endisadelimtheory
%
\isatagtheory
%
\endisatagtheory
{\isafoldtheory}%
%
\isadelimtheory
%
\endisadelimtheory
%
\end{isabellebody}%
\endinput
%:%file=~/projects/Reducing-Reactive-to-Strong-Bisimilarity/isabelle/Mapping_for_Formulas.thy%:%
%:%24=10%:%
%:%36=11%:%
%:%40=13%:%
%:%41=14%:%
%:%42=15%:%
%:%43=16%:%
%:%44=17%:%
%:%45=18%:%
%:%46=19%:%
%:%47=20%:%
%:%48=21%:%
%:%49=22%:%
%:%50=23%:%
%:%51=24%:%
%:%52=25%:%
%:%53=26%:%
%:%54=27%:%
%:%55=28%:%
%:%56=29%:%
%:%57=30%:%
%:%58=31%:%
%:%59=32%:%
%:%60=33%:%
%:%61=34%:%
%:%70=37%:%
%:%82=39%:%
%:%83=40%:%
%:%84=41%:%
%:%85=42%:%
%:%86=43%:%
%:%88=45%:%
%:%89=45%:%
%:%90=46%:%
%:%91=47%:%
%:%92=47%:%
%:%93=48%:%
%:%94=49%:%
%:%95=50%:%
%:%96=51%:%
%:%97=52%:%
%:%98=53%:%
%:%99=54%:%
%:%104=59%:%
%:%105=60%:%
%:%106=61%:%
%:%107=62%:%
%:%111=66%:%
%:%112=67%:%
%:%113=68%:%
%:%116=69%:%
%:%120=69%:%
%:%121=69%:%
%:%130=71%:%
%:%132=73%:%
%:%133=73%:%
%:%134=74%:%
%:%135=75%:%
%:%136=76%:%
%:%137=77%:%
%:%138=78%:%
%:%150=95%:%
%:%153=96%:%
%:%154=97%:%
%:%155=97%:%
%:%157=97%:%
%:%161=97%:%
%:%162=97%:%
%:%163=97%:%
%:%170=97%:%
%:%171=98%:%
%:%172=99%:%
%:%173=99%:%
%:%174=99%:%
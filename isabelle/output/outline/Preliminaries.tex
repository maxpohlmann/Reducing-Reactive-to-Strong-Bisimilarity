%
\begin{isabellebody}%
\setisabellecontext{Preliminaries}%
%
\isadelimtheory
%
\endisadelimtheory
%
\isatagtheory
%
\endisatagtheory
{\isafoldtheory}%
%
\isadelimtheory
%
\endisadelimtheory
%
\isadelimdocument
%
\endisadelimdocument
%
\isatagdocument
%
\isamarkupchapter{Preliminaries%
}
\isamarkuptrue%
%
\endisatagdocument
{\isafolddocument}%
%
\isadelimdocument
%
\endisadelimdocument
%
\begin{isamarkuptext}%
\label{chap:preliminaries}%
\end{isamarkuptext}\isamarkuptrue%
%
\begin{isamarkuptext}%
In this section, the concepts that are relevant for the main part of this thesis will be introduced in text, as well as formalised in Isabelle. The formalisations of section x and section y are based on those done by Benjamin Bisping in \cite{bisping2018computing}; the code is available on GitHub.%
\footnote{see \code{\href{https://gitlab.tubit.tu-berlin.de/mtv/coupled-similarity}{gitlab.tubit.tu-berlin.de/mtv/coupled-similarity}}}
All other formalisations were done by me.%
\end{isamarkuptext}\isamarkuptrue%
%
\isadelimdocument
%
\endisadelimdocument
%
\isatagdocument
%
\isamarkupparagraph{Note on metavariable usage%
}
\isamarkuptrue%
%
\endisatagdocument
{\isafolddocument}%
%
\isadelimdocument
%
\endisadelimdocument
%
\begin{isamarkuptext}%
Although the concepts will only be formally introduced in the following subsections, I want to clarify my usage of metavariables upfront.

States of an LTS range over \isa{p{\isacharcomma}{\kern0pt}\ q{\isacharcomma}{\kern0pt}\ p{\isacharprime}{\kern0pt}{\isacharcomma}{\kern0pt}\ q{\isacharprime}{\kern0pt}{\isacharcomma}{\kern0pt}\ {\isachardot}{\kern0pt}{\isachardot}{\kern0pt}{\isachardot}{\kern0pt}}, where \isa{p} and \isa{p{\isacharprime}{\kern0pt}} are used for states connected by some transition (i.e.\@ \isa{p\ {\isasymlongmapsto}{\isasymalpha}\ p{\isacharprime}{\kern0pt}}), whereas \isa{p} and \isa{q} are used for states possibly related by some equivalence.

An arbitrary action of an LTS will be referenced by \isa{{\isasymalpha}}, whereas an arbitrary \emph{visible} action will be referenced by \isa{a}.%
\end{isamarkuptext}\isamarkuptrue%
%
\isadelimtheory
%
\endisadelimtheory
%
\isatagtheory
%
\endisatagtheory
{\isafoldtheory}%
%
\isadelimtheory
%
\endisadelimtheory
%
\end{isabellebody}%
\endinput
%:%file=~/reactive-bisimilarity-reduction/isabelle/Preliminaries.thy%:%
%:%24=9%:%
%:%36=10%:%
%:%40=12%:%
%:%41=13%:%
%:%42=14%:%
%:%51=16%:%
%:%63=17%:%
%:%64=18%:%
%:%65=19%:%
%:%66=20%:%
%:%67=21%:%
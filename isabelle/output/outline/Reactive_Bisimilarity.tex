%
\begin{isabellebody}%
\setisabellecontext{Reactive{\isacharunderscore}{\kern0pt}Bisimilarity}%
%
\isadelimtheory
%
\endisadelimtheory
%
\isatagtheory
%
\endisatagtheory
{\isafoldtheory}%
%
\isadelimtheory
%
\endisadelimtheory
%
\isadelimdocument
%
\endisadelimdocument
%
\isatagdocument
%
\isamarkupsection{Reactive Bisimilarity%
}
\isamarkuptrue%
%
\endisatagdocument
{\isafolddocument}%
%
\isadelimdocument
%
\endisadelimdocument
%
\begin{isamarkuptext}%
\label{sec:reactive_bisimilarity}%
\end{isamarkuptext}\isamarkuptrue%
%
\begin{isamarkuptext}%
In the examples of the previous section, we saw that there are \LTSt{}s with transitions that can never be performed or that can only be performed in certain environments. The behavioural equivalence implied hereby is defined in \cite{rbs} as \emph{strong reactive bisimilarity}.

\example{%
The processes $p$ and $q$ are behaviourally equivalent for \LTSt{} semantics, i.e.\@ strongly reactive bisimilar.

\lts{
    \node[state]    (p0)                            {$p$};
    \node[state]    (p1) [below left of=p0]         {$p_1$};
    \node[state]    (p2) [below right of=p0]        {$p_2$};
    \node[state]    (p3) [below left of=p2]         {$p_3$};
    \node[state]    (p4) [below right of=p2]        {$p_4$};
    
    \path   (p0) edge node[above left]  {$a$}   (p1)
                 edge node              {$t$}   (p2)
            (p2) edge node[above left]  {$\tau$}(p3)
                 edge node              {$a$}   (p4);
                 
    \node[state]    (q0) [right of=p0,xshift=5cm]   {$q$};
    \node[state]    (q1) [below left of=q0]         {$q_1$};
    \node[state]    (q2) [below right of=q0]        {$q_2$};
    \node[state]    (q3) [below left of=q2]         {$q_3$};
    
    \path   (q0) edge node[above left]  {$a$}   (q1)
                 edge node              {$t$}   (q2)
            (q2) edge node[above left]  {$\tau$}(q3);
}}%
\end{isamarkuptext}\isamarkuptrue%
%
\isadelimdocument
%
\endisadelimdocument
%
\isatagdocument
%
\isamarkupsubsubsection{Strong Reactive Bisimulations%
}
\isamarkuptrue%
%
\endisatagdocument
{\isafolddocument}%
%
\isadelimdocument
%
\endisadelimdocument
%
\begin{isamarkuptext}%
Van~Glabbeek introduces several characterisations of this equivalence, beginning with \emph{strong reactive bisimulation} (SRB) relations. These differ from strong bisimulations in that the relations contain not only pairs of processes, $(p,q)$, but additionally triples consisting of two processes and a set of actions, $(p,X,q)$. The following definition of SRBs is quoted, with minor adaptations, from \cite[definition 1]{rbs}:

A \emph{strong reactive bisimulation} is a symmetric relation 
$$\mathcal{R} \subseteq (\Proc \times \mathcal{P}(A) \times \Proc) \cup (\Proc \times \Proc)$$
(meaning that $(p,X,q)\!\in\!\mathcal{R}\!\iff\!(q,X,p)\!\in\!\mathcal{R}$ and
$(p,q)\!\in\!\mathcal{R}\!\iff\!(q,p)\!\in\!\mathcal{R}$),\\
such that,
\newpage
for all $(p,q) \in \mathcal{R}$:
\begin{enumerate}
    \item if $p \xrightarrow{\tau} p'$, then $\exists q'$ such that $q \xrightarrow{\tau} q'$ and $(p',q') \in \mathcal{R}$,
    \item $(p,X,q) \in \mathcal{R}$ for all $X \subseteq A$,
\end{enumerate}
and for all $(p,X,q) \in \mathcal{R}$:
\begin{enumerate}
    \setcounter{enumi}{2}
    \item if $p \xrightarrow{a} p'$ with $a \in X$, then $\exists q'$ such that $q \xrightarrow{a} q'$ and $(p',q') \in \mathcal{R}$,
    \item if $p \xrightarrow{\tau} p'$, then $\exists q'$ such that $q \xrightarrow{\tau} q'$ and $(p',X,q') \in \mathcal{R}$,
    \item if $\mathcal{I}(p) \cap (X \cup \{\tau\}) = \emptyset$, then $(p,q) \in \mathcal{R}$, and
    \item if $\mathcal{I}(p) \cap (X \cup \{\tau\}) = \emptyset$ and $p \xrightarrow{t} p'$, then $\exists q'$ such that $q \xrightarrow{t} q'$\\and $(p',X,q') \in \mathcal{R}$.
\end{enumerate}

We can derive the following intuitions: an environment can either be stable, allowing a specific set of actions, or indeterminate. Indeterminate environments cannot facilitate any transitions, but they can stabilise into arbitrary stable environments. This is expressed by clause 2. Hence, $X$-bisimilarity is behavioural equivalence in stable environments~$X$, and reactive bisimilarity is behavioural equivalence in indeterminate environments (and thus in arbitrary stable environments).

Since only stable environments can facilitate transitions, there are no clauses involving visible action transitions for $(p,q) \in \mathcal{R}$. However, $\tau$-transitions can be performed regardless of the environment, hence clause 1.

At this point, it is important to discuss what exactly it means for an action to be visible or hidden in this context: as we saw in the last section, the environment cannot react (change its set of allowed actions) when the system performs a $\tau$- or a $t$-transition, since these are hidden actions. However, since we are talking about a \emph{strong} bisimilarity (as opposed to e.g.\@ weak bisimilarity), the performance of $\tau$- or $t$-transitions is still relevant when examining and comparing the behavior of systems.

With that, we can look more closely at the remaining clauses:
in clause 3, given $(p,X,q) \in \mathcal{R}$, for $p \xrightarrow{a} p'$ with $a \in X$, we require for the \enquote{mirroring} state $q'$ that $(p',q') \in \mathcal{R}$, because $a$ is a visible action and the transition can thus trigger a change of the environment;%
\footnote{This is why van~Glabbeek talks about \emph{triggered} environments rather than indeterminate ones. I will use both terms interchangeably.}
on the other hand, in clause 4, for $p \xrightarrow{\tau} p'$, and in clause 6, for $p \xrightarrow{t} p'$, we require $(p',X,q') \in \mathcal{R}$, because these actions are hidden and cannot trigger a change of the environment.

Lastly, clause 5 formalises the possibility of the environment timing out (i.e.\@ turning into an indeterminate environment) instead of the system.

These intuitions also form the basis for the process mapping which will be presented in \cref{sec:mapping_lts}.%
\end{isamarkuptext}\isamarkuptrue%
%
\isadelimdocument
%
\endisadelimdocument
%
\isatagdocument
%
\isamarkupsubsubsection{Strong Reactive/$X$-Bisimilarity%
}
\isamarkuptrue%
%
\endisatagdocument
{\isafolddocument}%
%
\isadelimdocument
%
\endisadelimdocument
%
\begin{isamarkuptext}%
Two processes $p$ and $q$ are \emph{strongly reactive bisimilar} ($p \leftrightarrow_r q$) iff there is an SRB containing $(p,q)$, and \emph{strongly $X$-bisimilar} ($p \leftrightarrow_r^X q$), i.e.\@ equivalent in environments~$X$, when there is an SRB containing $(p,X,q)$.%
\end{isamarkuptext}\isamarkuptrue%
%
\isadelimdocument
%
\endisadelimdocument
%
\isatagdocument
%
\isamarkupsubsubsection{Generalised Strong Reactive Bisimulations%
}
\isamarkuptrue%
%
\endisatagdocument
{\isafolddocument}%
%
\isadelimdocument
%
\endisadelimdocument
%
\begin{isamarkuptext}%
Another characterisation of reactive bisimilarity uses \emph{generalised strong reactive bisimulation} (GSRB) relations \cite[definition 3]{rbs}, defined over the same set as SRBs. It is proved that both characterisations do, in fact, characterise the same equivalence. More details will be discussed in the formalisation below.%
\end{isamarkuptext}\isamarkuptrue%
%
\isadelimdocument
%
\endisadelimdocument
%
\isatagdocument
%
\isamarkupsubsection{Isabelle%
}
\isamarkuptrue%
%
\endisatagdocument
{\isafolddocument}%
%
\isadelimdocument
%
\endisadelimdocument
%
\begin{isamarkuptext}%
We first formalise both SRB and GSRB relations (as well as strong reactive bisimilarity, defined by the existence of an SRB, as above), and then replicate the proof of their correspondence.%
\end{isamarkuptext}\isamarkuptrue%
%
\isadelimdocument
%
\endisadelimdocument
%
\isatagdocument
%
\isamarkupsubsubsection{Strong Reactive Bisimulations%
}
\isamarkuptrue%
%
\endisatagdocument
{\isafolddocument}%
%
\isadelimdocument
%
\endisadelimdocument
%
\begin{isamarkuptext}%
SRB relations are defined over the set
$$(\Proc \times \mathscr{P}(A) \times \Proc) \cup (\Proc \times \Proc).$$

As can be easily seen, this set it isomorphic to
$$(\Proc \times (\mathscr{P}(A) \cup \{\bot\}) \times \Proc),$$
which is a subset of
$$(\Proc \times (\mathscr{P}(\Act) \cup \{\bot\}) \times \Proc).$$ 

This last set can now be easily formalised in terms of a type, where we formalise
$\mathscr{P}(\Act) \cup \{\bot\}$
as \isa{{\isacharprime}{\kern0pt}a\ set\ option}.

The fact that SRBs are defined using the power set of visible actions ($A$), whereas our type uses all actions ($\Act$ / \isa{{\isacharprime}{\kern0pt}a}), is handled by the first line of the definition below. The second line formalises that symmetry is required by definition. All other lines are direct formalisations of the clauses of the original definition.
\pagebreak%
\end{isamarkuptext}\isamarkuptrue%
\isacommand{context}\isamarkupfalse%
\ lts{\isacharunderscore}{\kern0pt}timeout\ \isakeyword{begin}\isanewline
\isanewline
%
\isamarkupcmt{strong reactive bisimulation \cite[definition 1]{rbs}%
}\isanewline
\isacommand{definition}\isamarkupfalse%
\ SRB\ {\isacharcolon}{\kern0pt}{\isacharcolon}{\kern0pt}\ {\isacartoucheopen}{\isacharparenleft}{\kern0pt}{\isacharprime}{\kern0pt}s\ {\isasymRightarrow}\ {\isacharprime}{\kern0pt}a\ set\ option\ {\isasymRightarrow}\ {\isacharprime}{\kern0pt}s\ {\isasymRightarrow}\ bool{\isacharparenright}{\kern0pt}\ {\isasymRightarrow}\ bool{\isacartoucheclose}\isanewline
\ \ \isakeyword{where}\ {\isacartoucheopen}SRB\ R\ {\isasymequiv}\isanewline
\ \ {\isacharparenleft}{\kern0pt}{\isasymforall}\ p\ X\ q{\isachardot}{\kern0pt}\ R\ p\ {\isacharparenleft}{\kern0pt}Some\ X{\isacharparenright}{\kern0pt}\ q\ \ {\isasymlongrightarrow}\ \ X\ {\isasymsubseteq}\ visible{\isacharunderscore}{\kern0pt}actions{\isacharparenright}{\kern0pt}\ {\isasymand}\isanewline
\ \ {\isacharparenleft}{\kern0pt}{\isasymforall}\ p\ XoN\ q{\isachardot}{\kern0pt}\ R\ p\ XoN\ q\ \ {\isasymlongrightarrow}\ \ R\ q\ XoN\ p{\isacharparenright}{\kern0pt}\ {\isasymand}\isanewline
\isanewline
\ \ {\isacharparenleft}{\kern0pt}{\isasymforall}\ p\ q{\isachardot}{\kern0pt}\ R\ p\ None\ q\ {\isasymlongrightarrow}\isanewline
\ \ \ \ {\isacharparenleft}{\kern0pt}{\isasymforall}\ p{\isacharprime}{\kern0pt}{\isachardot}{\kern0pt}\ p\ {\isasymlongmapsto}{\isasymtau}\ p{\isacharprime}{\kern0pt}\ \ {\isasymlongrightarrow}\ \ {\isacharparenleft}{\kern0pt}{\isasymexists}\ q{\isacharprime}{\kern0pt}{\isachardot}{\kern0pt}\ {\isacharparenleft}{\kern0pt}q\ {\isasymlongmapsto}{\isasymtau}\ q{\isacharprime}{\kern0pt}{\isacharparenright}{\kern0pt}\ {\isasymand}\ R\ p{\isacharprime}{\kern0pt}\ None\ q{\isacharprime}{\kern0pt}{\isacharparenright}{\kern0pt}{\isacharparenright}{\kern0pt}\ {\isasymand}\isanewline
\ \ \ \ {\isacharparenleft}{\kern0pt}{\isasymforall}\ X\ {\isasymsubseteq}\ visible{\isacharunderscore}{\kern0pt}actions{\isachardot}{\kern0pt}\ {\isacharparenleft}{\kern0pt}R\ p\ {\isacharparenleft}{\kern0pt}Some\ X{\isacharparenright}{\kern0pt}\ q{\isacharparenright}{\kern0pt}{\isacharparenright}{\kern0pt}{\isacharparenright}{\kern0pt}\ {\isasymand}\isanewline
\isanewline
\ \ {\isacharparenleft}{\kern0pt}{\isasymforall}\ p\ X\ q{\isachardot}{\kern0pt}\ R\ p\ {\isacharparenleft}{\kern0pt}Some\ X{\isacharparenright}{\kern0pt}\ q\ {\isasymlongrightarrow}\isanewline
\ \ \ \ {\isacharparenleft}{\kern0pt}{\isasymforall}\ p{\isacharprime}{\kern0pt}\ a{\isachardot}{\kern0pt}\ p\ {\isasymlongmapsto}a\ p{\isacharprime}{\kern0pt}\ {\isasymand}\ a\ {\isasymin}\ X\ \ {\isasymlongrightarrow}\ \ {\isacharparenleft}{\kern0pt}{\isasymexists}\ q{\isacharprime}{\kern0pt}{\isachardot}{\kern0pt}\ {\isacharparenleft}{\kern0pt}q\ {\isasymlongmapsto}a\ q{\isacharprime}{\kern0pt}{\isacharparenright}{\kern0pt}\ {\isasymand}\ \isanewline
\ \ \ \ \ \ R\ p{\isacharprime}{\kern0pt}\ None\ q{\isacharprime}{\kern0pt}{\isacharparenright}{\kern0pt}{\isacharparenright}{\kern0pt}\ {\isasymand}\isanewline
\ \ \ \ {\isacharparenleft}{\kern0pt}{\isasymforall}\ p{\isacharprime}{\kern0pt}{\isachardot}{\kern0pt}\ p\ {\isasymlongmapsto}{\isasymtau}\ p{\isacharprime}{\kern0pt}\ \ {\isasymlongrightarrow}\ \ {\isacharparenleft}{\kern0pt}{\isasymexists}\ q{\isacharprime}{\kern0pt}{\isachardot}{\kern0pt}\ {\isacharparenleft}{\kern0pt}q\ {\isasymlongmapsto}{\isasymtau}\ q{\isacharprime}{\kern0pt}{\isacharparenright}{\kern0pt}\ {\isasymand}\ R\ p{\isacharprime}{\kern0pt}\ {\isacharparenleft}{\kern0pt}Some\ X{\isacharparenright}{\kern0pt}\ q{\isacharprime}{\kern0pt}{\isacharparenright}{\kern0pt}{\isacharparenright}{\kern0pt}\ {\isasymand}\isanewline
\ \ \ \ {\isacharparenleft}{\kern0pt}idle\ p\ X\ \ {\isasymlongrightarrow}\ \ R\ p\ None\ q{\isacharparenright}{\kern0pt}\ {\isasymand}\isanewline
\ \ \ \ {\isacharparenleft}{\kern0pt}{\isasymforall}\ p{\isacharprime}{\kern0pt}{\isachardot}{\kern0pt}\ idle\ p\ X\ {\isasymand}\ {\isacharparenleft}{\kern0pt}p\ {\isasymlongmapsto}t\ p{\isacharprime}{\kern0pt}{\isacharparenright}{\kern0pt}\ \ {\isasymlongrightarrow}\ \ {\isacharparenleft}{\kern0pt}{\isasymexists}\ q{\isacharprime}{\kern0pt}{\isachardot}{\kern0pt}\ q\ {\isasymlongmapsto}t\ q{\isacharprime}{\kern0pt}\ {\isasymand}\ \isanewline
\ \ \ \ \ \ R\ p{\isacharprime}{\kern0pt}\ {\isacharparenleft}{\kern0pt}Some\ X{\isacharparenright}{\kern0pt}\ q{\isacharprime}{\kern0pt}{\isacharparenright}{\kern0pt}{\isacharparenright}{\kern0pt}{\isacharparenright}{\kern0pt}{\isacartoucheclose}\isanewline
\isanewline
%
\isadelimunimportant
%
\endisadelimunimportant
%
\isatagunimportant
%
\endisatagunimportant
{\isafoldunimportant}%
%
\isadelimunimportant
%
\endisadelimunimportant
%
\isadelimdocument
%
\endisadelimdocument
%
\isatagdocument
%
\isamarkupsubsubsection{Strong Reactive/$X$-Bisimilarity%
}
\isamarkuptrue%
%
\endisatagdocument
{\isafolddocument}%
%
\isadelimdocument
%
\endisadelimdocument
%
\begin{isamarkuptext}%
Van~Glabbeek differentiates between strong reactive bisimilarity ($(p,q) \in \mathcal{R}$ for an SRB $\mathcal{R}$) and strong $X$-bisimilarity ($(p,X,q) \in \mathcal{R}$ for an SRB $\mathcal{R}$).%
\end{isamarkuptext}\isamarkuptrue%
\isacommand{definition}\isamarkupfalse%
\ strongly{\isacharunderscore}{\kern0pt}reactive{\isacharunderscore}{\kern0pt}bisimilar\ {\isacharcolon}{\kern0pt}{\isacharcolon}{\kern0pt}\ {\isacartoucheopen}{\isacharprime}{\kern0pt}s\ {\isasymRightarrow}\ {\isacharprime}{\kern0pt}s\ {\isasymRightarrow}\ bool{\isacartoucheclose}\ \isanewline
\ \ {\isacharparenleft}{\kern0pt}{\isacartoucheopen}{\isacharunderscore}{\kern0pt}\ {\isasymleftrightarrow}\isactrlsub r\ {\isacharunderscore}{\kern0pt}{\isacartoucheclose}\ {\isacharbrackleft}{\kern0pt}{\isadigit{7}}{\isadigit{0}}{\isacharcomma}{\kern0pt}\ {\isadigit{7}}{\isadigit{0}}{\isacharbrackright}{\kern0pt}\ {\isadigit{7}}{\isadigit{0}}{\isacharparenright}{\kern0pt}\isanewline
\ \ \isakeyword{where}\ {\isacartoucheopen}p\ {\isasymleftrightarrow}\isactrlsub r\ q\ {\isasymequiv}\ {\isasymexists}\ R{\isachardot}{\kern0pt}\ SRB\ R\ {\isasymand}\ R\ p\ None\ q{\isacartoucheclose}\isanewline
\isanewline
\isacommand{definition}\isamarkupfalse%
\ strongly{\isacharunderscore}{\kern0pt}X{\isacharunderscore}{\kern0pt}bisimilar\ {\isacharcolon}{\kern0pt}{\isacharcolon}{\kern0pt}\ {\isacartoucheopen}{\isacharprime}{\kern0pt}s\ {\isasymRightarrow}\ {\isacharprime}{\kern0pt}a\ set\ {\isasymRightarrow}\ {\isacharprime}{\kern0pt}s\ {\isasymRightarrow}\ bool{\isacartoucheclose}\ \isanewline
\ \ {\isacharparenleft}{\kern0pt}{\isacartoucheopen}{\isacharunderscore}{\kern0pt}\ {\isasymleftrightarrow}\isactrlsub r\isactrlsup {\isacharunderscore}{\kern0pt}\ {\isacharunderscore}{\kern0pt}{\isacartoucheclose}\ {\isacharbrackleft}{\kern0pt}{\isadigit{7}}{\isadigit{0}}{\isacharcomma}{\kern0pt}\ {\isadigit{7}}{\isadigit{0}}{\isacharcomma}{\kern0pt}\ {\isadigit{7}}{\isadigit{0}}{\isacharbrackright}{\kern0pt}\ {\isadigit{7}}{\isadigit{0}}{\isacharparenright}{\kern0pt}\isanewline
\ \ \isakeyword{where}\ {\isacartoucheopen}p\ {\isasymleftrightarrow}\isactrlsub r\isactrlsup X\ q\ {\isasymequiv}\ {\isasymexists}\ R{\isachardot}{\kern0pt}\ SRB\ R\ {\isasymand}\ R\ p\ {\isacharparenleft}{\kern0pt}Some\ X{\isacharparenright}{\kern0pt}\ q{\isacartoucheclose}%
\begin{isamarkuptext}%
For the upcoming proofs, it is useful to combine both reactive and $X$-bisimilarity into a single relation.%
\end{isamarkuptext}\isamarkuptrue%
\isacommand{definition}\isamarkupfalse%
\ strongly{\isacharunderscore}{\kern0pt}reactive{\isacharunderscore}{\kern0pt}or{\isacharunderscore}{\kern0pt}X{\isacharunderscore}{\kern0pt}bisimilar\ \isanewline
\ \ {\isacharcolon}{\kern0pt}{\isacharcolon}{\kern0pt}\ {\isacartoucheopen}{\isacharprime}{\kern0pt}s\ {\isasymRightarrow}\ {\isacharprime}{\kern0pt}a\ set\ option\ {\isasymRightarrow}\ {\isacharprime}{\kern0pt}s\ {\isasymRightarrow}\ bool{\isacartoucheclose}\isanewline
\ \ \isakeyword{where}\ {\isacartoucheopen}strongly{\isacharunderscore}{\kern0pt}reactive{\isacharunderscore}{\kern0pt}or{\isacharunderscore}{\kern0pt}X{\isacharunderscore}{\kern0pt}bisimilar\ p\ XoN\ q\ \isanewline
\ \ \ \ {\isasymequiv}\ {\isasymexists}\ R{\isachardot}{\kern0pt}\ SRB\ R\ {\isasymand}\ R\ p\ XoN\ q{\isacartoucheclose}%
\begin{isamarkuptext}%
Obviously, then, these relations coincide accordingly.%
\end{isamarkuptext}\isamarkuptrue%
\isacommand{corollary}\isamarkupfalse%
\ {\isacartoucheopen}p\ {\isasymleftrightarrow}\isactrlsub r\ q\ {\isasymLongleftrightarrow}\ strongly{\isacharunderscore}{\kern0pt}reactive{\isacharunderscore}{\kern0pt}or{\isacharunderscore}{\kern0pt}X{\isacharunderscore}{\kern0pt}bisimilar\ p\ None\ q{\isacartoucheclose}\isanewline
%
\isadelimproof
\ \ %
\endisadelimproof
%
\isatagproof
\isacommand{using}\isamarkupfalse%
\ strongly{\isacharunderscore}{\kern0pt}reactive{\isacharunderscore}{\kern0pt}bisimilar{\isacharunderscore}{\kern0pt}def\ strongly{\isacharunderscore}{\kern0pt}reactive{\isacharunderscore}{\kern0pt}or{\isacharunderscore}{\kern0pt}X{\isacharunderscore}{\kern0pt}bisimilar{\isacharunderscore}{\kern0pt}def\ \isacommand{by}\isamarkupfalse%
\ force%
\endisatagproof
{\isafoldproof}%
%
\isadelimproof
\isanewline
%
\endisadelimproof
\isacommand{corollary}\isamarkupfalse%
\ {\isacartoucheopen}p\ {\isasymleftrightarrow}\isactrlsub r\isactrlsup X\ q\ {\isasymLongleftrightarrow}\ strongly{\isacharunderscore}{\kern0pt}reactive{\isacharunderscore}{\kern0pt}or{\isacharunderscore}{\kern0pt}X{\isacharunderscore}{\kern0pt}bisimilar\ p\ {\isacharparenleft}{\kern0pt}Some\ X{\isacharparenright}{\kern0pt}\ q{\isacartoucheclose}\isanewline
%
\isadelimproof
\ \ %
\endisadelimproof
%
\isatagproof
\isacommand{using}\isamarkupfalse%
\ strongly{\isacharunderscore}{\kern0pt}X{\isacharunderscore}{\kern0pt}bisimilar{\isacharunderscore}{\kern0pt}def\ strongly{\isacharunderscore}{\kern0pt}reactive{\isacharunderscore}{\kern0pt}or{\isacharunderscore}{\kern0pt}X{\isacharunderscore}{\kern0pt}bisimilar{\isacharunderscore}{\kern0pt}def\ \isacommand{by}\isamarkupfalse%
\ force%
\endisatagproof
{\isafoldproof}%
%
\isadelimproof
%
\endisadelimproof
%
\isadelimdocument
%
\endisadelimdocument
%
\isatagdocument
%
\isamarkupsubsubsection{Generalised Strong Reactive Bisimulations%
}
\isamarkuptrue%
%
\endisatagdocument
{\isafolddocument}%
%
\isadelimdocument
%
\endisadelimdocument
%
\begin{isamarkuptext}%
Since GSRBs are defined over the same set as SRBs, the same considerations concerning the type and the clauses of the definition as above hold.%
\end{isamarkuptext}\isamarkuptrue%
%
\isamarkupcmt{generalised strong reactive bisimulation \cite[definition 3]{rbs}%
}\isanewline
\isacommand{definition}\isamarkupfalse%
\ GSRB\ {\isacharcolon}{\kern0pt}{\isacharcolon}{\kern0pt}\ {\isacartoucheopen}{\isacharparenleft}{\kern0pt}{\isacharprime}{\kern0pt}s\ {\isasymRightarrow}\ {\isacharprime}{\kern0pt}a\ set\ option\ {\isasymRightarrow}\ {\isacharprime}{\kern0pt}s\ {\isasymRightarrow}\ bool{\isacharparenright}{\kern0pt}\ {\isasymRightarrow}\ bool{\isacartoucheclose}\isanewline
\ \ \isakeyword{where}\ {\isacartoucheopen}GSRB\ R\ {\isasymequiv}\isanewline
\ \ \ \ {\isacharparenleft}{\kern0pt}{\isasymforall}\ p\ X\ q{\isachardot}{\kern0pt}\ R\ p\ {\isacharparenleft}{\kern0pt}Some\ X{\isacharparenright}{\kern0pt}\ q\ \ {\isasymlongrightarrow}\ \ X\ {\isasymsubseteq}\ visible{\isacharunderscore}{\kern0pt}actions{\isacharparenright}{\kern0pt}\ {\isasymand}\isanewline
\ \ \ \ {\isacharparenleft}{\kern0pt}{\isasymforall}\ p\ XoN\ q{\isachardot}{\kern0pt}\ R\ p\ XoN\ q\ \ {\isasymlongrightarrow}\ \ R\ q\ XoN\ p{\isacharparenright}{\kern0pt}\ {\isasymand}\isanewline
\isanewline
\ \ \ \ {\isacharparenleft}{\kern0pt}{\isasymforall}\ p\ q{\isachardot}{\kern0pt}\ R\ p\ None\ q\ {\isasymlongrightarrow}\isanewline
\ \ \ \ \ \ {\isacharparenleft}{\kern0pt}{\isasymforall}\ p{\isacharprime}{\kern0pt}\ a{\isachardot}{\kern0pt}\ p\ {\isasymlongmapsto}a\ p{\isacharprime}{\kern0pt}\ {\isasymand}\ a\ {\isasymin}\ visible{\isacharunderscore}{\kern0pt}actions\ {\isasymunion}\ {\isacharbraceleft}{\kern0pt}{\isasymtau}{\isacharbraceright}{\kern0pt}\ \ {\isasymlongrightarrow}\isanewline
\ \ \ \ \ \ \ \ {\isacharparenleft}{\kern0pt}{\isasymexists}\ q{\isacharprime}{\kern0pt}{\isachardot}{\kern0pt}\ q\ {\isasymlongmapsto}a\ q{\isacharprime}{\kern0pt}\ {\isasymand}\ R\ p{\isacharprime}{\kern0pt}\ None\ q{\isacharprime}{\kern0pt}{\isacharparenright}{\kern0pt}{\isacharparenright}{\kern0pt}\ {\isasymand}\isanewline
\ \ \ \ \ \ {\isacharparenleft}{\kern0pt}{\isasymforall}\ X\ p{\isacharprime}{\kern0pt}{\isachardot}{\kern0pt}\ idle\ p\ X\ {\isasymand}\ X\ {\isasymsubseteq}\ visible{\isacharunderscore}{\kern0pt}actions\ {\isasymand}\ p\ {\isasymlongmapsto}t\ p{\isacharprime}{\kern0pt}\ \ {\isasymlongrightarrow}\ \ \isanewline
\ \ \ \ \ \ \ \ {\isacharparenleft}{\kern0pt}{\isasymexists}\ q{\isacharprime}{\kern0pt}{\isachardot}{\kern0pt}\ q\ {\isasymlongmapsto}t\ q{\isacharprime}{\kern0pt}\ {\isasymand}\ R\ p{\isacharprime}{\kern0pt}\ {\isacharparenleft}{\kern0pt}Some\ X{\isacharparenright}{\kern0pt}\ q{\isacharprime}{\kern0pt}{\isacharparenright}{\kern0pt}{\isacharparenright}{\kern0pt}{\isacharparenright}{\kern0pt}\ {\isasymand}\isanewline
\ \ \ \ \isanewline
\ \ \ \ {\isacharparenleft}{\kern0pt}{\isasymforall}\ p\ Y\ q{\isachardot}{\kern0pt}\ R\ p\ {\isacharparenleft}{\kern0pt}Some\ Y{\isacharparenright}{\kern0pt}\ q\ {\isasymlongrightarrow}\isanewline
\ \ \ \ \ \ {\isacharparenleft}{\kern0pt}{\isasymforall}\ p{\isacharprime}{\kern0pt}\ a{\isachardot}{\kern0pt}\ a\ {\isasymin}\ visible{\isacharunderscore}{\kern0pt}actions\ {\isasymand}\ p\ {\isasymlongmapsto}a\ p{\isacharprime}{\kern0pt}\ {\isasymand}\ {\isacharparenleft}{\kern0pt}a{\isasymin}Y\ {\isasymor}\ idle\ p\ Y{\isacharparenright}{\kern0pt}\ {\isasymlongrightarrow}\isanewline
\ \ \ \ \ \ \ \ {\isacharparenleft}{\kern0pt}{\isasymexists}\ q{\isacharprime}{\kern0pt}{\isachardot}{\kern0pt}\ q\ {\isasymlongmapsto}a\ q{\isacharprime}{\kern0pt}\ {\isasymand}\ R\ p{\isacharprime}{\kern0pt}\ None\ q{\isacharprime}{\kern0pt}{\isacharparenright}{\kern0pt}{\isacharparenright}{\kern0pt}\ {\isasymand}\isanewline
\ \ \ \ \ \ {\isacharparenleft}{\kern0pt}{\isasymforall}\ p{\isacharprime}{\kern0pt}{\isachardot}{\kern0pt}\ p\ {\isasymlongmapsto}{\isasymtau}\ p{\isacharprime}{\kern0pt}\ \ {\isasymlongrightarrow}\ \ \isanewline
\ \ \ \ \ \ \ \ {\isacharparenleft}{\kern0pt}{\isasymexists}\ q{\isacharprime}{\kern0pt}{\isachardot}{\kern0pt}\ q\ {\isasymlongmapsto}{\isasymtau}\ q{\isacharprime}{\kern0pt}\ {\isasymand}\ R\ p{\isacharprime}{\kern0pt}\ {\isacharparenleft}{\kern0pt}Some\ Y{\isacharparenright}{\kern0pt}\ q{\isacharprime}{\kern0pt}{\isacharparenright}{\kern0pt}{\isacharparenright}{\kern0pt}\ {\isasymand}\isanewline
\ \ \ \ \ \ {\isacharparenleft}{\kern0pt}{\isasymforall}\ p{\isacharprime}{\kern0pt}\ X{\isachardot}{\kern0pt}\ idle\ p\ {\isacharparenleft}{\kern0pt}X{\isasymunion}Y{\isacharparenright}{\kern0pt}\ {\isasymand}\ X\ {\isasymsubseteq}\ visible{\isacharunderscore}{\kern0pt}actions\ {\isasymand}\ p\ {\isasymlongmapsto}t\ p{\isacharprime}{\kern0pt}\ \ {\isasymlongrightarrow}\ \ \isanewline
\ \ \ \ \ \ \ \ {\isacharparenleft}{\kern0pt}{\isasymexists}\ q{\isacharprime}{\kern0pt}{\isachardot}{\kern0pt}\ q\ {\isasymlongmapsto}t\ q{\isacharprime}{\kern0pt}\ {\isasymand}\ R\ p{\isacharprime}{\kern0pt}\ {\isacharparenleft}{\kern0pt}Some\ X{\isacharparenright}{\kern0pt}\ q{\isacharprime}{\kern0pt}{\isacharparenright}{\kern0pt}{\isacharparenright}{\kern0pt}{\isacharparenright}{\kern0pt}{\isacartoucheclose}\isanewline
\isanewline
%
\isadelimunimportant
%
\endisadelimunimportant
%
\isatagunimportant
%
\endisatagunimportant
{\isafoldunimportant}%
%
\isadelimunimportant
%
\endisadelimunimportant
%
\isadelimdocument
%
\endisadelimdocument
%
\isatagdocument
%
\isamarkupsubsubsection{GSRBs characterise strong reactive/$X$-bisimilarity%
}
\isamarkuptrue%
%
\endisatagdocument
{\isafolddocument}%
%
\isadelimdocument
%
\endisadelimdocument
%
\begin{isamarkuptext}%
\cite[proposition 4]{rbs} reads (notation adapted): \enquote{$p \leftrightarrow_r q$ iff there exists a GSRB $\mathcal{R}$ with $(p,q) \in \mathcal{R}$. Likewise, $p \leftrightarrow_r^X q$ iff there exists a GSRB $\mathcal{R}$ with $(p,X,q) \in \mathcal{R}$.} We shall now replicate the proof of this proposition. First, we prove that each SRB is a GSRB (by showing that each SRB satisfies all clauses of the definition of GSRBs).%
\end{isamarkuptext}\isamarkuptrue%
\isacommand{lemma}\isamarkupfalse%
\ SRB{\isacharunderscore}{\kern0pt}is{\isacharunderscore}{\kern0pt}GSRB{\isacharcolon}{\kern0pt}\isanewline
\ \ \isakeyword{assumes}\ {\isacartoucheopen}SRB\ R{\isacartoucheclose}\isanewline
\ \ \isakeyword{shows}\ {\isacartoucheopen}GSRB\ R{\isacartoucheclose}\isanewline
%
\isadelimproof
\ \ %
\endisadelimproof
%
\isatagproof
\isacommand{unfolding}\isamarkupfalse%
\ GSRB{\isacharunderscore}{\kern0pt}def\isanewline
\isacommand{proof}\isamarkupfalse%
\ {\isacharparenleft}{\kern0pt}safe{\isacharparenright}{\kern0pt}\isanewline
\ \ \isacommand{fix}\isamarkupfalse%
\ p\ XoN\ q\isanewline
\ \ \isacommand{assume}\isamarkupfalse%
\ {\isacartoucheopen}R\ p\ XoN\ q{\isacartoucheclose}\isanewline
\ \ \isacommand{thus}\isamarkupfalse%
\ {\isacartoucheopen}R\ q\ XoN\ p{\isacartoucheclose}\ \isanewline
\ \ \ \ \isacommand{using}\isamarkupfalse%
\ SRB{\isacharunderscore}{\kern0pt}ruleformat{\isacharbrackleft}{\kern0pt}OF\ assms{\isacharbrackright}{\kern0pt}\ \isacommand{by}\isamarkupfalse%
\ blast\isanewline
\isacommand{next}\isamarkupfalse%
\isanewline
\ \ \isacommand{fix}\isamarkupfalse%
\ p\ X\ q\ x\isanewline
\ \ \isacommand{assume}\isamarkupfalse%
\ {\isacartoucheopen}R\ p\ {\isacharparenleft}{\kern0pt}Some\ X{\isacharparenright}{\kern0pt}\ q{\isacartoucheclose}\ \isakeyword{and}\ {\isacartoucheopen}x\ {\isasymin}\ X{\isacartoucheclose}\isanewline
\ \ \isacommand{thus}\isamarkupfalse%
\ {\isacartoucheopen}x\ {\isasymin}\ visible{\isacharunderscore}{\kern0pt}actions{\isacartoucheclose}\ \isanewline
\ \ \ \ \isacommand{using}\isamarkupfalse%
\ SRB{\isacharunderscore}{\kern0pt}ruleformat{\isacharbrackleft}{\kern0pt}OF\ assms{\isacharbrackright}{\kern0pt}\ \isacommand{by}\isamarkupfalse%
\ blast\isanewline
\isacommand{next}\isamarkupfalse%
\isanewline
\ \ \isacommand{fix}\isamarkupfalse%
\ p\ q\ p{\isacharprime}{\kern0pt}\ a\isanewline
\ \ \isacommand{assume}\isamarkupfalse%
\ {\isacartoucheopen}R\ p\ None\ q{\isacartoucheclose}\ \isakeyword{and}\ {\isacartoucheopen}p\ {\isasymlongmapsto}a\ p{\isacharprime}{\kern0pt}{\isacartoucheclose}\ \isakeyword{and}\ {\isacartoucheopen}a\ {\isasymin}\ visible{\isacharunderscore}{\kern0pt}actions{\isacartoucheclose}\isanewline
\ \ \isacommand{thus}\isamarkupfalse%
\ {\isacartoucheopen}{\isasymexists}q{\isacharprime}{\kern0pt}{\isachardot}{\kern0pt}\ q\ {\isasymlongmapsto}a\ q{\isacharprime}{\kern0pt}\ {\isasymand}\ R\ p{\isacharprime}{\kern0pt}\ None\ q{\isacharprime}{\kern0pt}{\isacartoucheclose}\isanewline
\ \ \ \ \isacommand{using}\isamarkupfalse%
\ SRB{\isacharunderscore}{\kern0pt}ruleformat{\isacharparenleft}{\kern0pt}{\isadigit{4}}{\isacharcomma}{\kern0pt}\ {\isadigit{5}}{\isacharparenright}{\kern0pt}{\isacharbrackleft}{\kern0pt}OF\ assms{\isacharcomma}{\kern0pt}\ \isakeyword{where}\ {\isacharquery}{\kern0pt}X\ {\isacharequal}{\kern0pt}\ {\isacartoucheopen}{\isacharbraceleft}{\kern0pt}a{\isacharbraceright}{\kern0pt}{\isacartoucheclose}{\isacharbrackright}{\kern0pt}\ \isacommand{by}\isamarkupfalse%
\ blast\isanewline
\isacommand{next}\isamarkupfalse%
\isanewline
\ \ \isacommand{fix}\isamarkupfalse%
\ p\ q\ p{\isacharprime}{\kern0pt}\ a\isanewline
\ \ \isacommand{assume}\isamarkupfalse%
\ {\isacartoucheopen}R\ p\ None\ q{\isacartoucheclose}\ \isakeyword{and}\ {\isacartoucheopen}p\ {\isasymlongmapsto}{\isasymtau}\ \ p{\isacharprime}{\kern0pt}{\isacartoucheclose}\isanewline
\ \ \isacommand{thus}\isamarkupfalse%
\ {\isacartoucheopen}{\isasymexists}q{\isacharprime}{\kern0pt}{\isachardot}{\kern0pt}\ q\ {\isasymlongmapsto}{\isasymtau}\ \ q{\isacharprime}{\kern0pt}\ {\isasymand}\ R\ p{\isacharprime}{\kern0pt}\ None\ q{\isacharprime}{\kern0pt}{\isacartoucheclose}\isanewline
\ \ \ \ \isacommand{using}\isamarkupfalse%
\ SRB{\isacharunderscore}{\kern0pt}ruleformat{\isacharparenleft}{\kern0pt}{\isadigit{3}}{\isacharparenright}{\kern0pt}{\isacharbrackleft}{\kern0pt}OF\ assms{\isacharbrackright}{\kern0pt}\ \isacommand{by}\isamarkupfalse%
\ blast\isanewline
\isacommand{next}\isamarkupfalse%
\isanewline
\ \ \isacommand{fix}\isamarkupfalse%
\ p\ q\ X\ p{\isacharprime}{\kern0pt}\isanewline
\ \ \isacommand{assume}\isamarkupfalse%
\ {\isacartoucheopen}R\ p\ None\ q{\isacartoucheclose}\ \isakeyword{and}\ {\isacartoucheopen}idle\ p\ X{\isacartoucheclose}\ \isakeyword{and}\ {\isacartoucheopen}X\ {\isasymsubseteq}\ visible{\isacharunderscore}{\kern0pt}actions{\isacartoucheclose}\ \isakeyword{and}\ {\isacartoucheopen}p\ {\isasymlongmapsto}t\ p{\isacharprime}{\kern0pt}{\isacartoucheclose}\isanewline
\ \ \isacommand{thus}\isamarkupfalse%
\ {\isacartoucheopen}{\isasymexists}q{\isacharprime}{\kern0pt}{\isachardot}{\kern0pt}\ q\ {\isasymlongmapsto}t\ \ q{\isacharprime}{\kern0pt}\ {\isasymand}\ R\ p{\isacharprime}{\kern0pt}\ {\isacharparenleft}{\kern0pt}Some\ X{\isacharparenright}{\kern0pt}\ q{\isacharprime}{\kern0pt}{\isacartoucheclose}\isanewline
\ \ \ \ \isacommand{using}\isamarkupfalse%
\ SRB{\isacharunderscore}{\kern0pt}ruleformat{\isacharparenleft}{\kern0pt}{\isadigit{4}}{\isacharcomma}{\kern0pt}\ {\isadigit{8}}{\isacharparenright}{\kern0pt}{\isacharbrackleft}{\kern0pt}OF\ assms{\isacharbrackright}{\kern0pt}\ \isacommand{by}\isamarkupfalse%
\ blast\isanewline
\isacommand{next}\isamarkupfalse%
\isanewline
\ \ \isacommand{fix}\isamarkupfalse%
\ p\ Y\ q\ p{\isacharprime}{\kern0pt}\ a\isanewline
\ \ \isacommand{assume}\isamarkupfalse%
\ {\isacartoucheopen}R\ p\ {\isacharparenleft}{\kern0pt}Some\ Y{\isacharparenright}{\kern0pt}\ q{\isacartoucheclose}\ \isakeyword{and}\ {\isacartoucheopen}p\ {\isasymlongmapsto}a\ p{\isacharprime}{\kern0pt}{\isacartoucheclose}\ \isakeyword{and}\ {\isacartoucheopen}a\ {\isasymin}\ Y{\isacartoucheclose}\isanewline
\ \ \isacommand{thus}\isamarkupfalse%
\ {\isacartoucheopen}{\isasymexists}q{\isacharprime}{\kern0pt}{\isachardot}{\kern0pt}\ q\ {\isasymlongmapsto}a\ q{\isacharprime}{\kern0pt}\ {\isasymand}\ R\ p{\isacharprime}{\kern0pt}\ None\ q{\isacharprime}{\kern0pt}{\isacartoucheclose}\isanewline
\ \ \ \ \isacommand{using}\isamarkupfalse%
\ SRB{\isacharunderscore}{\kern0pt}ruleformat{\isacharparenleft}{\kern0pt}{\isadigit{5}}{\isacharparenright}{\kern0pt}{\isacharbrackleft}{\kern0pt}OF\ assms{\isacharbrackright}{\kern0pt}\ \isacommand{by}\isamarkupfalse%
\ blast\isanewline
\isacommand{next}\isamarkupfalse%
\isanewline
\ \ \isacommand{fix}\isamarkupfalse%
\ p\ Y\ q\ p{\isacharprime}{\kern0pt}\ a\isanewline
\ \ \isacommand{assume}\isamarkupfalse%
\ {\isacartoucheopen}R\ p\ {\isacharparenleft}{\kern0pt}Some\ Y{\isacharparenright}{\kern0pt}\ q{\isacartoucheclose}\ {\isacartoucheopen}a\ {\isasymin}\ visible{\isacharunderscore}{\kern0pt}actions{\isacartoucheclose}\ {\isacartoucheopen}p\ {\isasymlongmapsto}a\ p{\isacharprime}{\kern0pt}{\isacartoucheclose}\ \ {\isacartoucheopen}idle\ p\ Y{\isacartoucheclose}\isanewline
\ \ \isacommand{hence}\isamarkupfalse%
\ {\isacartoucheopen}R\ p\ None\ q{\isacartoucheclose}\ \isacommand{using}\isamarkupfalse%
\ SRB{\isacharunderscore}{\kern0pt}ruleformat{\isacharparenleft}{\kern0pt}{\isadigit{7}}{\isacharparenright}{\kern0pt}{\isacharbrackleft}{\kern0pt}OF\ assms{\isacharbrackright}{\kern0pt}\ \isacommand{by}\isamarkupfalse%
\ simp\isanewline
\ \ \isacommand{hence}\isamarkupfalse%
\ {\isacartoucheopen}R\ p\ {\isacharparenleft}{\kern0pt}Some\ {\isacharbraceleft}{\kern0pt}a{\isacharbraceright}{\kern0pt}{\isacharparenright}{\kern0pt}\ q{\isacartoucheclose}\ \isacommand{using}\isamarkupfalse%
\ SRB{\isacharunderscore}{\kern0pt}ruleformat{\isacharparenleft}{\kern0pt}{\isadigit{4}}{\isacharparenright}{\kern0pt}{\isacharbrackleft}{\kern0pt}OF\ assms{\isacharbrackright}{\kern0pt}\ {\isacartoucheopen}a\ {\isasymin}\ visible{\isacharunderscore}{\kern0pt}actions{\isacartoucheclose}\ \isacommand{by}\isamarkupfalse%
\ simp\isanewline
\ \ \isacommand{thus}\isamarkupfalse%
\ {\isacartoucheopen}{\isasymexists}q{\isacharprime}{\kern0pt}{\isachardot}{\kern0pt}\ q\ {\isasymlongmapsto}a\ q{\isacharprime}{\kern0pt}\ {\isasymand}\ R\ p{\isacharprime}{\kern0pt}\ None\ q{\isacharprime}{\kern0pt}{\isacartoucheclose}\ \isacommand{using}\isamarkupfalse%
\ SRB{\isacharunderscore}{\kern0pt}ruleformat{\isacharparenleft}{\kern0pt}{\isadigit{5}}{\isacharparenright}{\kern0pt}{\isacharbrackleft}{\kern0pt}OF\ assms{\isacharbrackright}{\kern0pt}\ {\isacartoucheopen}p\ {\isasymlongmapsto}a\ p{\isacharprime}{\kern0pt}{\isacartoucheclose}\ \isacommand{by}\isamarkupfalse%
\ blast\isanewline
\isacommand{next}\isamarkupfalse%
\isanewline
\ \ \isacommand{fix}\isamarkupfalse%
\ p\ Y\ q\ p{\isacharprime}{\kern0pt}\isanewline
\ \ \isacommand{assume}\isamarkupfalse%
\ {\isacartoucheopen}R\ p\ {\isacharparenleft}{\kern0pt}Some\ Y{\isacharparenright}{\kern0pt}\ q{\isacartoucheclose}\ \isakeyword{and}\ {\isacartoucheopen}p\ {\isasymlongmapsto}{\isasymtau}\ p{\isacharprime}{\kern0pt}{\isacartoucheclose}\isanewline
\ \ \isacommand{thus}\isamarkupfalse%
\ {\isacartoucheopen}{\isasymexists}q{\isacharprime}{\kern0pt}{\isachardot}{\kern0pt}\ q\ {\isasymlongmapsto}{\isasymtau}\ \ q{\isacharprime}{\kern0pt}\ {\isasymand}\ R\ p{\isacharprime}{\kern0pt}\ {\isacharparenleft}{\kern0pt}Some\ Y{\isacharparenright}{\kern0pt}\ q{\isacharprime}{\kern0pt}{\isacartoucheclose}\isanewline
\ \ \ \ \isacommand{using}\isamarkupfalse%
\ SRB{\isacharunderscore}{\kern0pt}ruleformat{\isacharparenleft}{\kern0pt}{\isadigit{6}}{\isacharparenright}{\kern0pt}{\isacharbrackleft}{\kern0pt}OF\ assms{\isacharbrackright}{\kern0pt}\ \isacommand{by}\isamarkupfalse%
\ blast\isanewline
\isacommand{next}\isamarkupfalse%
\isanewline
\ \ \isacommand{fix}\isamarkupfalse%
\ p\ Y\ q\ p{\isacharprime}{\kern0pt}\ X\isanewline
\ \ \isacommand{assume}\isamarkupfalse%
\ {\isacartoucheopen}R\ p\ {\isacharparenleft}{\kern0pt}Some\ Y{\isacharparenright}{\kern0pt}\ q{\isacartoucheclose}\ {\isacartoucheopen}idle\ p\ {\isacharparenleft}{\kern0pt}X\ {\isasymunion}\ Y{\isacharparenright}{\kern0pt}{\isacartoucheclose}\ {\isacartoucheopen}X\ {\isasymsubseteq}\ visible{\isacharunderscore}{\kern0pt}actions{\isacartoucheclose}\ {\isacartoucheopen}p\ {\isasymlongmapsto}t\ p{\isacharprime}{\kern0pt}{\isacartoucheclose}\isanewline
\ \ \isacommand{from}\isamarkupfalse%
\ {\isacartoucheopen}idle\ p\ {\isacharparenleft}{\kern0pt}X\ {\isasymunion}\ Y{\isacharparenright}{\kern0pt}{\isacartoucheclose}\ \isacommand{have}\isamarkupfalse%
\ {\isacartoucheopen}idle\ p\ Y{\isacartoucheclose}\ \isakeyword{and}\ {\isacartoucheopen}idle\ p\ X{\isacartoucheclose}\isanewline
\ \ \ \ \isacommand{by}\isamarkupfalse%
\ {\isacharparenleft}{\kern0pt}simp\ add{\isacharcolon}{\kern0pt}\ Int{\isacharunderscore}{\kern0pt}Un{\isacharunderscore}{\kern0pt}distrib{\isacharparenright}{\kern0pt}{\isacharplus}{\kern0pt}\isanewline
\ \ \isacommand{from}\isamarkupfalse%
\ {\isacartoucheopen}R\ p\ {\isacharparenleft}{\kern0pt}Some\ Y{\isacharparenright}{\kern0pt}\ q{\isacartoucheclose}\ {\isacartoucheopen}idle\ p\ Y{\isacartoucheclose}\ \isacommand{have}\isamarkupfalse%
\ {\isacartoucheopen}R\ p\ None\ q{\isacartoucheclose}\isanewline
\ \ \ \ \isacommand{using}\isamarkupfalse%
\ SRB{\isacharunderscore}{\kern0pt}ruleformat{\isacharparenleft}{\kern0pt}{\isadigit{7}}{\isacharparenright}{\kern0pt}{\isacharbrackleft}{\kern0pt}OF\ assms{\isacharbrackright}{\kern0pt}\ \isacommand{by}\isamarkupfalse%
\ blast\isanewline
\ \ \isacommand{with}\isamarkupfalse%
\ {\isacartoucheopen}X\ {\isasymsubseteq}\ visible{\isacharunderscore}{\kern0pt}actions{\isacartoucheclose}\ \isacommand{have}\isamarkupfalse%
\ {\isacartoucheopen}R\ p\ {\isacharparenleft}{\kern0pt}Some\ X{\isacharparenright}{\kern0pt}\ q{\isacartoucheclose}\ \isanewline
\ \ \ \ \isacommand{using}\isamarkupfalse%
\ SRB{\isacharunderscore}{\kern0pt}ruleformat{\isacharparenleft}{\kern0pt}{\isadigit{4}}{\isacharparenright}{\kern0pt}{\isacharbrackleft}{\kern0pt}OF\ assms{\isacharbrackright}{\kern0pt}\ \isacommand{by}\isamarkupfalse%
\ blast\isanewline
\ \ \isacommand{with}\isamarkupfalse%
\ {\isacartoucheopen}idle\ p\ X{\isacartoucheclose}\ {\isacartoucheopen}p\ {\isasymlongmapsto}t\ p{\isacharprime}{\kern0pt}{\isacartoucheclose}\ \isacommand{show}\isamarkupfalse%
\ {\isacartoucheopen}{\isasymexists}q{\isacharprime}{\kern0pt}{\isachardot}{\kern0pt}\ q\ {\isasymlongmapsto}t\ \ q{\isacharprime}{\kern0pt}\ {\isasymand}\ R\ p{\isacharprime}{\kern0pt}\ {\isacharparenleft}{\kern0pt}Some\ X{\isacharparenright}{\kern0pt}\ q{\isacharprime}{\kern0pt}{\isacartoucheclose}\isanewline
\ \ \ \ \isacommand{using}\isamarkupfalse%
\ SRB{\isacharunderscore}{\kern0pt}ruleformat{\isacharparenleft}{\kern0pt}{\isadigit{8}}{\isacharparenright}{\kern0pt}{\isacharbrackleft}{\kern0pt}OF\ assms{\isacharbrackright}{\kern0pt}\ \isacommand{by}\isamarkupfalse%
\ blast\isanewline
\isacommand{qed}\isamarkupfalse%
%
\endisatagproof
{\isafoldproof}%
%
\isadelimproof
%
\endisadelimproof
%
\begin{isamarkuptext}%
Then, we show that each GSRB can be extended to yield an SRB. First, we define this extension. Generally, GSRBs can be smaller than SRBs when proving reactive bisimilarity of processes, because they require triples $(p,X,q)$ only after encountering $t$-transitions, whereas SRBs require these triples for all processes and all environments. Furthermore, some process pairs $(p,q)$ related to environment time-outs are also omitted in GSRBs. These tuples are re-added by this extension.
\pagebreak%
\end{isamarkuptext}\isamarkuptrue%
\isacommand{definition}\isamarkupfalse%
\ GSRB{\isacharunderscore}{\kern0pt}extension\ \isanewline
\ \ {\isacharcolon}{\kern0pt}{\isacharcolon}{\kern0pt}\ {\isacartoucheopen}{\isacharparenleft}{\kern0pt}{\isacharprime}{\kern0pt}s{\isasymRightarrow}{\isacharprime}{\kern0pt}a\ set\ option{\isasymRightarrow}{\isacharprime}{\kern0pt}s\ {\isasymRightarrow}\ bool{\isacharparenright}{\kern0pt}{\isasymRightarrow}{\isacharparenleft}{\kern0pt}{\isacharprime}{\kern0pt}s{\isasymRightarrow}{\isacharprime}{\kern0pt}a\ set\ option{\isasymRightarrow}{\isacharprime}{\kern0pt}s\ {\isasymRightarrow}\ bool{\isacharparenright}{\kern0pt}{\isacartoucheclose}\isanewline
\ \ \isakeyword{where}\ {\isacartoucheopen}{\isacharparenleft}{\kern0pt}GSRB{\isacharunderscore}{\kern0pt}extension\ R{\isacharparenright}{\kern0pt}\ p\ XoN\ q\ {\isasymequiv}\isanewline
\ \ \ \ {\isacharparenleft}{\kern0pt}R\ p\ XoN\ q{\isacharparenright}{\kern0pt}\isanewline
\ \ \ \ {\isasymor}\ {\isacharparenleft}{\kern0pt}some{\isacharunderscore}{\kern0pt}visible{\isacharunderscore}{\kern0pt}subset\ XoN\ {\isasymand}\ R\ p\ None\ q{\isacharparenright}{\kern0pt}\isanewline
\ \ \ \ {\isasymor}\ {\isacharparenleft}{\kern0pt}{\isacharparenleft}{\kern0pt}XoN\ {\isacharequal}{\kern0pt}\ None\ {\isasymor}\ some{\isacharunderscore}{\kern0pt}visible{\isacharunderscore}{\kern0pt}subset\ XoN{\isacharparenright}{\kern0pt}\ \isanewline
\ \ \ \ \ \ {\isasymand}\ {\isacharparenleft}{\kern0pt}{\isasymexists}\ Y{\isachardot}{\kern0pt}\ R\ p\ {\isacharparenleft}{\kern0pt}Some\ Y{\isacharparenright}{\kern0pt}\ q\ {\isasymand}\ idle\ p\ Y{\isacharparenright}{\kern0pt}{\isacharparenright}{\kern0pt}{\isacartoucheclose}%
\begin{isamarkuptext}%
Now we show that this extension does, in fact, yield an SRB (again, by showing that all clauses of the definition of SRBs are satisfied).%
\end{isamarkuptext}\isamarkuptrue%
%
\isadelimunimportant
%
\endisadelimunimportant
%
\isatagunimportant
%
\endisatagunimportant
{\isafoldunimportant}%
%
\isadelimunimportant
\isanewline
%
\endisadelimunimportant
\isacommand{lemma}\isamarkupfalse%
\ GSRB{\isacharunderscore}{\kern0pt}extension{\isacharunderscore}{\kern0pt}is{\isacharunderscore}{\kern0pt}SRB{\isacharcolon}{\kern0pt}\isanewline
\ \ \isakeyword{assumes}\isanewline
\ \ \ \ {\isacartoucheopen}GSRB\ R{\isacartoucheclose}\isanewline
\ \ \isakeyword{shows}\isanewline
\ \ \ \ {\isacartoucheopen}SRB\ {\isacharparenleft}{\kern0pt}GSRB{\isacharunderscore}{\kern0pt}extension\ R{\isacharparenright}{\kern0pt}{\isacartoucheclose}\ {\isacharparenleft}{\kern0pt}\isakeyword{is}\ {\isacartoucheopen}SRB\ {\isacharquery}{\kern0pt}R{\isacharunderscore}{\kern0pt}ext{\isacartoucheclose}{\isacharparenright}{\kern0pt}\isanewline
%
\isadelimproof
\ \ %
\endisadelimproof
%
\isatagproof
\isacommand{unfolding}\isamarkupfalse%
\ SRB{\isacharunderscore}{\kern0pt}def\isanewline
\isacommand{proof}\isamarkupfalse%
\ {\isacharparenleft}{\kern0pt}safe{\isacharparenright}{\kern0pt}\isanewline
\ \ \isacommand{fix}\isamarkupfalse%
\ p\ XoN\ q\isanewline
\ \ \isacommand{assume}\isamarkupfalse%
\ {\isacartoucheopen}{\isacharquery}{\kern0pt}R{\isacharunderscore}{\kern0pt}ext\ p\ XoN\ q{\isacartoucheclose}\isanewline
\ \ \isacommand{thus}\isamarkupfalse%
\ {\isacartoucheopen}{\isacharquery}{\kern0pt}R{\isacharunderscore}{\kern0pt}ext\ q\ XoN\ p{\isacartoucheclose}\ \isanewline
\ \ \ \ \isacommand{unfolding}\isamarkupfalse%
\ GSRB{\isacharunderscore}{\kern0pt}extension{\isacharunderscore}{\kern0pt}def\isanewline
\ \ \isacommand{proof}\isamarkupfalse%
\ {\isacharparenleft}{\kern0pt}elim\ disjE{\isacharcomma}{\kern0pt}\ goal{\isacharunderscore}{\kern0pt}cases{\isacharparenright}{\kern0pt}\isanewline
\ \ \ \ \isacommand{case}\isamarkupfalse%
\ {\isadigit{1}}\isanewline
\ \ \ \ \isacommand{hence}\isamarkupfalse%
\ {\isacartoucheopen}R\ q\ XoN\ p{\isacartoucheclose}\isanewline
\ \ \ \ \ \ \isacommand{using}\isamarkupfalse%
\ assms\ GSRB{\isacharunderscore}{\kern0pt}def\ \isacommand{by}\isamarkupfalse%
\ auto\isanewline
\ \ \ \ \isacommand{thus}\isamarkupfalse%
\ {\isacharquery}{\kern0pt}case\ \isacommand{by}\isamarkupfalse%
\ simp\isanewline
\ \ \isacommand{next}\isamarkupfalse%
\isanewline
\ \ \ \ \isacommand{case}\isamarkupfalse%
\ {\isadigit{2}}\isanewline
\ \ \ \ \isacommand{hence}\isamarkupfalse%
\ {\isacartoucheopen}some{\isacharunderscore}{\kern0pt}visible{\isacharunderscore}{\kern0pt}subset\ XoN\ {\isasymand}\ R\ q\ None\ p{\isacartoucheclose}\isanewline
\ \ \ \ \ \ \isacommand{using}\isamarkupfalse%
\ assms\ GSRB{\isacharunderscore}{\kern0pt}def\ \isacommand{by}\isamarkupfalse%
\ auto\isanewline
\ \ \ \ \isacommand{thus}\isamarkupfalse%
\ {\isacharquery}{\kern0pt}case\ \isacommand{by}\isamarkupfalse%
\ simp\isanewline
\ \ \isacommand{next}\isamarkupfalse%
\isanewline
\ \ \ \ \isacommand{case}\isamarkupfalse%
\ {\isadigit{3}}\isanewline
\ \ \ \ \isacommand{then}\isamarkupfalse%
\ \isacommand{obtain}\isamarkupfalse%
\ Y\ \isakeyword{where}\ {\isacartoucheopen}R\ p\ {\isacharparenleft}{\kern0pt}Some\ Y{\isacharparenright}{\kern0pt}\ q{\isacartoucheclose}\ {\isacartoucheopen}idle\ p\ Y{\isacartoucheclose}\ \isacommand{by}\isamarkupfalse%
\ auto\isanewline
\ \ \ \ \isacommand{hence}\isamarkupfalse%
\ {\isacartoucheopen}R\ q\ {\isacharparenleft}{\kern0pt}Some\ Y{\isacharparenright}{\kern0pt}\ p{\isacartoucheclose}\isanewline
\ \ \ \ \ \ \isacommand{using}\isamarkupfalse%
\ assms\ GSRB{\isacharunderscore}{\kern0pt}def\ \isacommand{by}\isamarkupfalse%
\ auto\isanewline
\ \ \ \ \isacommand{have}\isamarkupfalse%
\ {\isacartoucheopen}idle\ q\ Y{\isacartoucheclose}\isanewline
\ \ \ \ \ \ \isacommand{using}\isamarkupfalse%
\ GSRB{\isacharunderscore}{\kern0pt}preserves{\isacharunderscore}{\kern0pt}idleness{\isacharbrackleft}{\kern0pt}OF\ assms{\isacharbrackright}{\kern0pt}\ {\isacartoucheopen}R\ p\ {\isacharparenleft}{\kern0pt}Some\ Y{\isacharparenright}{\kern0pt}\ q{\isacartoucheclose}\ {\isacartoucheopen}idle\ p\ Y{\isacartoucheclose}\ \isacommand{{\isachardot}{\kern0pt}}\isamarkupfalse%
\isanewline
\ \ \ \ \isacommand{from}\isamarkupfalse%
\ {\isadigit{3}}\ {\isacartoucheopen}R\ q\ {\isacharparenleft}{\kern0pt}Some\ Y{\isacharparenright}{\kern0pt}\ p{\isacartoucheclose}\ {\isacartoucheopen}idle\ q\ Y{\isacartoucheclose}\ \isacommand{show}\isamarkupfalse%
\ {\isacharquery}{\kern0pt}case\ \isacommand{by}\isamarkupfalse%
\ blast\isanewline
\ \ \isacommand{qed}\isamarkupfalse%
\isanewline
\isacommand{next}\isamarkupfalse%
\isanewline
\ \ \isacommand{fix}\isamarkupfalse%
\ p\ X\ q\ x\isanewline
\ \ \isacommand{assume}\isamarkupfalse%
\ {\isacartoucheopen}{\isacharquery}{\kern0pt}R{\isacharunderscore}{\kern0pt}ext\ p\ {\isacharparenleft}{\kern0pt}Some\ X{\isacharparenright}{\kern0pt}\ q{\isacartoucheclose}\ {\isacartoucheopen}x\ {\isasymin}\ X{\isacartoucheclose}\isanewline
\ \ \isacommand{thus}\isamarkupfalse%
\ {\isacartoucheopen}x\ {\isasymin}\ visible{\isacharunderscore}{\kern0pt}actions{\isacartoucheclose}\ \isanewline
\ \ \ \ \isacommand{unfolding}\isamarkupfalse%
\ GSRB{\isacharunderscore}{\kern0pt}extension{\isacharunderscore}{\kern0pt}def\isanewline
\ \ \isacommand{proof}\isamarkupfalse%
\ {\isacharparenleft}{\kern0pt}elim\ disjE{\isacharcomma}{\kern0pt}\ goal{\isacharunderscore}{\kern0pt}cases{\isacharparenright}{\kern0pt}\isanewline
\ \ \ \ \isacommand{case}\isamarkupfalse%
\ {\isadigit{1}}\isanewline
\ \ \ \ \isacommand{thus}\isamarkupfalse%
\ {\isacharquery}{\kern0pt}case\ \isacommand{using}\isamarkupfalse%
\ GSRB{\isacharunderscore}{\kern0pt}ruleformat{\isacharparenleft}{\kern0pt}{\isadigit{1}}{\isacharparenright}{\kern0pt}{\isacharbrackleft}{\kern0pt}OF\ assms{\isacharbrackright}{\kern0pt}\ \isacommand{by}\isamarkupfalse%
\ blast\isanewline
\ \ \isacommand{next}\isamarkupfalse%
\isanewline
\ \ \ \ \isacommand{case}\isamarkupfalse%
\ {\isadigit{2}}\isanewline
\ \ \ \ \isacommand{thus}\isamarkupfalse%
\ {\isacharquery}{\kern0pt}case\ \isacommand{by}\isamarkupfalse%
\ auto\isanewline
\ \ \isacommand{next}\isamarkupfalse%
\isanewline
\ \ \ \ \isacommand{case}\isamarkupfalse%
\ {\isadigit{3}}\isanewline
\ \ \ \ \isacommand{thus}\isamarkupfalse%
\ {\isacharquery}{\kern0pt}case\ \isacommand{by}\isamarkupfalse%
\ auto\isanewline
\ \ \isacommand{qed}\isamarkupfalse%
\isanewline
\isacommand{next}\isamarkupfalse%
\isanewline
\ \ \isacommand{fix}\isamarkupfalse%
\ p\ q\ p{\isacharprime}{\kern0pt}\isanewline
\ \ \isacommand{assume}\isamarkupfalse%
\ {\isacartoucheopen}{\isacharquery}{\kern0pt}R{\isacharunderscore}{\kern0pt}ext\ p\ None\ q{\isacartoucheclose}\ {\isacartoucheopen}p\ {\isasymlongmapsto}{\isasymtau}\ p{\isacharprime}{\kern0pt}{\isacartoucheclose}\isanewline
\ \ \isacommand{thus}\isamarkupfalse%
\ {\isacartoucheopen}{\isasymexists}q{\isacharprime}{\kern0pt}{\isachardot}{\kern0pt}\ q\ {\isasymlongmapsto}{\isasymtau}\ \ q{\isacharprime}{\kern0pt}\ {\isasymand}\ {\isacharquery}{\kern0pt}R{\isacharunderscore}{\kern0pt}ext\ p{\isacharprime}{\kern0pt}\ None\ q{\isacharprime}{\kern0pt}{\isacartoucheclose}\ \isanewline
\ \ \ \ \isacommand{unfolding}\isamarkupfalse%
\ GSRB{\isacharunderscore}{\kern0pt}extension{\isacharunderscore}{\kern0pt}def\isanewline
\ \ \isacommand{proof}\isamarkupfalse%
\ {\isacharparenleft}{\kern0pt}elim\ disjE{\isacharcomma}{\kern0pt}\ goal{\isacharunderscore}{\kern0pt}cases{\isacharparenright}{\kern0pt}\isanewline
\ \ \ \ \isacommand{case}\isamarkupfalse%
\ {\isadigit{1}}\isanewline
\ \ \ \ \isacommand{then}\isamarkupfalse%
\ \isacommand{obtain}\isamarkupfalse%
\ q{\isacharprime}{\kern0pt}\ \isakeyword{where}\ {\isacartoucheopen}q\ {\isasymlongmapsto}{\isasymtau}\ q{\isacharprime}{\kern0pt}{\isacartoucheclose}\ {\isacartoucheopen}R\ p{\isacharprime}{\kern0pt}\ None\ q{\isacharprime}{\kern0pt}{\isacartoucheclose}\isanewline
\ \ \ \ \ \ \isacommand{using}\isamarkupfalse%
\ GSRB{\isacharunderscore}{\kern0pt}ruleformat{\isacharparenleft}{\kern0pt}{\isadigit{3}}{\isacharparenright}{\kern0pt}{\isacharbrackleft}{\kern0pt}OF\ assms{\isacharbrackright}{\kern0pt}\ lts{\isacharunderscore}{\kern0pt}timeout{\isacharunderscore}{\kern0pt}axioms\ \isacommand{by}\isamarkupfalse%
\ fastforce\isanewline
\ \ \ \ \isacommand{thus}\isamarkupfalse%
\ {\isacharquery}{\kern0pt}case\ \isacommand{by}\isamarkupfalse%
\ auto\isanewline
\ \ \isacommand{next}\isamarkupfalse%
\isanewline
\ \ \ \ \isacommand{case}\isamarkupfalse%
\ {\isadigit{2}}\isanewline
\ \ \ \ \isacommand{hence}\isamarkupfalse%
\ False\ \isacommand{by}\isamarkupfalse%
\ auto\isanewline
\ \ \ \ \isacommand{thus}\isamarkupfalse%
\ {\isacharquery}{\kern0pt}case\ \isacommand{by}\isamarkupfalse%
\ simp\isanewline
\ \ \isacommand{next}\isamarkupfalse%
\isanewline
\ \ \ \ \isacommand{thm}\isamarkupfalse%
\ GSRB{\isacharunderscore}{\kern0pt}ruleformat{\isacharparenleft}{\kern0pt}{\isadigit{5}}{\isacharparenright}{\kern0pt}{\isacharbrackleft}{\kern0pt}OF\ assms{\isacharcomma}{\kern0pt}\ \isakeyword{where}\ {\isacharquery}{\kern0pt}a{\isacharequal}{\kern0pt}{\isasymtau}{\isacharbrackright}{\kern0pt}\isanewline
\ \ \ \ \isacommand{case}\isamarkupfalse%
\ {\isadigit{3}}\isanewline
\ \ \ \ \isacommand{hence}\isamarkupfalse%
\ {\isacartoucheopen}{\isasymexists}q{\isacharprime}{\kern0pt}{\isachardot}{\kern0pt}\ q\ {\isasymlongmapsto}{\isasymtau}\ \ q{\isacharprime}{\kern0pt}\ {\isasymand}\ R\ p{\isacharprime}{\kern0pt}\ None\ q{\isacharprime}{\kern0pt}{\isacartoucheclose}\isanewline
\ \ \ \ \ \ \isacommand{using}\isamarkupfalse%
\ initial{\isacharunderscore}{\kern0pt}actions{\isacharunderscore}{\kern0pt}def\ \isacommand{by}\isamarkupfalse%
\ fastforce\isanewline
\ \ \ \ \isacommand{thus}\isamarkupfalse%
\ {\isacharquery}{\kern0pt}case\ \isacommand{by}\isamarkupfalse%
\ auto\isanewline
\ \ \isacommand{qed}\isamarkupfalse%
\isanewline
\isacommand{next}\isamarkupfalse%
\isanewline
\ \ \isacommand{fix}\isamarkupfalse%
\ p\ q\ X\isanewline
\ \ \isacommand{assume}\isamarkupfalse%
\ {\isacartoucheopen}{\isacharquery}{\kern0pt}R{\isacharunderscore}{\kern0pt}ext\ p\ None\ q{\isacartoucheclose}\ {\isacartoucheopen}X\ {\isasymsubseteq}\ visible{\isacharunderscore}{\kern0pt}actions{\isacartoucheclose}\isanewline
\ \ \isacommand{thus}\isamarkupfalse%
\ {\isacartoucheopen}{\isacharquery}{\kern0pt}R{\isacharunderscore}{\kern0pt}ext\ p\ {\isacharparenleft}{\kern0pt}Some\ X{\isacharparenright}{\kern0pt}\ q{\isacartoucheclose}\ \isanewline
\ \ \ \ \isacommand{unfolding}\isamarkupfalse%
\ GSRB{\isacharunderscore}{\kern0pt}extension{\isacharunderscore}{\kern0pt}def\isanewline
\ \ \isacommand{proof}\isamarkupfalse%
\ {\isacharparenleft}{\kern0pt}elim\ disjE{\isacharcomma}{\kern0pt}\ goal{\isacharunderscore}{\kern0pt}cases{\isacharparenright}{\kern0pt}\isanewline
\ \ \ \ \isacommand{case}\isamarkupfalse%
\ {\isadigit{1}}\isanewline
\ \ \ \ \isacommand{thus}\isamarkupfalse%
\ {\isacharquery}{\kern0pt}case\ \isacommand{by}\isamarkupfalse%
\ auto\isanewline
\ \ \isacommand{next}\isamarkupfalse%
\isanewline
\ \ \ \ \isacommand{case}\isamarkupfalse%
\ {\isadigit{2}}\isanewline
\ \ \ \ \isacommand{hence}\isamarkupfalse%
\ False\ \isacommand{by}\isamarkupfalse%
\ auto\isanewline
\ \ \ \ \isacommand{thus}\isamarkupfalse%
\ {\isacharquery}{\kern0pt}case\ \isacommand{by}\isamarkupfalse%
\ simp\isanewline
\ \ \isacommand{next}\isamarkupfalse%
\isanewline
\ \ \ \ \isacommand{case}\isamarkupfalse%
\ {\isadigit{3}}\isanewline
\ \ \ \ \isacommand{hence}\isamarkupfalse%
\ {\isacartoucheopen}some{\isacharunderscore}{\kern0pt}visible{\isacharunderscore}{\kern0pt}subset\ {\isacharparenleft}{\kern0pt}Some\ X{\isacharparenright}{\kern0pt}{\isacartoucheclose}\ \isacommand{by}\isamarkupfalse%
\ simp\isanewline
\ \ \ \ \isacommand{with}\isamarkupfalse%
\ {\isadigit{3}}\ \isacommand{show}\isamarkupfalse%
\ {\isacharquery}{\kern0pt}case\ \isacommand{by}\isamarkupfalse%
\ simp\isanewline
\ \ \isacommand{qed}\isamarkupfalse%
\isanewline
\isacommand{next}\isamarkupfalse%
\isanewline
\ \ \isacommand{fix}\isamarkupfalse%
\ p\ X\ q\ p{\isacharprime}{\kern0pt}\ a\isanewline
\ \ \isacommand{assume}\isamarkupfalse%
\ {\isacartoucheopen}{\isacharquery}{\kern0pt}R{\isacharunderscore}{\kern0pt}ext\ p\ {\isacharparenleft}{\kern0pt}Some\ X{\isacharparenright}{\kern0pt}\ q{\isacartoucheclose}\ {\isacartoucheopen}p\ {\isasymlongmapsto}a\ p{\isacharprime}{\kern0pt}{\isacartoucheclose}\ {\isacartoucheopen}a\ {\isasymin}\ X{\isacartoucheclose}\isanewline
\ \ \isacommand{thus}\isamarkupfalse%
\ {\isacartoucheopen}{\isasymexists}q{\isacharprime}{\kern0pt}{\isachardot}{\kern0pt}\ q\ {\isasymlongmapsto}a\ q{\isacharprime}{\kern0pt}\ {\isasymand}\ {\isacharquery}{\kern0pt}R{\isacharunderscore}{\kern0pt}ext\ p{\isacharprime}{\kern0pt}\ None\ q{\isacharprime}{\kern0pt}{\isacartoucheclose}\ \isanewline
\ \ \ \ \isacommand{unfolding}\isamarkupfalse%
\ GSRB{\isacharunderscore}{\kern0pt}extension{\isacharunderscore}{\kern0pt}def\isanewline
\ \ \isacommand{proof}\isamarkupfalse%
\ {\isacharparenleft}{\kern0pt}elim\ disjE{\isacharcomma}{\kern0pt}\ goal{\isacharunderscore}{\kern0pt}cases{\isacharparenright}{\kern0pt}\isanewline
\ \ \ \ \isacommand{case}\isamarkupfalse%
\ {\isadigit{1}}\isanewline
\ \ \ \ \isacommand{thus}\isamarkupfalse%
\ {\isacharquery}{\kern0pt}case\ \isanewline
\ \ \ \ \ \ \isacommand{using}\isamarkupfalse%
\ GSRB{\isacharunderscore}{\kern0pt}ruleformat{\isacharparenleft}{\kern0pt}{\isadigit{1}}{\isacharcomma}{\kern0pt}{\isadigit{5}}{\isacharparenright}{\kern0pt}{\isacharbrackleft}{\kern0pt}OF\ assms{\isacharbrackright}{\kern0pt}\ \isacommand{by}\isamarkupfalse%
\ blast\isanewline
\ \ \isacommand{next}\isamarkupfalse%
\isanewline
\ \ \ \ \isacommand{case}\isamarkupfalse%
\ {\isadigit{2}}\isanewline
\ \ \ \ \isacommand{thus}\isamarkupfalse%
\ {\isacharquery}{\kern0pt}case\ \isanewline
\ \ \ \ \ \ \isacommand{using}\isamarkupfalse%
\ GSRB{\isacharunderscore}{\kern0pt}ruleformat{\isacharparenleft}{\kern0pt}{\isadigit{3}}{\isacharparenright}{\kern0pt}{\isacharbrackleft}{\kern0pt}OF\ assms{\isacharbrackright}{\kern0pt}\ \isacommand{by}\isamarkupfalse%
\ blast\isanewline
\ \ \isacommand{next}\isamarkupfalse%
\isanewline
\ \ \ \ \isacommand{case}\isamarkupfalse%
\ {\isadigit{3}}\isanewline
\ \ \ \ \isacommand{then}\isamarkupfalse%
\ \isacommand{obtain}\isamarkupfalse%
\ Y\ \isakeyword{where}\ {\isacartoucheopen}R\ p\ {\isacharparenleft}{\kern0pt}Some\ Y{\isacharparenright}{\kern0pt}\ q{\isacartoucheclose}\ {\isacartoucheopen}idle\ p\ Y{\isacartoucheclose}\ \isacommand{by}\isamarkupfalse%
\ blast\isanewline
\ \ \ \ \isacommand{with}\isamarkupfalse%
\ {\isadigit{3}}\ \isacommand{have}\isamarkupfalse%
\ {\isacartoucheopen}a\ {\isasymin}\ visible{\isacharunderscore}{\kern0pt}actions{\isacartoucheclose}\isanewline
\ \ \ \ \ \ \isacommand{using}\isamarkupfalse%
\ GSRB{\isacharunderscore}{\kern0pt}ruleformat{\isacharparenleft}{\kern0pt}{\isadigit{2}}{\isacharparenright}{\kern0pt}{\isacharbrackleft}{\kern0pt}OF\ assms{\isacharbrackright}{\kern0pt}\ \isacommand{by}\isamarkupfalse%
\ blast\isanewline
\ \ \ \ \isacommand{from}\isamarkupfalse%
\ {\isadigit{3}}\ {\isacartoucheopen}idle\ p\ Y{\isacartoucheclose}\ \isacommand{show}\isamarkupfalse%
\ {\isacharquery}{\kern0pt}case\ \isanewline
\ \ \ \ \ \ \isacommand{using}\isamarkupfalse%
\ GSRB{\isacharunderscore}{\kern0pt}ruleformat{\isacharparenleft}{\kern0pt}{\isadigit{5}}{\isacharparenright}{\kern0pt}{\isacharbrackleft}{\kern0pt}OF\ assms\ {\isacartoucheopen}R\ p\ {\isacharparenleft}{\kern0pt}Some\ Y{\isacharparenright}{\kern0pt}\ q{\isacartoucheclose}\ {\isacartoucheopen}a\ {\isasymin}\ visible{\isacharunderscore}{\kern0pt}actions{\isacartoucheclose}{\isacharbrackright}{\kern0pt}\ \isacommand{by}\isamarkupfalse%
\ metis\isanewline
\ \ \isacommand{qed}\isamarkupfalse%
\isanewline
\isacommand{next}\isamarkupfalse%
\isanewline
\ \ \isacommand{fix}\isamarkupfalse%
\ p\ X\ q\ p{\isacharprime}{\kern0pt}\isanewline
\ \ \isacommand{assume}\isamarkupfalse%
\ {\isacartoucheopen}{\isacharquery}{\kern0pt}R{\isacharunderscore}{\kern0pt}ext\ p\ {\isacharparenleft}{\kern0pt}Some\ X{\isacharparenright}{\kern0pt}\ q{\isacartoucheclose}\ {\isacartoucheopen}p\ {\isasymlongmapsto}{\isasymtau}\ p{\isacharprime}{\kern0pt}{\isacartoucheclose}\isanewline
\ \ \isacommand{thus}\isamarkupfalse%
\ {\isacartoucheopen}{\isasymexists}q{\isacharprime}{\kern0pt}{\isachardot}{\kern0pt}\ q\ {\isasymlongmapsto}{\isasymtau}\ \ q{\isacharprime}{\kern0pt}\ {\isasymand}\ {\isacharquery}{\kern0pt}R{\isacharunderscore}{\kern0pt}ext\ p{\isacharprime}{\kern0pt}\ {\isacharparenleft}{\kern0pt}Some\ X{\isacharparenright}{\kern0pt}\ q{\isacharprime}{\kern0pt}{\isacartoucheclose}\ \isanewline
\ \ \ \ \isacommand{unfolding}\isamarkupfalse%
\ GSRB{\isacharunderscore}{\kern0pt}extension{\isacharunderscore}{\kern0pt}def\isanewline
\ \ \isacommand{proof}\isamarkupfalse%
\ {\isacharparenleft}{\kern0pt}elim\ disjE{\isacharcomma}{\kern0pt}\ goal{\isacharunderscore}{\kern0pt}cases{\isacharparenright}{\kern0pt}\isanewline
\ \ \ \ \isacommand{case}\isamarkupfalse%
\ {\isadigit{1}}\isanewline
\ \ \ \ \isacommand{thus}\isamarkupfalse%
\ {\isacharquery}{\kern0pt}case\ \isanewline
\ \ \ \ \ \ \isacommand{using}\isamarkupfalse%
\ GSRB{\isacharunderscore}{\kern0pt}ruleformat{\isacharparenleft}{\kern0pt}{\isadigit{6}}{\isacharparenright}{\kern0pt}{\isacharbrackleft}{\kern0pt}OF\ assms{\isacharbrackright}{\kern0pt}\ \isacommand{by}\isamarkupfalse%
\ meson\isanewline
\ \ \isacommand{next}\isamarkupfalse%
\isanewline
\ \ \ \ \isacommand{case}\isamarkupfalse%
\ {\isadigit{2}}\isanewline
\ \ \ \ \isacommand{thus}\isamarkupfalse%
\ {\isacharquery}{\kern0pt}case\ \isanewline
\ \ \ \ \ \ \isacommand{using}\isamarkupfalse%
\ GSRB{\isacharunderscore}{\kern0pt}ruleformat{\isacharparenleft}{\kern0pt}{\isadigit{3}}{\isacharparenright}{\kern0pt}{\isacharbrackleft}{\kern0pt}OF\ assms{\isacharbrackright}{\kern0pt}\ \isacommand{by}\isamarkupfalse%
\ blast\isanewline
\ \ \isacommand{next}\isamarkupfalse%
\isanewline
\ \ \ \ \isacommand{case}\isamarkupfalse%
\ {\isadigit{3}}\isanewline
\ \ \ \ \isacommand{then}\isamarkupfalse%
\ \isacommand{obtain}\isamarkupfalse%
\ Y\ \isakeyword{where}\ {\isacartoucheopen}idle\ p\ Y{\isacartoucheclose}\ \isacommand{by}\isamarkupfalse%
\ blast\isanewline
\ \ \ \ \isacommand{with}\isamarkupfalse%
\ {\isadigit{3}}{\isacharparenleft}{\kern0pt}{\isadigit{1}}{\isacharparenright}{\kern0pt}\ \isacommand{have}\isamarkupfalse%
\ False\ \isanewline
\ \ \ \ \ \ \isacommand{using}\isamarkupfalse%
\ initial{\isacharunderscore}{\kern0pt}actions{\isacharunderscore}{\kern0pt}def\ \isacommand{by}\isamarkupfalse%
\ auto\isanewline
\ \ \ \ \isacommand{thus}\isamarkupfalse%
\ {\isacharquery}{\kern0pt}case\ \isacommand{by}\isamarkupfalse%
\ simp\isanewline
\ \ \isacommand{qed}\isamarkupfalse%
\isanewline
\isacommand{next}\isamarkupfalse%
\isanewline
\ \ \isacommand{fix}\isamarkupfalse%
\ p\ X\ q\isanewline
\ \ \isacommand{assume}\isamarkupfalse%
\ {\isacartoucheopen}{\isacharquery}{\kern0pt}R{\isacharunderscore}{\kern0pt}ext\ p\ {\isacharparenleft}{\kern0pt}Some\ X{\isacharparenright}{\kern0pt}\ q{\isacartoucheclose}\ {\isacartoucheopen}idle\ p\ X{\isacartoucheclose}\isanewline
\ \ \isacommand{thus}\isamarkupfalse%
\ {\isacartoucheopen}{\isacharquery}{\kern0pt}R{\isacharunderscore}{\kern0pt}ext\ p\ None\ q{\isacartoucheclose}\ \isanewline
\ \ \ \ \isacommand{unfolding}\isamarkupfalse%
\ GSRB{\isacharunderscore}{\kern0pt}extension{\isacharunderscore}{\kern0pt}def\ \isacommand{by}\isamarkupfalse%
\ auto\isanewline
\isacommand{next}\isamarkupfalse%
\isanewline
\ \ \isacommand{fix}\isamarkupfalse%
\ p\ X\ q\ p{\isacharprime}{\kern0pt}\isanewline
\ \ \isacommand{assume}\isamarkupfalse%
\ {\isacartoucheopen}{\isacharquery}{\kern0pt}R{\isacharunderscore}{\kern0pt}ext\ p\ {\isacharparenleft}{\kern0pt}Some\ X{\isacharparenright}{\kern0pt}\ q{\isacartoucheclose}\ {\isacartoucheopen}idle\ p\ X{\isacartoucheclose}\ {\isacartoucheopen}p\ {\isasymlongmapsto}t\ p{\isacharprime}{\kern0pt}{\isacartoucheclose}\isanewline
\ \ \isacommand{thus}\isamarkupfalse%
\ {\isacartoucheopen}{\isasymexists}q{\isacharprime}{\kern0pt}{\isachardot}{\kern0pt}\ q\ {\isasymlongmapsto}t\ \ q{\isacharprime}{\kern0pt}\ {\isasymand}\ {\isacharquery}{\kern0pt}R{\isacharunderscore}{\kern0pt}ext\ p{\isacharprime}{\kern0pt}\ {\isacharparenleft}{\kern0pt}Some\ X{\isacharparenright}{\kern0pt}\ q{\isacharprime}{\kern0pt}{\isacartoucheclose}\ \isanewline
\ \ \ \ \isacommand{unfolding}\isamarkupfalse%
\ GSRB{\isacharunderscore}{\kern0pt}extension{\isacharunderscore}{\kern0pt}def\isanewline
\ \ \isacommand{proof}\isamarkupfalse%
\ {\isacharparenleft}{\kern0pt}elim\ disjE{\isacharcomma}{\kern0pt}\ goal{\isacharunderscore}{\kern0pt}cases{\isacharparenright}{\kern0pt}\isanewline
\ \ \ \ \isacommand{case}\isamarkupfalse%
\ {\isadigit{1}}\isanewline
\ \ \ \ \isacommand{from}\isamarkupfalse%
\ {\isadigit{1}}{\isacharparenleft}{\kern0pt}{\isadigit{1}}{\isacharparenright}{\kern0pt}\ \isacommand{have}\isamarkupfalse%
\ {\isacartoucheopen}idle\ p\ {\isacharparenleft}{\kern0pt}X\ {\isasymunion}\ X{\isacharparenright}{\kern0pt}{\isacartoucheclose}\ \isacommand{by}\isamarkupfalse%
\ simp\isanewline
\ \ \ \ \isacommand{from}\isamarkupfalse%
\ GSRB{\isacharunderscore}{\kern0pt}ruleformat{\isacharparenleft}{\kern0pt}{\isadigit{1}}{\isacharparenright}{\kern0pt}{\isacharbrackleft}{\kern0pt}OF\ assms\ {\isadigit{1}}{\isacharparenleft}{\kern0pt}{\isadigit{3}}{\isacharparenright}{\kern0pt}{\isacharbrackright}{\kern0pt}\ \isacommand{have}\isamarkupfalse%
\ {\isacartoucheopen}X\ {\isasymsubseteq}\ visible{\isacharunderscore}{\kern0pt}actions{\isacartoucheclose}\ \isacommand{{\isachardot}{\kern0pt}}\isamarkupfalse%
\isanewline
\ \ \ \ \isacommand{from}\isamarkupfalse%
\ GSRB{\isacharunderscore}{\kern0pt}ruleformat{\isacharparenleft}{\kern0pt}{\isadigit{7}}{\isacharparenright}{\kern0pt}{\isacharbrackleft}{\kern0pt}OF\ assms\ {\isadigit{1}}{\isacharparenleft}{\kern0pt}{\isadigit{3}}{\isacharparenright}{\kern0pt}\ {\isacartoucheopen}idle\ p\ {\isacharparenleft}{\kern0pt}X\ {\isasymunion}\ X{\isacharparenright}{\kern0pt}{\isacartoucheclose}\ {\isacartoucheopen}X\ {\isasymsubseteq}\ visible{\isacharunderscore}{\kern0pt}actions{\isacartoucheclose}\ {\isadigit{1}}{\isacharparenleft}{\kern0pt}{\isadigit{2}}{\isacharparenright}{\kern0pt}{\isacharbrackright}{\kern0pt}\isanewline
\ \ \ \ \isacommand{show}\isamarkupfalse%
\ {\isacharquery}{\kern0pt}case\ \isacommand{by}\isamarkupfalse%
\ auto\isanewline
\ \ \isacommand{next}\isamarkupfalse%
\isanewline
\ \ \ \ \isacommand{case}\isamarkupfalse%
\ {\isadigit{2}}\isanewline
\ \ \ \ \isacommand{thus}\isamarkupfalse%
\ {\isacharquery}{\kern0pt}case\isanewline
\ \ \ \ \ \ \isacommand{using}\isamarkupfalse%
\ GSRB{\isacharunderscore}{\kern0pt}ruleformat{\isacharparenleft}{\kern0pt}{\isadigit{4}}{\isacharparenright}{\kern0pt}{\isacharbrackleft}{\kern0pt}OF\ assms{\isacharbrackright}{\kern0pt}\isanewline
\ \ \ \ \ \ \isacommand{by}\isamarkupfalse%
\ {\isacharparenleft}{\kern0pt}metis\ option{\isachardot}{\kern0pt}inject{\isacharparenright}{\kern0pt}\isanewline
\ \ \isacommand{next}\isamarkupfalse%
\isanewline
\ \ \ \ \isacommand{case}\isamarkupfalse%
\ {\isadigit{3}}\isanewline
\ \ \ \ \isacommand{then}\isamarkupfalse%
\ \isacommand{obtain}\isamarkupfalse%
\ Y\ \isakeyword{where}\ {\isacartoucheopen}R\ p\ {\isacharparenleft}{\kern0pt}Some\ Y{\isacharparenright}{\kern0pt}\ q{\isacartoucheclose}\ {\isacartoucheopen}idle\ p\ Y{\isacartoucheclose}\ \isacommand{by}\isamarkupfalse%
\ blast\isanewline
\ \ \ \ \isacommand{from}\isamarkupfalse%
\ {\isacartoucheopen}idle\ p\ X{\isacartoucheclose}\ {\isacartoucheopen}idle\ p\ Y{\isacartoucheclose}\ \isacommand{have}\isamarkupfalse%
\ {\isacartoucheopen}idle\ p\ {\isacharparenleft}{\kern0pt}X\ {\isasymunion}\ Y{\isacharparenright}{\kern0pt}{\isacartoucheclose}\isanewline
\ \ \ \ \ \ \isacommand{by}\isamarkupfalse%
\ {\isacharparenleft}{\kern0pt}smt\ bot{\isacharunderscore}{\kern0pt}eq{\isacharunderscore}{\kern0pt}sup{\isacharunderscore}{\kern0pt}iff\ inf{\isacharunderscore}{\kern0pt}sup{\isacharunderscore}{\kern0pt}distrib{\isadigit{1}}{\isacharparenright}{\kern0pt}\isanewline
\ \ \ \ \isacommand{from}\isamarkupfalse%
\ {\isadigit{3}}{\isacharparenleft}{\kern0pt}{\isadigit{3}}{\isacharparenright}{\kern0pt}\ \isacommand{have}\isamarkupfalse%
\ {\isacartoucheopen}X\ {\isasymsubseteq}\ visible{\isacharunderscore}{\kern0pt}actions{\isacartoucheclose}\ \isacommand{by}\isamarkupfalse%
\ blast\isanewline
\ \ \ \ \isacommand{from}\isamarkupfalse%
\ GSRB{\isacharunderscore}{\kern0pt}ruleformat{\isacharparenleft}{\kern0pt}{\isadigit{7}}{\isacharparenright}{\kern0pt}{\isacharbrackleft}{\kern0pt}OF\ assms\ {\isacartoucheopen}R\ p\ {\isacharparenleft}{\kern0pt}Some\ Y{\isacharparenright}{\kern0pt}\ q{\isacartoucheclose}\ {\isacartoucheopen}idle\ p\ {\isacharparenleft}{\kern0pt}X\ {\isasymunion}\ Y{\isacharparenright}{\kern0pt}{\isacartoucheclose}\ {\isacartoucheopen}X\ {\isasymsubseteq}\ visible{\isacharunderscore}{\kern0pt}actions{\isacartoucheclose}\ {\isadigit{3}}{\isacharparenleft}{\kern0pt}{\isadigit{2}}{\isacharparenright}{\kern0pt}{\isacharbrackright}{\kern0pt}\isanewline
\ \ \ \ \isacommand{show}\isamarkupfalse%
\ {\isacharquery}{\kern0pt}case\ \isacommand{by}\isamarkupfalse%
\ auto\isanewline
\ \ \isacommand{qed}\isamarkupfalse%
\isanewline
\isacommand{qed}\isamarkupfalse%
%
\endisatagproof
{\isafoldproof}%
%
\isadelimproof
%
\endisadelimproof
%
\begin{isamarkuptext}%
Finally, we can conclude the following:%
\end{isamarkuptext}\isamarkuptrue%
\isacommand{lemma}\isamarkupfalse%
\ GSRB{\isacharunderscore}{\kern0pt}whenever{\isacharunderscore}{\kern0pt}SRB{\isacharcolon}{\kern0pt}\isanewline
\ \ \isakeyword{shows}\ {\isacartoucheopen}{\isacharparenleft}{\kern0pt}{\isasymexists}\ R{\isachardot}{\kern0pt}\ GSRB\ R\ {\isasymand}\ R\ p\ XoN\ q{\isacharparenright}{\kern0pt}\ \ {\isasymLongleftrightarrow}\ \ {\isacharparenleft}{\kern0pt}{\isasymexists}\ R{\isachardot}{\kern0pt}\ SRB\ R\ {\isasymand}\ R\ p\ XoN\ q{\isacharparenright}{\kern0pt}{\isacartoucheclose}\isanewline
%
\isadelimproof
\ \ %
\endisadelimproof
%
\isatagproof
\isacommand{by}\isamarkupfalse%
\ {\isacharparenleft}{\kern0pt}metis\ GSRB{\isacharunderscore}{\kern0pt}extension{\isacharunderscore}{\kern0pt}def\ GSRB{\isacharunderscore}{\kern0pt}extension{\isacharunderscore}{\kern0pt}is{\isacharunderscore}{\kern0pt}SRB\ SRB{\isacharunderscore}{\kern0pt}is{\isacharunderscore}{\kern0pt}GSRB{\isacharparenright}{\kern0pt}%
\endisatagproof
{\isafoldproof}%
%
\isadelimproof
%
\endisadelimproof
%
\begin{isamarkuptext}%
This, now, directly implies that GSRBs do charactarise strong reactive/$X$-bisimilarity.%
\end{isamarkuptext}\isamarkuptrue%
%
\isadelimvisible
%
\endisadelimvisible
%
\isatagvisible
\isacommand{proposition}\isamarkupfalse%
\ GSRBs{\isacharunderscore}{\kern0pt}characterise{\isacharunderscore}{\kern0pt}strong{\isacharunderscore}{\kern0pt}reactive{\isacharunderscore}{\kern0pt}bisimilarity{\isacharcolon}{\kern0pt}\isanewline
\ \ {\isacartoucheopen}p\ {\isasymleftrightarrow}\isactrlsub r\ q\ {\isasymLongleftrightarrow}\ {\isacharparenleft}{\kern0pt}{\isasymexists}\ R{\isachardot}{\kern0pt}\ GSRB\ R\ {\isasymand}\ R\ p\ None\ q{\isacharparenright}{\kern0pt}{\isacartoucheclose}\isanewline
\ \ \isacommand{using}\isamarkupfalse%
\ GSRB{\isacharunderscore}{\kern0pt}whenever{\isacharunderscore}{\kern0pt}SRB\ strongly{\isacharunderscore}{\kern0pt}reactive{\isacharunderscore}{\kern0pt}bisimilar{\isacharunderscore}{\kern0pt}def\ \isacommand{by}\isamarkupfalse%
\ blast\isanewline
\isanewline
\isacommand{proposition}\isamarkupfalse%
\ GSRBs{\isacharunderscore}{\kern0pt}characterise{\isacharunderscore}{\kern0pt}strong{\isacharunderscore}{\kern0pt}X{\isacharunderscore}{\kern0pt}bisimilarity{\isacharcolon}{\kern0pt}\isanewline
\ \ {\isacartoucheopen}p\ {\isasymleftrightarrow}\isactrlsub r\isactrlsup X\ q\ {\isasymLongleftrightarrow}\ {\isacharparenleft}{\kern0pt}{\isasymexists}\ R{\isachardot}{\kern0pt}\ GSRB\ R\ {\isasymand}\ R\ p\ {\isacharparenleft}{\kern0pt}Some\ X{\isacharparenright}{\kern0pt}\ q{\isacharparenright}{\kern0pt}{\isacartoucheclose}\isanewline
\ \ \isacommand{using}\isamarkupfalse%
\ GSRB{\isacharunderscore}{\kern0pt}whenever{\isacharunderscore}{\kern0pt}SRB\ strongly{\isacharunderscore}{\kern0pt}X{\isacharunderscore}{\kern0pt}bisimilar{\isacharunderscore}{\kern0pt}def\ \isacommand{by}\isamarkupfalse%
\ blast%
\endisatagvisible
{\isafoldvisible}%
%
\isadelimvisible
%
\endisadelimvisible
\isanewline
\isanewline
\isacommand{end}\isamarkupfalse%
\ %
\isamarkupcmt{of \isa{context\ lts{\isacharunderscore}{\kern0pt}timeout}%
}%
\begin{isamarkuptext}%
As a little meta-comment, I would like to point out that van~Glabbeek's proof spans a total of five lines (\enquote{Clearly, \textelp{}}, \enquote{It is straightforward to check \textelp{}}), whereas the Isabelle proof takes up around 250 lines of code. This just goes to show that for things which are clear and straightforward for humans, it might require quite some effort to \enquote{explain} them to a computer.
\pagebreak%
\end{isamarkuptext}\isamarkuptrue%
%
\isadelimtheory
%
\endisadelimtheory
%
\isatagtheory
%
\endisatagtheory
{\isafoldtheory}%
%
\isadelimtheory
%
\endisadelimtheory
%
\end{isabellebody}%
\endinput
%:%file=~/projects/Reducing-Reactive-to-Strong-Bisimilarity/isabelle/Reactive_Bisimilarity.thy%:%
%:%24=8%:%
%:%36=9%:%
%:%40=11%:%
%:%41=12%:%
%:%42=13%:%
%:%43=14%:%
%:%44=15%:%
%:%45=16%:%
%:%46=17%:%
%:%47=18%:%
%:%48=19%:%
%:%49=20%:%
%:%50=21%:%
%:%51=22%:%
%:%52=23%:%
%:%53=24%:%
%:%54=25%:%
%:%55=26%:%
%:%56=27%:%
%:%57=28%:%
%:%58=29%:%
%:%59=30%:%
%:%60=31%:%
%:%61=32%:%
%:%62=33%:%
%:%63=34%:%
%:%64=35%:%
%:%65=36%:%
%:%74=38%:%
%:%86=39%:%
%:%87=40%:%
%:%88=41%:%
%:%89=42%:%
%:%90=43%:%
%:%91=44%:%
%:%92=45%:%
%:%93=46%:%
%:%94=47%:%
%:%95=48%:%
%:%96=49%:%
%:%97=50%:%
%:%98=51%:%
%:%99=52%:%
%:%100=53%:%
%:%101=54%:%
%:%102=55%:%
%:%103=56%:%
%:%104=57%:%
%:%105=58%:%
%:%106=59%:%
%:%107=60%:%
%:%108=61%:%
%:%109=62%:%
%:%110=63%:%
%:%111=64%:%
%:%112=65%:%
%:%113=66%:%
%:%114=67%:%
%:%115=68%:%
%:%116=69%:%
%:%117=70%:%
%:%118=71%:%
%:%119=72%:%
%:%120=73%:%
%:%121=74%:%
%:%130=77%:%
%:%142=78%:%
%:%151=81%:%
%:%163=82%:%
%:%172=85%:%
%:%184=87%:%
%:%193=90%:%
%:%205=92%:%
%:%206=93%:%
%:%207=94%:%
%:%208=95%:%
%:%209=96%:%
%:%210=97%:%
%:%211=98%:%
%:%212=99%:%
%:%213=100%:%
%:%214=101%:%
%:%215=102%:%
%:%216=103%:%
%:%217=104%:%
%:%218=105%:%
%:%220=107%:%
%:%221=107%:%
%:%222=108%:%
%:%224=109%:%
%:%225=109%:%
%:%226=110%:%
%:%227=110%:%
%:%228=111%:%
%:%242=125%:%
%:%243=126%:%
%:%264=143%:%
%:%276=145%:%
%:%278=147%:%
%:%279=147%:%
%:%280=148%:%
%:%281=149%:%
%:%282=150%:%
%:%283=151%:%
%:%284=151%:%
%:%285=152%:%
%:%286=153%:%
%:%288=155%:%
%:%290=157%:%
%:%291=157%:%
%:%292=158%:%
%:%293=159%:%
%:%296=162%:%
%:%298=164%:%
%:%299=164%:%
%:%302=165%:%
%:%306=165%:%
%:%307=165%:%
%:%308=165%:%
%:%313=165%:%
%:%316=166%:%
%:%317=166%:%
%:%320=167%:%
%:%324=167%:%
%:%325=167%:%
%:%326=167%:%
%:%340=170%:%
%:%352=172%:%
%:%355=174%:%
%:%356=174%:%
%:%357=175%:%
%:%358=175%:%
%:%359=176%:%
%:%375=192%:%
%:%376=193%:%
%:%397=210%:%
%:%409=212%:%
%:%411=214%:%
%:%412=214%:%
%:%413=215%:%
%:%414=216%:%
%:%417=217%:%
%:%421=217%:%
%:%422=217%:%
%:%423=218%:%
%:%424=218%:%
%:%425=219%:%
%:%426=219%:%
%:%427=220%:%
%:%428=220%:%
%:%429=221%:%
%:%430=221%:%
%:%431=222%:%
%:%432=222%:%
%:%433=222%:%
%:%434=223%:%
%:%435=223%:%
%:%436=224%:%
%:%437=224%:%
%:%438=225%:%
%:%439=225%:%
%:%440=226%:%
%:%441=226%:%
%:%442=227%:%
%:%443=227%:%
%:%444=227%:%
%:%445=228%:%
%:%446=228%:%
%:%447=229%:%
%:%448=229%:%
%:%449=230%:%
%:%450=230%:%
%:%451=231%:%
%:%452=231%:%
%:%453=232%:%
%:%454=232%:%
%:%455=232%:%
%:%456=233%:%
%:%457=233%:%
%:%458=234%:%
%:%459=234%:%
%:%460=235%:%
%:%461=235%:%
%:%462=236%:%
%:%463=236%:%
%:%464=237%:%
%:%465=237%:%
%:%466=237%:%
%:%467=238%:%
%:%468=238%:%
%:%469=239%:%
%:%470=239%:%
%:%471=240%:%
%:%472=240%:%
%:%473=241%:%
%:%474=241%:%
%:%475=242%:%
%:%476=242%:%
%:%477=242%:%
%:%478=243%:%
%:%479=243%:%
%:%480=244%:%
%:%481=244%:%
%:%482=245%:%
%:%483=245%:%
%:%484=246%:%
%:%485=246%:%
%:%486=247%:%
%:%487=247%:%
%:%488=247%:%
%:%489=248%:%
%:%490=248%:%
%:%491=249%:%
%:%492=249%:%
%:%493=250%:%
%:%494=250%:%
%:%495=251%:%
%:%496=251%:%
%:%497=251%:%
%:%498=251%:%
%:%499=252%:%
%:%500=252%:%
%:%501=252%:%
%:%502=252%:%
%:%503=253%:%
%:%504=253%:%
%:%505=253%:%
%:%506=253%:%
%:%507=254%:%
%:%508=254%:%
%:%509=255%:%
%:%510=255%:%
%:%511=256%:%
%:%512=256%:%
%:%513=257%:%
%:%514=257%:%
%:%515=258%:%
%:%516=258%:%
%:%517=258%:%
%:%518=259%:%
%:%519=259%:%
%:%520=260%:%
%:%521=260%:%
%:%522=261%:%
%:%523=261%:%
%:%524=262%:%
%:%525=262%:%
%:%526=262%:%
%:%527=263%:%
%:%528=263%:%
%:%529=264%:%
%:%530=264%:%
%:%531=264%:%
%:%532=265%:%
%:%533=265%:%
%:%534=265%:%
%:%535=266%:%
%:%536=266%:%
%:%537=266%:%
%:%538=267%:%
%:%539=267%:%
%:%540=267%:%
%:%541=268%:%
%:%542=268%:%
%:%543=268%:%
%:%544=269%:%
%:%545=269%:%
%:%546=269%:%
%:%547=270%:%
%:%557=272%:%
%:%558=273%:%
%:%560=275%:%
%:%561=275%:%
%:%562=276%:%
%:%563=277%:%
%:%569=283%:%
%:%582=321%:%
%:%585=322%:%
%:%586=322%:%
%:%587=323%:%
%:%588=324%:%
%:%589=325%:%
%:%590=326%:%
%:%593=327%:%
%:%597=327%:%
%:%598=327%:%
%:%599=328%:%
%:%600=328%:%
%:%601=329%:%
%:%602=329%:%
%:%603=330%:%
%:%604=330%:%
%:%605=331%:%
%:%606=331%:%
%:%607=332%:%
%:%608=332%:%
%:%609=333%:%
%:%610=333%:%
%:%611=334%:%
%:%612=334%:%
%:%613=335%:%
%:%614=335%:%
%:%615=336%:%
%:%616=336%:%
%:%617=336%:%
%:%618=337%:%
%:%619=337%:%
%:%620=337%:%
%:%621=338%:%
%:%622=338%:%
%:%623=339%:%
%:%624=339%:%
%:%625=340%:%
%:%626=340%:%
%:%627=341%:%
%:%628=341%:%
%:%629=341%:%
%:%630=342%:%
%:%631=342%:%
%:%632=342%:%
%:%633=343%:%
%:%634=343%:%
%:%635=344%:%
%:%636=344%:%
%:%637=345%:%
%:%638=345%:%
%:%639=345%:%
%:%640=345%:%
%:%641=346%:%
%:%642=346%:%
%:%643=347%:%
%:%644=347%:%
%:%645=347%:%
%:%646=348%:%
%:%647=348%:%
%:%648=349%:%
%:%649=349%:%
%:%650=349%:%
%:%651=350%:%
%:%652=350%:%
%:%653=350%:%
%:%654=350%:%
%:%655=351%:%
%:%656=351%:%
%:%657=352%:%
%:%658=352%:%
%:%659=353%:%
%:%660=353%:%
%:%661=354%:%
%:%662=354%:%
%:%663=355%:%
%:%664=355%:%
%:%665=356%:%
%:%666=356%:%
%:%667=357%:%
%:%668=357%:%
%:%669=358%:%
%:%670=358%:%
%:%671=359%:%
%:%672=359%:%
%:%673=359%:%
%:%674=359%:%
%:%675=360%:%
%:%676=360%:%
%:%677=361%:%
%:%678=361%:%
%:%679=362%:%
%:%680=362%:%
%:%681=362%:%
%:%682=363%:%
%:%683=363%:%
%:%684=364%:%
%:%685=364%:%
%:%686=365%:%
%:%687=365%:%
%:%688=365%:%
%:%689=366%:%
%:%690=366%:%
%:%691=367%:%
%:%692=367%:%
%:%693=368%:%
%:%694=368%:%
%:%695=369%:%
%:%696=369%:%
%:%697=370%:%
%:%698=370%:%
%:%699=371%:%
%:%700=371%:%
%:%701=372%:%
%:%702=372%:%
%:%703=373%:%
%:%704=373%:%
%:%705=374%:%
%:%706=374%:%
%:%707=374%:%
%:%708=375%:%
%:%709=375%:%
%:%710=375%:%
%:%711=376%:%
%:%712=376%:%
%:%713=376%:%
%:%714=377%:%
%:%715=377%:%
%:%716=378%:%
%:%717=378%:%
%:%718=379%:%
%:%719=379%:%
%:%720=379%:%
%:%721=380%:%
%:%722=380%:%
%:%723=380%:%
%:%724=381%:%
%:%725=381%:%
%:%726=382%:%
%:%727=382%:%
%:%728=383%:%
%:%729=383%:%
%:%730=384%:%
%:%731=384%:%
%:%732=385%:%
%:%733=385%:%
%:%734=385%:%
%:%735=386%:%
%:%736=386%:%
%:%737=386%:%
%:%738=387%:%
%:%739=387%:%
%:%740=388%:%
%:%741=388%:%
%:%742=389%:%
%:%743=389%:%
%:%744=390%:%
%:%745=390%:%
%:%746=391%:%
%:%747=391%:%
%:%748=392%:%
%:%749=392%:%
%:%750=393%:%
%:%751=393%:%
%:%752=394%:%
%:%753=394%:%
%:%754=395%:%
%:%755=395%:%
%:%756=395%:%
%:%757=396%:%
%:%758=396%:%
%:%759=397%:%
%:%760=397%:%
%:%761=398%:%
%:%762=398%:%
%:%763=398%:%
%:%764=399%:%
%:%765=399%:%
%:%766=399%:%
%:%767=400%:%
%:%768=400%:%
%:%769=401%:%
%:%770=401%:%
%:%771=402%:%
%:%772=402%:%
%:%773=402%:%
%:%774=403%:%
%:%775=403%:%
%:%776=403%:%
%:%777=403%:%
%:%778=404%:%
%:%779=404%:%
%:%780=405%:%
%:%781=405%:%
%:%782=406%:%
%:%783=406%:%
%:%784=407%:%
%:%785=407%:%
%:%786=408%:%
%:%787=408%:%
%:%788=409%:%
%:%789=409%:%
%:%790=410%:%
%:%791=410%:%
%:%792=411%:%
%:%793=411%:%
%:%794=412%:%
%:%795=412%:%
%:%796=413%:%
%:%797=413%:%
%:%798=413%:%
%:%799=414%:%
%:%800=414%:%
%:%801=415%:%
%:%802=415%:%
%:%803=416%:%
%:%804=416%:%
%:%805=417%:%
%:%806=417%:%
%:%807=417%:%
%:%808=418%:%
%:%809=418%:%
%:%810=419%:%
%:%811=419%:%
%:%812=420%:%
%:%813=420%:%
%:%814=420%:%
%:%815=420%:%
%:%816=421%:%
%:%817=421%:%
%:%818=421%:%
%:%819=422%:%
%:%820=422%:%
%:%821=422%:%
%:%822=423%:%
%:%823=423%:%
%:%824=423%:%
%:%825=424%:%
%:%826=424%:%
%:%827=424%:%
%:%828=425%:%
%:%829=425%:%
%:%830=426%:%
%:%831=426%:%
%:%832=427%:%
%:%833=427%:%
%:%834=428%:%
%:%835=428%:%
%:%836=429%:%
%:%837=429%:%
%:%838=430%:%
%:%839=430%:%
%:%840=431%:%
%:%841=431%:%
%:%842=432%:%
%:%843=432%:%
%:%844=433%:%
%:%845=433%:%
%:%846=434%:%
%:%847=434%:%
%:%848=434%:%
%:%849=435%:%
%:%850=435%:%
%:%851=436%:%
%:%852=436%:%
%:%853=437%:%
%:%854=437%:%
%:%855=438%:%
%:%856=438%:%
%:%857=438%:%
%:%858=439%:%
%:%859=439%:%
%:%860=440%:%
%:%861=440%:%
%:%862=441%:%
%:%863=441%:%
%:%864=441%:%
%:%865=441%:%
%:%866=442%:%
%:%867=442%:%
%:%868=442%:%
%:%869=443%:%
%:%870=443%:%
%:%871=443%:%
%:%872=444%:%
%:%873=444%:%
%:%874=444%:%
%:%875=445%:%
%:%876=445%:%
%:%877=446%:%
%:%878=446%:%
%:%879=447%:%
%:%880=447%:%
%:%881=448%:%
%:%882=448%:%
%:%883=449%:%
%:%884=449%:%
%:%885=450%:%
%:%886=450%:%
%:%887=450%:%
%:%888=451%:%
%:%889=451%:%
%:%890=452%:%
%:%891=452%:%
%:%892=453%:%
%:%893=453%:%
%:%894=454%:%
%:%895=454%:%
%:%896=455%:%
%:%897=455%:%
%:%898=456%:%
%:%899=456%:%
%:%900=457%:%
%:%901=457%:%
%:%902=458%:%
%:%903=458%:%
%:%904=458%:%
%:%905=458%:%
%:%906=459%:%
%:%907=459%:%
%:%908=459%:%
%:%909=459%:%
%:%910=460%:%
%:%911=460%:%
%:%912=461%:%
%:%913=461%:%
%:%914=461%:%
%:%915=462%:%
%:%916=462%:%
%:%917=463%:%
%:%918=463%:%
%:%919=464%:%
%:%920=464%:%
%:%921=465%:%
%:%922=465%:%
%:%923=466%:%
%:%924=466%:%
%:%925=467%:%
%:%926=467%:%
%:%927=468%:%
%:%928=468%:%
%:%929=469%:%
%:%930=469%:%
%:%931=469%:%
%:%932=469%:%
%:%933=470%:%
%:%934=470%:%
%:%935=470%:%
%:%936=471%:%
%:%937=471%:%
%:%938=472%:%
%:%939=472%:%
%:%940=472%:%
%:%941=472%:%
%:%942=473%:%
%:%943=473%:%
%:%944=474%:%
%:%945=474%:%
%:%946=474%:%
%:%947=475%:%
%:%948=475%:%
%:%949=476%:%
%:%959=478%:%
%:%961=480%:%
%:%962=480%:%
%:%963=481%:%
%:%966=482%:%
%:%970=482%:%
%:%971=482%:%
%:%980=484%:%
%:%988=486%:%
%:%989=486%:%
%:%990=487%:%
%:%991=488%:%
%:%992=488%:%
%:%993=488%:%
%:%994=489%:%
%:%995=490%:%
%:%996=490%:%
%:%997=491%:%
%:%998=492%:%
%:%999=492%:%
%:%1000=492%:%
%:%1007=492%:%
%:%1008=493%:%
%:%1009=494%:%
%:%1010=494%:%
%:%1011=494%:%
%:%1014=497%:%
%:%1015=498%:%
%
\begin{isabellebody}%
\setisabellecontext{Mapping{\isacharunderscore}{\kern0pt}for{\isacharunderscore}{\kern0pt}Transition{\isacharunderscore}{\kern0pt}Systems}%
%
\isadelimtheory
%
\endisadelimtheory
%
\isatagtheory
%
\endisatagtheory
{\isafoldtheory}%
%
\isadelimtheory
%
\endisadelimtheory
%
\isadelimdocument
%
\endisadelimdocument
%
\isatagdocument
%
\isamarkupsection{A Mapping for Transition Systems%
}
\isamarkuptrue%
%
\endisatagdocument
{\isafolddocument}%
%
\isadelimdocument
%
\endisadelimdocument
%
\begin{isamarkuptext}%
\label{sec:mapping_lts}%
\end{isamarkuptext}\isamarkuptrue%
%
\begin{isamarkuptext}%
Let $\mathbb{T} = (\Proc, \Act, \rightarrow)$ be an \LTSt{}. Let $A = \Act \!\setminus\! \{\tau, t\}$.

In reference to van~Glabbeek's $\theta_X$-operators, I introduce a family of operators $\vartheta_X$ with similar but not identical semantics. Additionally, I introduce the operator $\vartheta$ that places a process in an indeterminate environment.

Furthermore, I introduce a family of special actions $\varepsilon_X$ for $X \subseteq A$ that represent a triggered environment stabilising into an environment~$X$, as well as a special action $t_\varepsilon$ that represents a time-out of the environment.

We assume that $t_\varepsilon \notin \Act$ and $\forall X \subseteq A.\; \varepsilon_X \notin \Act$. 

Then we define $\mathbb{T}_\vartheta = (\Proc_\vartheta, \Act_\vartheta, \rightarrow_\vartheta)$ with
\begin{align*}
    \Proc_\vartheta &= \{ \vartheta(p) \mid p \in \Proc \} \cup \{ \vartheta_X(p) \mid p \in \Proc \wedge X \subseteq A \}, \\
    \Act_\vartheta &= \Act \cup \{t_\varepsilon\} \cup \{ \varepsilon_X \mid X \subseteq A \},
\end{align*}
and $\rightarrow_\vartheta$ defined by the following rules:

$$
(1)\,\frac{}{\vartheta(p) \xrightarrow{\varepsilon_X}_\vartheta \vartheta_X(p)} \; X \subseteq A
\qquad
(2)\,\frac{p \xrightarrow{\tau} p'}{\vartheta(p) \xrightarrow{\tau}_\vartheta \vartheta(p')}
$$
\\[-5pt]
$$
(3)\,\frac{p \not\xrightarrow{\alpha} \text{ for all } \alpha \in X \cup \{\tau\}}
{\vartheta_X(p) \xrightarrow{t_\varepsilon}_\vartheta \vartheta(p)}
$$
\\[-5pt]
$$
(4)\,\frac{p \xrightarrow{a} p'}{\vartheta_X(p) \xrightarrow{a}_\vartheta \vartheta(p')} \; a \in X
\qquad
(5)\,\frac{p \xrightarrow{\tau} p'}{\vartheta_X(p) \xrightarrow{\tau}_\vartheta \vartheta_X(p')}
$$
\\[-5pt]
$$
(6)\,\frac{p \not\xrightarrow{\alpha} \text{ for all } \alpha \in X \cup \{\tau\} \quad p \xrightarrow{t} p'}
{\vartheta_X(p) \xrightarrow{t}_\vartheta \vartheta_X(p')}
$$\\

These rules are motivated by the intuitions developed in \cref{sec:LTSt,sec:reactive_bisimilarity}:

\begin{enumerate}[nosep]
    \item triggered environments can stabilise into arbitrary stable environments~$X$ for $X \subseteq A$,
    \item $\tau$-transitions can be performed regardless of the environment,
    \item if the underlying system is idle, the environment may time-out and turn into an indeterminate/triggered environment,
    \item facilitated visible transitions can be performed and can trigger a change in the environment,
    \item $\tau$-transitions cannot be observed by the environment and hence cannot trigger a change,
    \item if the underlying system is idle and has a $t$-transition, the transition may be performed and is not observable by the environment.
\end{enumerate}

\example{%
The \LTSt{} on the left (with $\Act = \{a,\tau,t\}$) gets mapped to the LTS on the right. States that have no incoming or outgoing transitions other than $\varepsilon_X$ or $t_\varepsilon$ are omitted. Note how $\vartheta(T)$ is not reachable from $\vartheta(P)$.

\phantomsection
\label{fig:mapping_example}
\lts{
    \node[state]    (P)                             {$p$};
    \node[state]    (Q) [below left of=P]           {$p_1$};
    \node[state]    (R) [below right of=P]          {$p_2$};
    \node[state]    (S) [below left of=R]           {$p_3$};
    \node[state]    (T) [below right of=R]          {$p_4$};
    
    \path   (P) edge node[above left]               {$a$}               (Q)
                edge node[above right]              {$t$}               (R)
            (R) edge node[above left]               {$\tau$}            (S)
                edge node[above right]              {$a$}               (T);


    \node[state]    (Pt) [right of=P, xshift=4.5cm] {$\vartheta(p)$};
    \node[state]    (Pa) [below left of=Pt]         {$\vartheta_{\{a\}}(p)$};
    \node[state]    (Pe) [below right of=Pt]        {$\vartheta_\emptyset(p)$};
    \node[state]    (Rt) [right of=Pe]              {$\vartheta(p_2)$};
    \node[state]    (Qt) [below of=Pa]              {$\vartheta(p_1)$};
    \node[state]    (Re) [below of=Pe]              {$\vartheta_\emptyset(p_2)$};
    \node[state]    (Se) [below of=Re]              {$\vartheta_\emptyset(p_3)$};
    \node[state]    (Ra) [below of=Rt,xshift=1.8cm] {$\vartheta_{\{a\}}(p_2)$};
    \node[state]    (Tt) [right of=Ra,xshift=.7cm]  {$\vartheta(p_4)$};
    \node[state]    (Sa) [below of=Ra]              {$\vartheta_{\{a\}}(p_3)$};
    \node[state]    (St) [right of=Se,yshift=-2cm]  {$\vartheta(p_3)$};
    
    \path   (Pt)    edge node[above left]           {$\varepsilon_{\{a\}}$}     (Pa)
            (Pt)    edge node[left]                 {$\varepsilon_\emptyset$}   (Pe)
            (Pe)    edge[bend right] node[right]    {$t_\varepsilon$}           (Pt)
            (Pa)    edge node[left]                 {$a$}                       (Qt)
            (Pe)    edge node[left]                 {$t$}                       (Re)
            (Re)    edge node[left]                 {$\tau$}                    (Se)
            (Rt)    edge node[right]                {$\varepsilon_{\{a\}}$}     (Ra)
            (Rt)    edge node[left]                 {$\varepsilon_\emptyset$}   (Re)
            (Ra)    edge node[above]                {$a$}                       (Tt)
            (Ra)    edge node[left]                 {$\tau$}                    (Sa)
            (Se)    edge node[above]                {$t_\varepsilon$}           (St)
            (St)    edge[bend left] node[left]      {$\varepsilon_\emptyset$}   (Se)
            (Sa)    edge node[above]                {$t_\varepsilon$}           (St)
            (St)    edge[bend right] node[right]    {$\varepsilon_{\{a\}}$}   (Sa)
            (Rt)    edge node[right]                {$\tau$}                    (St);
            
    \coordinate [below of=Qt,yshift=10pt] (Qc);
    \coordinate [below of=Tt,yshift=10pt] (Tc);
            
    \draw [dotted,->,shorten >= 7pt] (Qt) to[bend right] node[left] {$\varepsilon_{\dots}$} (Qc);
    \draw [dotted,->,shorten <= 7pt] (Qc) to[bend right] node[right] {$t_\varepsilon$} (Qt);
    \draw [dotted,->,shorten >= 7pt] (Tt) to[bend right] node[left] {$\varepsilon_{\dots}$} (Tc);
    \draw [dotted,->,shorten <= 7pt] (Tc) to[bend right] node[right] {$t_\varepsilon$} (Tt);
}}%
\end{isamarkuptext}\isamarkuptrue%
%
\isadelimdocument
%
\endisadelimdocument
%
\isatagdocument
%
\isamarkupsubsection{Isabelle%
}
\isamarkuptrue%
%
\isamarkupsubsubsection{Formalising \boldmath{$\Proc_\vartheta$} and \boldmath{$\Act_\vartheta$}%
}
\isamarkuptrue%
%
\endisatagdocument
{\isafolddocument}%
%
\isadelimdocument
%
\endisadelimdocument
%
\begin{isamarkuptext}%
For the formalisation, we specify another type of LTS based on \isa{lts{\isacharunderscore}{\kern0pt}timeout}, where the aforementioned special actions and operators are considered; we call it \isa{lts{\isacharunderscore}{\kern0pt}timeout{\isacharunderscore}{\kern0pt}mappable}. 
Since $\Proc \cap \Proc_\vartheta = \emptyset$, we introduce a new type variable \isa{{\isacharprime}{\kern0pt}ss} for $\Proc_\vartheta$, but use \isa{{\isacharprime}{\kern0pt}a} for both $\Act$ and $\Act_\vartheta$.
We formalise the family of special actions $\varepsilon_X$ as a mapping \isa{{\isasymepsilon}{\isacharbrackleft}{\kern0pt}{\isacharunderscore}{\kern0pt}{\isacharbrackright}{\kern0pt}\ {\isacharcolon}{\kern0pt}{\isacharcolon}{\kern0pt}\ {\isacharprime}{\kern0pt}a\ set\ {\isasymRightarrow}\ {\isacharprime}{\kern0pt}a}, and the environment operators $\vartheta$/$\vartheta_X$ as a single mapping \isa{{\isasymtheta}{\isacharquery}{\kern0pt}{\isacharbrackleft}{\kern0pt}{\isacharunderscore}{\kern0pt}{\isacharbrackright}{\kern0pt}{\isacharparenleft}{\kern0pt}{\isacharunderscore}{\kern0pt}{\isacharparenright}{\kern0pt}\ {\isacharcolon}{\kern0pt}{\isacharcolon}{\kern0pt}\ {\isacharprime}{\kern0pt}a\ set\ option\ {\isasymRightarrow}\ {\isacharprime}{\kern0pt}s\ {\isasymRightarrow}\ {\isacharprime}{\kern0pt}ss}.

As for \isa{lts{\isacharunderscore}{\kern0pt}timeout} in \cref{sec:LTSt}, we require that no two special actions are equal, formalised by the first set of assumptions \isa{distinctness{\isacharunderscore}{\kern0pt}special{\isacharunderscore}{\kern0pt}actions}.

As an operator, the term $\vartheta_X(p)$ simply refers to the specific state $p$ in an environment~$X$; when understood as a mapping, we have to be more careful, since \isa{{\isasymtheta}{\isacharquery}{\kern0pt}{\isacharbrackleft}{\kern0pt}Some\ X{\isacharbrackright}{\kern0pt}{\isacharparenleft}{\kern0pt}p{\isacharparenright}{\kern0pt}} is now a state itself. Specifically, we have to assume that the mapping \isa{{\isasymtheta}{\isacharquery}{\kern0pt}{\isacharbrackleft}{\kern0pt}{\isacharunderscore}{\kern0pt}{\isacharbrackright}{\kern0pt}{\isacharparenleft}{\kern0pt}{\isacharunderscore}{\kern0pt}{\isacharparenright}{\kern0pt}} is injective (when restricted to domains where \isa{X\ {\isasymsubseteq}\ visible{\isacharunderscore}{\kern0pt}actions}). Otherwise, we might have that \isa{{\isasymtheta}{\isacharquery}{\kern0pt}{\isacharbrackleft}{\kern0pt}None{\isacharbrackright}{\kern0pt}{\isacharparenleft}{\kern0pt}p{\isacharparenright}{\kern0pt}\ {\isacharequal}{\kern0pt}\ {\isasymtheta}{\isacharquery}{\kern0pt}{\isacharbrackleft}{\kern0pt}None{\isacharbrackright}{\kern0pt}{\isacharparenleft}{\kern0pt}q{\isacharparenright}{\kern0pt}} for \isa{p\ {\isasymnoteq}\ q}, which is obviously problematic. Hence, the (restricted) injectivity of \isa{{\isasymtheta}{\isacharquery}{\kern0pt}{\isacharbrackleft}{\kern0pt}{\isacharunderscore}{\kern0pt}{\isacharbrackright}{\kern0pt}{\isacharparenleft}{\kern0pt}{\isacharunderscore}{\kern0pt}{\isacharparenright}{\kern0pt}} is formalised as the set of assumptions \isa{injectivity{\isacharunderscore}{\kern0pt}theta}. (We do not require injectivity for subsets of the domain with \isa{{\isasymnot}\ X\ {\isasymsubseteq}\ visible{\isacharunderscore}{\kern0pt}actions}, since $\vartheta_X$ is not defined for those $X$).

The same is required for the mapping \isa{{\isasymepsilon}{\isacharbrackleft}{\kern0pt}{\isacharunderscore}{\kern0pt}{\isacharbrackright}{\kern0pt}}, as formalised in the last clause of the set of assumptions \isa{distinctness{\isacharunderscore}{\kern0pt}special{\isacharunderscore}{\kern0pt}actions} (the (restricted) injectivity of \isa{{\isasymepsilon}{\isacharbrackleft}{\kern0pt}{\isacharunderscore}{\kern0pt}{\isacharbrackright}{\kern0pt}} is part of the requirement that all special actions must be distinct). Again, we only require injectivity for the mapping restricted to the domain \isa{visible{\isacharunderscore}{\kern0pt}actions}. If we would require that \isa{{\isasymepsilon}{\isacharbrackleft}{\kern0pt}{\isacharunderscore}{\kern0pt}{\isacharbrackright}{\kern0pt}\ {\isacharcolon}{\kern0pt}{\isacharcolon}{\kern0pt}\ {\isacharprime}{\kern0pt}a\ set\ {\isasymRightarrow}\ {\isacharprime}{\kern0pt}a} were injective over its entire domain \isa{{\isacharprime}{\kern0pt}a\ set}, we would run into problems, since such a function cannot exist by Cantor's theorem.

For now, we assume that such mappings exist, in order to proof the general case for any such mappings. In \cref{chap:example_instantiation}, I give examples for these mappings and show that, with these, any \isa{lts{\isacharunderscore}{\kern0pt}timeout} can be interpreted as an \isa{lts{\isacharunderscore}{\kern0pt}timeout{\isacharunderscore}{\kern0pt}mappable}.

Lastly, we formalise our requirements $t_\varepsilon \notin \Act$ and $\forall X \subseteq A.\; \varepsilon_X \notin \Act$ as the last set of assumptions \isa{no{\isacharunderscore}{\kern0pt}epsilon{\isacharunderscore}{\kern0pt}in{\isacharunderscore}{\kern0pt}tran}. Technically, these assumptions only state that the $\varepsilon$-actions do not label any transition of $\mathbb{T}$. However, we can assume that 
$\Act = \{ \alpha \mid \exists p, p'.\; p \xrightarrow{\alpha} p' \}$, 
since actions that do not label any transitions are not relevant to the behaviour of an LTS.%
\end{isamarkuptext}\isamarkuptrue%
\isacommand{locale}\isamarkupfalse%
\ lts{\isacharunderscore}{\kern0pt}timeout{\isacharunderscore}{\kern0pt}mappable\ {\isacharequal}{\kern0pt}\ lts{\isacharunderscore}{\kern0pt}timeout\ tran\ {\isasymtau}\ t\ \isanewline
\ \ \isakeyword{for}\ tran\ {\isacharcolon}{\kern0pt}{\isacharcolon}{\kern0pt}\ {\isachardoublequoteopen}{\isacharprime}{\kern0pt}s\ {\isasymRightarrow}\ {\isacharprime}{\kern0pt}a\ {\isasymRightarrow}\ {\isacharprime}{\kern0pt}s\ {\isasymRightarrow}\ bool{\isachardoublequoteclose}\ \isanewline
\ \ \ \ \ \ {\isacharparenleft}{\kern0pt}{\isachardoublequoteopen}{\isacharunderscore}{\kern0pt}\ {\isasymlongmapsto}{\isacharunderscore}{\kern0pt}\ {\isacharunderscore}{\kern0pt}{\isachardoublequoteclose}\ {\isacharbrackleft}{\kern0pt}{\isadigit{7}}{\isadigit{0}}{\isacharcomma}{\kern0pt}\ {\isadigit{7}}{\isadigit{0}}{\isacharcomma}{\kern0pt}\ {\isadigit{7}}{\isadigit{0}}{\isacharbrackright}{\kern0pt}\ {\isadigit{8}}{\isadigit{0}}{\isacharparenright}{\kern0pt}\ \isanewline
\ \ \ \ \isakeyword{and}\ {\isasymtau}\ {\isacharcolon}{\kern0pt}{\isacharcolon}{\kern0pt}\ {\isacharprime}{\kern0pt}a\ \isanewline
\ \ \ \ \isakeyword{and}\ t\ {\isacharcolon}{\kern0pt}{\isacharcolon}{\kern0pt}\ {\isacharprime}{\kern0pt}a\ {\isacharplus}{\kern0pt}\isanewline
\ \ \isakeyword{fixes}\ t{\isacharunderscore}{\kern0pt}{\isasymepsilon}\ {\isacharcolon}{\kern0pt}{\isacharcolon}{\kern0pt}\ {\isacharprime}{\kern0pt}a\ \isanewline
\ \ \ \ \isakeyword{and}\ stabilise\ {\isacharcolon}{\kern0pt}{\isacharcolon}{\kern0pt}\ {\isacartoucheopen}{\isacharprime}{\kern0pt}a\ set\ {\isasymRightarrow}\ {\isacharprime}{\kern0pt}a{\isacartoucheclose}\ \isanewline
\ \ \ \ \ \ {\isacharparenleft}{\kern0pt}{\isacartoucheopen}{\isasymepsilon}{\isacharbrackleft}{\kern0pt}{\isacharunderscore}{\kern0pt}{\isacharbrackright}{\kern0pt}{\isacartoucheclose}{\isacharparenright}{\kern0pt}\ \isanewline
\ \ \ \ \isakeyword{and}\ in{\isacharunderscore}{\kern0pt}env\ {\isacharcolon}{\kern0pt}{\isacharcolon}{\kern0pt}\ {\isacartoucheopen}{\isacharprime}{\kern0pt}a\ set\ option\ {\isasymRightarrow}\ {\isacharprime}{\kern0pt}s\ {\isasymRightarrow}\ {\isacharprime}{\kern0pt}ss{\isacartoucheclose}\ \isanewline
\ \ \ \ \ \ {\isacharparenleft}{\kern0pt}{\isacartoucheopen}{\isasymtheta}{\isacharquery}{\kern0pt}{\isacharbrackleft}{\kern0pt}{\isacharunderscore}{\kern0pt}{\isacharbrackright}{\kern0pt}{\isacharprime}{\kern0pt}{\isacharparenleft}{\kern0pt}{\isacharunderscore}{\kern0pt}{\isacharprime}{\kern0pt}{\isacharparenright}{\kern0pt}{\isacartoucheclose}{\isacharparenright}{\kern0pt}\isanewline
\ \ \isakeyword{assumes}\isanewline
\ \ \ \ distinctness{\isacharunderscore}{\kern0pt}special{\isacharunderscore}{\kern0pt}actions{\isacharcolon}{\kern0pt}\isanewline
\ \ \ \ {\isacartoucheopen}{\isasymtau}\ {\isasymnoteq}\ t{\isacartoucheclose}\ {\isacartoucheopen}{\isasymtau}\ {\isasymnoteq}\ t{\isacharunderscore}{\kern0pt}{\isasymepsilon}{\isacartoucheclose}\ {\isacartoucheopen}t\ {\isasymnoteq}\ t{\isacharunderscore}{\kern0pt}{\isasymepsilon}{\isacartoucheclose}\isanewline
\ \ \ \ {\isacartoucheopen}{\isasymepsilon}{\isacharbrackleft}{\kern0pt}X{\isacharbrackright}{\kern0pt}\ {\isasymnoteq}\ {\isasymtau}{\isacartoucheclose}\ {\isacartoucheopen}{\isasymepsilon}{\isacharbrackleft}{\kern0pt}X{\isacharbrackright}{\kern0pt}\ {\isasymnoteq}\ t{\isacartoucheclose}\ {\isacartoucheopen}{\isasymepsilon}{\isacharbrackleft}{\kern0pt}X{\isacharbrackright}{\kern0pt}\ {\isasymnoteq}\ t{\isacharunderscore}{\kern0pt}{\isasymepsilon}{\isacartoucheclose}\ \isanewline
\ \ \ \ {\isacartoucheopen}X\ {\isasymsubseteq}\ visible{\isacharunderscore}{\kern0pt}actions\ {\isasymLongrightarrow}\ {\isasymepsilon}{\isacharbrackleft}{\kern0pt}X{\isacharbrackright}{\kern0pt}\ {\isacharequal}{\kern0pt}\ {\isasymepsilon}{\isacharbrackleft}{\kern0pt}Y{\isacharbrackright}{\kern0pt}\ {\isasymLongrightarrow}\ X\ {\isacharequal}{\kern0pt}\ Y{\isacartoucheclose}\isanewline
\ \ \ \ \isakeyword{and}\isanewline
\ \ \ \ \isanewline
\ \ \ \ injectivity{\isacharunderscore}{\kern0pt}theta{\isacharcolon}{\kern0pt}\isanewline
\ \ \ \ {\isacartoucheopen}{\isasymtheta}{\isacharquery}{\kern0pt}{\isacharbrackleft}{\kern0pt}None{\isacharbrackright}{\kern0pt}{\isacharparenleft}{\kern0pt}p{\isacharparenright}{\kern0pt}\ {\isasymnoteq}\ {\isasymtheta}{\isacharquery}{\kern0pt}{\isacharbrackleft}{\kern0pt}Some\ X{\isacharbrackright}{\kern0pt}{\isacharparenleft}{\kern0pt}q{\isacharparenright}{\kern0pt}{\isacartoucheclose}\isanewline
\ \ \ \ {\isacartoucheopen}{\isacharparenleft}{\kern0pt}{\isasymtheta}{\isacharquery}{\kern0pt}{\isacharbrackleft}{\kern0pt}None{\isacharbrackright}{\kern0pt}{\isacharparenleft}{\kern0pt}p{\isacharparenright}{\kern0pt}\ {\isacharequal}{\kern0pt}\ {\isasymtheta}{\isacharquery}{\kern0pt}{\isacharbrackleft}{\kern0pt}None{\isacharbrackright}{\kern0pt}{\isacharparenleft}{\kern0pt}q{\isacharparenright}{\kern0pt}{\isacharparenright}{\kern0pt}\ {\isasymlongrightarrow}\ p\ {\isacharequal}{\kern0pt}\ q{\isacartoucheclose}\isanewline
\ \ \ \ {\isacartoucheopen}X\ {\isasymsubseteq}\ visible{\isacharunderscore}{\kern0pt}actions\ {\isasymLongrightarrow}\ \isanewline
\ \ \ \ \ \ {\isacharparenleft}{\kern0pt}{\isasymtheta}{\isacharquery}{\kern0pt}{\isacharbrackleft}{\kern0pt}Some\ X{\isacharbrackright}{\kern0pt}{\isacharparenleft}{\kern0pt}p{\isacharparenright}{\kern0pt}\ {\isacharequal}{\kern0pt}\ {\isasymtheta}{\isacharquery}{\kern0pt}{\isacharbrackleft}{\kern0pt}Some\ Y{\isacharbrackright}{\kern0pt}{\isacharparenleft}{\kern0pt}q{\isacharparenright}{\kern0pt}{\isacharparenright}{\kern0pt}\ {\isasymlongrightarrow}\ X\ {\isacharequal}{\kern0pt}\ Y\ {\isasymand}\ p\ {\isacharequal}{\kern0pt}\ q{\isacartoucheclose}\isanewline
\ \ \ \ \isakeyword{and}\isanewline
\isanewline
\ \ \ \ no{\isacharunderscore}{\kern0pt}epsilon{\isacharunderscore}{\kern0pt}in{\isacharunderscore}{\kern0pt}tran{\isacharcolon}{\kern0pt}\isanewline
\ \ \ \ {\isacartoucheopen}{\isasymnot}\ p\ {\isasymlongmapsto}{\isasymepsilon}{\isacharbrackleft}{\kern0pt}X{\isacharbrackright}{\kern0pt}\ q{\isacartoucheclose}\isanewline
\ \ \ \ {\isacartoucheopen}{\isasymnot}\ p\ {\isasymlongmapsto}t{\isacharunderscore}{\kern0pt}{\isasymepsilon}\ q{\isacartoucheclose}\isanewline
\isakeyword{begin}%
\begin{isamarkuptext}%
We can now define abbreviations with notations that correspond more closely to our operators defined above.%
\end{isamarkuptext}\isamarkuptrue%
\isacommand{abbreviation}\isamarkupfalse%
\ triggered{\isacharunderscore}{\kern0pt}env\ {\isacharcolon}{\kern0pt}{\isacharcolon}{\kern0pt}\ {\isacartoucheopen}{\isacharprime}{\kern0pt}s\ {\isasymRightarrow}\ {\isacharprime}{\kern0pt}ss{\isacartoucheclose}\ \isanewline
\ \ {\isacharparenleft}{\kern0pt}{\isacartoucheopen}{\isasymtheta}{\isacharprime}{\kern0pt}{\isacharparenleft}{\kern0pt}{\isacharunderscore}{\kern0pt}{\isacharprime}{\kern0pt}{\isacharparenright}{\kern0pt}{\isacartoucheclose}{\isacharparenright}{\kern0pt}\isanewline
\ \ \isakeyword{where}\ {\isacartoucheopen}{\isasymtheta}{\isacharparenleft}{\kern0pt}p{\isacharparenright}{\kern0pt}\ {\isasymequiv}\ {\isasymtheta}{\isacharquery}{\kern0pt}{\isacharbrackleft}{\kern0pt}None{\isacharbrackright}{\kern0pt}{\isacharparenleft}{\kern0pt}p{\isacharparenright}{\kern0pt}{\isacartoucheclose}\isanewline
\isacommand{abbreviation}\isamarkupfalse%
\ stable{\isacharunderscore}{\kern0pt}env\ {\isacharcolon}{\kern0pt}{\isacharcolon}{\kern0pt}\ {\isacartoucheopen}{\isacharprime}{\kern0pt}a\ set\ {\isasymRightarrow}\ {\isacharprime}{\kern0pt}s\ {\isasymRightarrow}\ {\isacharprime}{\kern0pt}ss{\isacartoucheclose}\ \isanewline
\ \ {\isacharparenleft}{\kern0pt}{\isacartoucheopen}{\isasymtheta}{\isacharbrackleft}{\kern0pt}{\isacharunderscore}{\kern0pt}{\isacharbrackright}{\kern0pt}{\isacharprime}{\kern0pt}{\isacharparenleft}{\kern0pt}{\isacharunderscore}{\kern0pt}{\isacharprime}{\kern0pt}{\isacharparenright}{\kern0pt}{\isacartoucheclose}{\isacharparenright}{\kern0pt}\isanewline
\ \ \isakeyword{where}\ {\isacartoucheopen}{\isasymtheta}{\isacharbrackleft}{\kern0pt}X{\isacharbrackright}{\kern0pt}{\isacharparenleft}{\kern0pt}p{\isacharparenright}{\kern0pt}\ {\isasymequiv}\ {\isasymtheta}{\isacharquery}{\kern0pt}{\isacharbrackleft}{\kern0pt}Some\ X{\isacharbrackright}{\kern0pt}{\isacharparenleft}{\kern0pt}p{\isacharparenright}{\kern0pt}{\isacartoucheclose}%
\begin{isamarkuptext}%
With four special actions (three plus a family), the following abbreviation will be convenient later on.%
\end{isamarkuptext}\isamarkuptrue%
\isacommand{abbreviation}\isamarkupfalse%
\ no{\isacharunderscore}{\kern0pt}special{\isacharunderscore}{\kern0pt}action\ {\isacharcolon}{\kern0pt}{\isacharcolon}{\kern0pt}\ {\isacartoucheopen}{\isacharprime}{\kern0pt}a\ {\isasymRightarrow}\ bool{\isacartoucheclose}\isanewline
\ \ \isakeyword{where}\ {\isacartoucheopen}no{\isacharunderscore}{\kern0pt}special{\isacharunderscore}{\kern0pt}action\ {\isasymalpha}\ \isanewline
\ \ \ \ {\isasymequiv}\ {\isasymalpha}\ {\isasymnoteq}\ {\isasymtau}\ {\isasymand}\ {\isasymalpha}\ {\isasymnoteq}\ t\ {\isasymand}\ {\isasymalpha}\ {\isasymnoteq}\ t{\isacharunderscore}{\kern0pt}{\isasymepsilon}\ {\isasymand}\ {\isacharparenleft}{\kern0pt}{\isasymforall}\ X{\isachardot}{\kern0pt}\ {\isasymalpha}\ {\isasymnoteq}\ {\isasymepsilon}{\isacharbrackleft}{\kern0pt}X{\isacharbrackright}{\kern0pt}{\isacharparenright}{\kern0pt}{\isacartoucheclose}\isanewline
\isanewline
%
\isadelimunimportant
%
\endisadelimunimportant
%
\isatagunimportant
%
\endisatagunimportant
{\isafoldunimportant}%
%
\isadelimunimportant
%
\endisadelimunimportant
%
\isadelimdocument
%
\endisadelimdocument
%
\isatagdocument
%
\isamarkupsubsubsection{Formalising \boldmath{$\rightarrow_\vartheta$}%
}
\isamarkuptrue%
%
\endisatagdocument
{\isafolddocument}%
%
\isadelimdocument
%
\endisadelimdocument
%
\begin{isamarkuptext}%
We now formalise the transition relation of our mapping, given above by the structural operational rules. We use the notation \isa{{\isacharunderscore}{\kern0pt}\ {\isasymlongmapsto}\isactrlsup {\isasymtheta}{\isacharunderscore}{\kern0pt}\ {\isacharunderscore}{\kern0pt}} instead of the more obvious \isa{{\isacharunderscore}{\kern0pt}\ {\isasymlongmapsto}\isactrlsub {\isasymtheta}{\isacharunderscore}{\kern0pt}\ {\isacharunderscore}{\kern0pt}} simply because of better readability.

It should be easy to see that the clauses below correspond directly to the rules above. Like in previous sections, we have to take extra care to handle the requirement \isa{X\ {\isasymsubseteq}\ visible{\isacharunderscore}{\kern0pt}actions}.

We use the \isa{inductive} command, because this allows us to define separate individual (as opposed to the \isa{definition} command). Technically speaking, this inductive definition only has base cases, though, since no premise involves \isa{{\isasymlongmapsto}\isactrlsup {\isasymtheta}}.%
\end{isamarkuptext}\isamarkuptrue%
\isacommand{inductive}\isamarkupfalse%
\ tran{\isacharunderscore}{\kern0pt}theta\ {\isacharcolon}{\kern0pt}{\isacharcolon}{\kern0pt}\ {\isacartoucheopen}{\isacharprime}{\kern0pt}ss\ {\isasymRightarrow}\ {\isacharprime}{\kern0pt}a\ {\isasymRightarrow}\ {\isacharprime}{\kern0pt}ss\ {\isasymRightarrow}\ bool{\isacartoucheclose}\ \isanewline
\ \ {\isacharparenleft}{\kern0pt}{\isacartoucheopen}{\isacharunderscore}{\kern0pt}\ {\isasymlongmapsto}\isactrlsup {\isasymtheta}{\isacharunderscore}{\kern0pt}\ {\isacharunderscore}{\kern0pt}{\isacartoucheclose}\ {\isacharbrackleft}{\kern0pt}{\isadigit{7}}{\isadigit{0}}{\isacharcomma}{\kern0pt}\ {\isadigit{7}}{\isadigit{0}}{\isacharcomma}{\kern0pt}\ {\isadigit{7}}{\isadigit{0}}{\isacharbrackright}{\kern0pt}\ {\isadigit{7}}{\isadigit{0}}{\isacharparenright}{\kern0pt}\isanewline
\ \ \isakeyword{where}\ \isanewline
\ \ \ \ env{\isacharunderscore}{\kern0pt}stabilise{\isacharcolon}{\kern0pt}\ \ {\isacartoucheopen}X\ {\isasymsubseteq}\ visible{\isacharunderscore}{\kern0pt}actions\ {\isasymLongrightarrow}\ \isanewline
\ \ \ \ \ \ {\isasymtheta}{\isacharparenleft}{\kern0pt}p{\isacharparenright}{\kern0pt}\ {\isasymlongmapsto}\isactrlsup {\isasymtheta}{\isasymepsilon}{\isacharbrackleft}{\kern0pt}X{\isacharbrackright}{\kern0pt}\ {\isasymtheta}{\isacharbrackleft}{\kern0pt}X{\isacharbrackright}{\kern0pt}{\isacharparenleft}{\kern0pt}p{\isacharparenright}{\kern0pt}{\isacartoucheclose}\isanewline
\ \ {\isacharbar}{\kern0pt}\ triggered{\isacharunderscore}{\kern0pt}tau{\isacharcolon}{\kern0pt}\ \ \isanewline
\ \ \ \ \ \ {\isacartoucheopen}p\ {\isasymlongmapsto}{\isasymtau}\ q\ {\isasymLongrightarrow}\ {\isasymtheta}{\isacharparenleft}{\kern0pt}p{\isacharparenright}{\kern0pt}\ {\isasymlongmapsto}\isactrlsup {\isasymtheta}{\isasymtau}\ {\isasymtheta}{\isacharparenleft}{\kern0pt}q{\isacharparenright}{\kern0pt}{\isacartoucheclose}\isanewline
\ \ {\isacharbar}{\kern0pt}\ env{\isacharunderscore}{\kern0pt}timeout{\isacharcolon}{\kern0pt}\ \ \ \ {\isacartoucheopen}X\ {\isasymsubseteq}\ visible{\isacharunderscore}{\kern0pt}actions\ {\isasymLongrightarrow}\ \isanewline
\ \ \ \ \ \ idle\ p\ X\ {\isasymLongrightarrow}\ {\isasymtheta}{\isacharbrackleft}{\kern0pt}X{\isacharbrackright}{\kern0pt}{\isacharparenleft}{\kern0pt}p{\isacharparenright}{\kern0pt}\ {\isasymlongmapsto}\isactrlsup {\isasymtheta}t{\isacharunderscore}{\kern0pt}{\isasymepsilon}\ {\isasymtheta}{\isacharparenleft}{\kern0pt}p{\isacharparenright}{\kern0pt}{\isacartoucheclose}\isanewline
\ \ {\isacharbar}{\kern0pt}\ tran{\isacharunderscore}{\kern0pt}visible{\isacharcolon}{\kern0pt}\ \ {\isacartoucheopen}X\ {\isasymsubseteq}\ visible{\isacharunderscore}{\kern0pt}actions\ {\isasymLongrightarrow}\ \isanewline
\ \ \ \ \ \ a\ {\isasymin}\ X\ {\isasymLongrightarrow}\ p\ {\isasymlongmapsto}a\ q\ {\isasymLongrightarrow}\ {\isasymtheta}{\isacharbrackleft}{\kern0pt}X{\isacharbrackright}{\kern0pt}{\isacharparenleft}{\kern0pt}p{\isacharparenright}{\kern0pt}\ {\isasymlongmapsto}\isactrlsup {\isasymtheta}a\ {\isasymtheta}{\isacharparenleft}{\kern0pt}q{\isacharparenright}{\kern0pt}{\isacartoucheclose}\isanewline
\ \ {\isacharbar}{\kern0pt}\ stable{\isacharunderscore}{\kern0pt}tau{\isacharcolon}{\kern0pt}\ \ \ \ \ \ {\isacartoucheopen}X\ {\isasymsubseteq}\ visible{\isacharunderscore}{\kern0pt}actions\ {\isasymLongrightarrow}\ \isanewline
\ \ \ \ \ \ p\ {\isasymlongmapsto}{\isasymtau}\ q\ {\isasymLongrightarrow}\ {\isasymtheta}{\isacharbrackleft}{\kern0pt}X{\isacharbrackright}{\kern0pt}{\isacharparenleft}{\kern0pt}p{\isacharparenright}{\kern0pt}\ {\isasymlongmapsto}\isactrlsup {\isasymtheta}{\isasymtau}\ {\isasymtheta}{\isacharbrackleft}{\kern0pt}X{\isacharbrackright}{\kern0pt}{\isacharparenleft}{\kern0pt}q{\isacharparenright}{\kern0pt}{\isacartoucheclose}\isanewline
\ \ {\isacharbar}{\kern0pt}\ sys{\isacharunderscore}{\kern0pt}timeout{\isacharcolon}{\kern0pt}\ \ {\isacartoucheopen}X\ {\isasymsubseteq}\ visible{\isacharunderscore}{\kern0pt}actions\ {\isasymLongrightarrow}\ \isanewline
\ \ \ \ \ \ idle\ p\ X\ {\isasymLongrightarrow}\ p\ {\isasymlongmapsto}t\ q\ {\isasymLongrightarrow}\ {\isasymtheta}{\isacharbrackleft}{\kern0pt}X{\isacharbrackright}{\kern0pt}{\isacharparenleft}{\kern0pt}p{\isacharparenright}{\kern0pt}\ {\isasymlongmapsto}\isactrlsup {\isasymtheta}t\ {\isasymtheta}{\isacharbrackleft}{\kern0pt}X{\isacharbrackright}{\kern0pt}{\isacharparenleft}{\kern0pt}q{\isacharparenright}{\kern0pt}{\isacartoucheclose}%
\isadelimdocument
%
\endisadelimdocument
%
\isatagdocument
%
\isamarkupsubsubsection{Generation Lemmas%
}
\isamarkuptrue%
%
\endisatagdocument
{\isafolddocument}%
%
\isadelimdocument
%
\endisadelimdocument
%
\begin{isamarkuptext}%
For the remainder of this section, we will derive a set of generation lemmas, i.e.\@ lemmas that allow us to reason backwards: if we know \isa{P\ {\isasymlongmapsto}\isactrlsup {\isasymtheta}{\isasymalpha}\ P{\isacharprime}{\kern0pt}} and some other information about \isa{P} and/or \isa{{\isasymalpha}}, we can deduce some information about the other variables as well as the transitions of the original LTS.%
\end{isamarkuptext}\isamarkuptrue%
\isacommand{lemma}\isamarkupfalse%
\ generation{\isacharunderscore}{\kern0pt}triggered{\isacharunderscore}{\kern0pt}transitions{\isacharcolon}{\kern0pt}\isanewline
\ \ \isakeyword{assumes}\ {\isacartoucheopen}{\isasymtheta}{\isacharparenleft}{\kern0pt}p{\isacharparenright}{\kern0pt}\ {\isasymlongmapsto}\isactrlsup {\isasymtheta}{\isasymalpha}\ P{\isacharprime}{\kern0pt}{\isacartoucheclose}\isanewline
\ \ \isakeyword{shows}\ {\isacartoucheopen}{\isacharparenleft}{\kern0pt}{\isasymexists}X{\isachardot}{\kern0pt}\ {\isasymalpha}\ {\isacharequal}{\kern0pt}\ {\isasymepsilon}{\isacharbrackleft}{\kern0pt}X{\isacharbrackright}{\kern0pt}\ {\isasymand}\ P{\isacharprime}{\kern0pt}\ {\isacharequal}{\kern0pt}\ {\isasymtheta}{\isacharbrackleft}{\kern0pt}X{\isacharbrackright}{\kern0pt}{\isacharparenleft}{\kern0pt}p{\isacharparenright}{\kern0pt}\ {\isasymand}\ X\ {\isasymsubseteq}\ visible{\isacharunderscore}{\kern0pt}actions{\isacharparenright}{\kern0pt}\ \isanewline
\ \ \ \ {\isasymor}\ {\isacharparenleft}{\kern0pt}{\isasymalpha}\ {\isacharequal}{\kern0pt}\ {\isasymtau}\ {\isasymand}\ {\isacharparenleft}{\kern0pt}{\isasymexists}p{\isacharprime}{\kern0pt}{\isachardot}{\kern0pt}\ p\ {\isasymlongmapsto}{\isasymtau}\ p{\isacharprime}{\kern0pt}{\isacharparenright}{\kern0pt}{\isacharparenright}{\kern0pt}{\isacartoucheclose}\isanewline
%
\isadelimproof
\ \ %
\endisadelimproof
%
\isatagproof
\isacommand{using}\isamarkupfalse%
\ iffD{\isadigit{1}}{\isacharbrackleft}{\kern0pt}OF\ tran{\isacharunderscore}{\kern0pt}theta{\isachardot}{\kern0pt}simps\ assms{\isacharbrackright}{\kern0pt}\isanewline
\ \ \isacommand{by}\isamarkupfalse%
\ {\isacharparenleft}{\kern0pt}smt\ {\isacharparenleft}{\kern0pt}z{\isadigit{3}}{\isacharparenright}{\kern0pt}\ injectivity{\isacharunderscore}{\kern0pt}theta{\isacharparenleft}{\kern0pt}{\isadigit{1}}{\isacharcomma}{\kern0pt}{\isadigit{2}}{\isacharparenright}{\kern0pt}{\isacharparenright}{\kern0pt}%
\endisatagproof
{\isafoldproof}%
%
\isadelimproof
\isanewline
%
\endisadelimproof
\isanewline
\isacommand{lemma}\isamarkupfalse%
\ generation{\isacharunderscore}{\kern0pt}stable{\isacharunderscore}{\kern0pt}transitions{\isacharcolon}{\kern0pt}\isanewline
\ \ \isakeyword{assumes}\ {\isacartoucheopen}{\isasymtheta}{\isacharbrackleft}{\kern0pt}X{\isacharbrackright}{\kern0pt}{\isacharparenleft}{\kern0pt}p{\isacharparenright}{\kern0pt}\ {\isasymlongmapsto}\isactrlsup {\isasymtheta}{\isasymalpha}\ P{\isacharprime}{\kern0pt}{\isacartoucheclose}\isanewline
\ \ \isakeyword{shows}\ {\isacartoucheopen}{\isasymalpha}\ {\isacharequal}{\kern0pt}\ t{\isacharunderscore}{\kern0pt}{\isasymepsilon}\ {\isasymor}\ {\isacharparenleft}{\kern0pt}{\isasymexists}\ p{\isacharprime}{\kern0pt}{\isachardot}{\kern0pt}\ p\ {\isasymlongmapsto}{\isasymalpha}\ p{\isacharprime}{\kern0pt}\ {\isasymand}\ {\isacharparenleft}{\kern0pt}{\isasymalpha}\ {\isasymin}\ X\ {\isasymor}\ {\isasymalpha}\ {\isacharequal}{\kern0pt}\ {\isasymtau}\ {\isasymor}\ {\isasymalpha}\ {\isacharequal}{\kern0pt}\ t{\isacharparenright}{\kern0pt}{\isacharparenright}{\kern0pt}{\isacartoucheclose}\isanewline
%
\isadelimproof
\ \ %
\endisadelimproof
%
\isatagproof
\isacommand{using}\isamarkupfalse%
\ iffD{\isadigit{1}}{\isacharbrackleft}{\kern0pt}OF\ tran{\isacharunderscore}{\kern0pt}theta{\isachardot}{\kern0pt}simps\ assms{\isacharbrackright}{\kern0pt}\isanewline
\ \ \isacommand{by}\isamarkupfalse%
\ {\isacharparenleft}{\kern0pt}smt\ injectivity{\isacharunderscore}{\kern0pt}theta{\isacharparenleft}{\kern0pt}{\isadigit{1}}{\isacharcomma}{\kern0pt}{\isadigit{3}}{\isacharparenright}{\kern0pt}\ lts{\isacharunderscore}{\kern0pt}timeout{\isacharunderscore}{\kern0pt}mappable{\isacharunderscore}{\kern0pt}axioms{\isacharparenright}{\kern0pt}%
\endisatagproof
{\isafoldproof}%
%
\isadelimproof
\isanewline
%
\endisadelimproof
\ \ \isanewline
\isacommand{lemma}\isamarkupfalse%
\ generation{\isacharunderscore}{\kern0pt}env{\isacharunderscore}{\kern0pt}stabilise{\isacharcolon}{\kern0pt}\isanewline
\ \ \isakeyword{assumes}\ {\isacartoucheopen}P\ {\isasymlongmapsto}\isactrlsup {\isasymtheta}{\isasymepsilon}{\isacharbrackleft}{\kern0pt}X{\isacharbrackright}{\kern0pt}\ P{\isacharprime}{\kern0pt}{\isacartoucheclose}\isanewline
\ \ \isakeyword{shows}\ {\isacartoucheopen}{\isasymexists}\ p{\isachardot}{\kern0pt}\ P\ {\isacharequal}{\kern0pt}\ {\isasymtheta}{\isacharparenleft}{\kern0pt}p{\isacharparenright}{\kern0pt}\ {\isasymand}\ P{\isacharprime}{\kern0pt}\ {\isacharequal}{\kern0pt}\ {\isasymtheta}{\isacharbrackleft}{\kern0pt}X{\isacharbrackright}{\kern0pt}{\isacharparenleft}{\kern0pt}p{\isacharparenright}{\kern0pt}{\isacartoucheclose}\ \isanewline
%
\isadelimproof
\ \ %
\endisadelimproof
%
\isatagproof
\isacommand{using}\isamarkupfalse%
\ iffD{\isadigit{1}}{\isacharbrackleft}{\kern0pt}OF\ tran{\isacharunderscore}{\kern0pt}theta{\isachardot}{\kern0pt}simps\ assms{\isacharparenleft}{\kern0pt}{\isadigit{1}}{\isacharparenright}{\kern0pt}{\isacharbrackright}{\kern0pt}\ \isanewline
\ \ \isacommand{by}\isamarkupfalse%
\ {\isacharparenleft}{\kern0pt}smt\ {\isacharparenleft}{\kern0pt}z{\isadigit{3}}{\isacharparenright}{\kern0pt}\ distinctness{\isacharunderscore}{\kern0pt}special{\isacharunderscore}{\kern0pt}actions{\isacharparenleft}{\kern0pt}{\isadigit{6}}{\isacharparenright}{\kern0pt}\ distinctness{\isacharunderscore}{\kern0pt}special{\isacharunderscore}{\kern0pt}actions{\isacharparenleft}{\kern0pt}{\isadigit{7}}{\isacharparenright}{\kern0pt}\ no{\isacharunderscore}{\kern0pt}epsilon{\isacharunderscore}{\kern0pt}in{\isacharunderscore}{\kern0pt}tran{\isacharparenleft}{\kern0pt}{\isadigit{1}}{\isacharparenright}{\kern0pt}{\isacharparenright}{\kern0pt}%
\endisatagproof
{\isafoldproof}%
%
\isadelimproof
\isanewline
%
\endisadelimproof
\isanewline
\isacommand{lemma}\isamarkupfalse%
\ generation{\isacharunderscore}{\kern0pt}triggered{\isacharunderscore}{\kern0pt}tau{\isacharcolon}{\kern0pt}\isanewline
\ \ \isakeyword{assumes}\ {\isacartoucheopen}{\isasymtheta}{\isacharparenleft}{\kern0pt}p{\isacharparenright}{\kern0pt}\ {\isasymlongmapsto}\isactrlsup {\isasymtheta}{\isasymtau}\ P{\isacharprime}{\kern0pt}{\isacartoucheclose}\isanewline
\ \ \isakeyword{shows}\ {\isacartoucheopen}{\isasymexists}\ p{\isacharprime}{\kern0pt}{\isachardot}{\kern0pt}\ P{\isacharprime}{\kern0pt}\ {\isacharequal}{\kern0pt}\ {\isasymtheta}{\isacharparenleft}{\kern0pt}p{\isacharprime}{\kern0pt}{\isacharparenright}{\kern0pt}\ {\isasymand}\ p\ {\isasymlongmapsto}{\isasymtau}\ p{\isacharprime}{\kern0pt}{\isacartoucheclose}\isanewline
%
\isadelimproof
\ \ %
\endisadelimproof
%
\isatagproof
\isacommand{using}\isamarkupfalse%
\ iffD{\isadigit{1}}{\isacharbrackleft}{\kern0pt}OF\ tran{\isacharunderscore}{\kern0pt}theta{\isachardot}{\kern0pt}simps\ assms{\isacharbrackright}{\kern0pt}\isanewline
\ \ \isacommand{using}\isamarkupfalse%
\ distinctness{\isacharunderscore}{\kern0pt}special{\isacharunderscore}{\kern0pt}actions{\isacharparenleft}{\kern0pt}{\isadigit{4}}{\isacharparenright}{\kern0pt}\ injectivity{\isacharunderscore}{\kern0pt}theta{\isacharparenleft}{\kern0pt}{\isadigit{1}}{\isacharparenright}{\kern0pt}\ injectivity{\isacharunderscore}{\kern0pt}theta{\isacharparenleft}{\kern0pt}{\isadigit{2}}{\isacharparenright}{\kern0pt}\ \isacommand{by}\isamarkupfalse%
\ blast%
\endisatagproof
{\isafoldproof}%
%
\isadelimproof
\isanewline
%
\endisadelimproof
\ \ \isanewline
\isacommand{lemma}\isamarkupfalse%
\ generation{\isacharunderscore}{\kern0pt}env{\isacharunderscore}{\kern0pt}timeout{\isacharcolon}{\kern0pt}\isanewline
\ \ \isakeyword{assumes}\ {\isacartoucheopen}{\isasymtheta}{\isacharbrackleft}{\kern0pt}X{\isacharbrackright}{\kern0pt}{\isacharparenleft}{\kern0pt}p{\isacharparenright}{\kern0pt}\ {\isasymlongmapsto}\isactrlsup {\isasymtheta}t{\isacharunderscore}{\kern0pt}{\isasymepsilon}\ P{\isacharprime}{\kern0pt}{\isacartoucheclose}\isanewline
\ \ \isakeyword{shows}\ {\isacartoucheopen}P{\isacharprime}{\kern0pt}\ {\isacharequal}{\kern0pt}\ {\isasymtheta}{\isacharparenleft}{\kern0pt}p{\isacharparenright}{\kern0pt}\ {\isasymand}\ idle\ p\ X{\isacartoucheclose}\isanewline
%
\isadelimproof
\ \ %
\endisadelimproof
%
\isatagproof
\isacommand{using}\isamarkupfalse%
\ iffD{\isadigit{1}}{\isacharbrackleft}{\kern0pt}OF\ tran{\isacharunderscore}{\kern0pt}theta{\isachardot}{\kern0pt}simps\ assms{\isacharbrackright}{\kern0pt}\ distinctness{\isacharunderscore}{\kern0pt}special{\isacharunderscore}{\kern0pt}actions\isanewline
\ \ \isacommand{by}\isamarkupfalse%
\ {\isacharparenleft}{\kern0pt}smt\ injectivity{\isacharunderscore}{\kern0pt}theta{\isacharparenleft}{\kern0pt}{\isadigit{3}}{\isacharparenright}{\kern0pt}\ insertCI\ no{\isacharunderscore}{\kern0pt}epsilon{\isacharunderscore}{\kern0pt}in{\isacharunderscore}{\kern0pt}tran{\isacharparenleft}{\kern0pt}{\isadigit{2}}{\isacharparenright}{\kern0pt}{\isacharparenright}{\kern0pt}{\isacharplus}{\kern0pt}%
\endisatagproof
{\isafoldproof}%
%
\isadelimproof
\isanewline
%
\endisadelimproof
\isanewline
\isacommand{lemma}\isamarkupfalse%
\ generation{\isacharunderscore}{\kern0pt}tran{\isacharunderscore}{\kern0pt}visible{\isacharcolon}{\kern0pt}\isanewline
\ \ \isakeyword{assumes}\ {\isacartoucheopen}{\isasymtheta}{\isacharbrackleft}{\kern0pt}X{\isacharbrackright}{\kern0pt}{\isacharparenleft}{\kern0pt}p{\isacharparenright}{\kern0pt}\ {\isasymlongmapsto}\isactrlsup {\isasymtheta}a\ P{\isacharprime}{\kern0pt}{\isacartoucheclose}\ {\isacartoucheopen}a\ {\isasymin}\ visible{\isacharunderscore}{\kern0pt}actions{\isacartoucheclose}\isanewline
\ \ \isakeyword{shows}\ {\isacartoucheopen}a\ {\isasymin}\ X\ {\isasymand}\ {\isacharparenleft}{\kern0pt}{\isasymexists}\ p{\isacharprime}{\kern0pt}{\isachardot}{\kern0pt}\ P{\isacharprime}{\kern0pt}\ {\isacharequal}{\kern0pt}\ {\isasymtheta}{\isacharparenleft}{\kern0pt}p{\isacharprime}{\kern0pt}{\isacharparenright}{\kern0pt}\ {\isasymand}\ p\ {\isasymlongmapsto}a\ p{\isacharprime}{\kern0pt}{\isacharparenright}{\kern0pt}{\isacartoucheclose}\isanewline
%
\isadelimproof
\ \ %
\endisadelimproof
%
\isatagproof
\isacommand{using}\isamarkupfalse%
\ iffD{\isadigit{1}}{\isacharbrackleft}{\kern0pt}OF\ tran{\isacharunderscore}{\kern0pt}theta{\isachardot}{\kern0pt}simps\ assms{\isacharparenleft}{\kern0pt}{\isadigit{1}}{\isacharparenright}{\kern0pt}{\isacharbrackright}{\kern0pt}\isanewline
\isacommand{proof}\isamarkupfalse%
\ {\isacharparenleft}{\kern0pt}elim\ disjE{\isacharcomma}{\kern0pt}\ goal{\isacharunderscore}{\kern0pt}cases{\isacharparenright}{\kern0pt}\isanewline
\ \ \isacommand{case}\isamarkupfalse%
\ {\isadigit{1}}\isanewline
\ \ \isacommand{then}\isamarkupfalse%
\ \isacommand{show}\isamarkupfalse%
\ {\isacharquery}{\kern0pt}case\ \isacommand{by}\isamarkupfalse%
\ {\isacharparenleft}{\kern0pt}metis\ injectivity{\isacharunderscore}{\kern0pt}theta{\isacharparenleft}{\kern0pt}{\isadigit{1}}{\isacharparenright}{\kern0pt}{\isacharparenright}{\kern0pt}\isanewline
\isacommand{next}\isamarkupfalse%
\isanewline
\ \ \isacommand{case}\isamarkupfalse%
\ {\isadigit{2}}\isanewline
\ \ \isacommand{then}\isamarkupfalse%
\ \isacommand{obtain}\isamarkupfalse%
\ p{\isacharprime}{\kern0pt}\ \isakeyword{where}\ {\isacartoucheopen}{\isasymtheta}{\isacharbrackleft}{\kern0pt}X{\isacharbrackright}{\kern0pt}{\isacharparenleft}{\kern0pt}p{\isacharparenright}{\kern0pt}\ {\isacharequal}{\kern0pt}\ {\isasymtheta}{\isacharparenleft}{\kern0pt}p{\isacharprime}{\kern0pt}{\isacharparenright}{\kern0pt}{\isacartoucheclose}\ \isacommand{by}\isamarkupfalse%
\ blast\isanewline
\ \ \isacommand{hence}\isamarkupfalse%
\ False\ \isacommand{using}\isamarkupfalse%
\ injectivity{\isacharunderscore}{\kern0pt}theta{\isacharparenleft}{\kern0pt}{\isadigit{1}}{\isacharparenright}{\kern0pt}\ \isacommand{by}\isamarkupfalse%
\ metis\isanewline
\ \ \isacommand{thus}\isamarkupfalse%
\ {\isacharquery}{\kern0pt}case\ \isacommand{by}\isamarkupfalse%
\ simp\isanewline
\isacommand{next}\isamarkupfalse%
\isanewline
\ \ \isacommand{case}\isamarkupfalse%
\ {\isadigit{3}}\isanewline
\ \ \isacommand{then}\isamarkupfalse%
\ \isacommand{show}\isamarkupfalse%
\ {\isacharquery}{\kern0pt}case\ \isacommand{using}\isamarkupfalse%
\ assms{\isacharparenleft}{\kern0pt}{\isadigit{2}}{\isacharparenright}{\kern0pt}\ no{\isacharunderscore}{\kern0pt}epsilon{\isacharunderscore}{\kern0pt}in{\isacharunderscore}{\kern0pt}tran{\isacharparenleft}{\kern0pt}{\isadigit{2}}{\isacharparenright}{\kern0pt}\ visible{\isacharunderscore}{\kern0pt}actions{\isacharunderscore}{\kern0pt}def\ \isacommand{by}\isamarkupfalse%
\ auto\isanewline
\isacommand{next}\isamarkupfalse%
\isanewline
\ \ \isacommand{case}\isamarkupfalse%
\ {\isadigit{4}}\isanewline
\ \ \isacommand{hence}\isamarkupfalse%
\ {\isacartoucheopen}X\ {\isasymsubseteq}\ visible{\isacharunderscore}{\kern0pt}actions{\isacartoucheclose}\ \isacommand{by}\isamarkupfalse%
\ {\isacharparenleft}{\kern0pt}metis\ injectivity{\isacharunderscore}{\kern0pt}theta{\isacharparenleft}{\kern0pt}{\isadigit{3}}{\isacharparenright}{\kern0pt}{\isacharparenright}{\kern0pt}\isanewline
\ \ \isacommand{with}\isamarkupfalse%
\ {\isadigit{4}}\ \isacommand{show}\isamarkupfalse%
\ {\isacharquery}{\kern0pt}case\ \isacommand{using}\isamarkupfalse%
\ injectivity{\isacharunderscore}{\kern0pt}theta{\isacharparenleft}{\kern0pt}{\isadigit{3}}{\isacharparenright}{\kern0pt}\ \isacommand{by}\isamarkupfalse%
\ blast\isanewline
\isacommand{next}\isamarkupfalse%
\isanewline
\ \ \isacommand{case}\isamarkupfalse%
\ {\isadigit{5}}\isanewline
\ \ \isacommand{hence}\isamarkupfalse%
\ False\ \isacommand{using}\isamarkupfalse%
\ visible{\isacharunderscore}{\kern0pt}actions{\isacharunderscore}{\kern0pt}def\ assms{\isacharparenleft}{\kern0pt}{\isadigit{2}}{\isacharparenright}{\kern0pt}\ \isacommand{by}\isamarkupfalse%
\ simp\isanewline
\ \ \isacommand{thus}\isamarkupfalse%
\ {\isacharquery}{\kern0pt}case\ \isacommand{by}\isamarkupfalse%
\ simp\isanewline
\isacommand{next}\isamarkupfalse%
\isanewline
\ \ \isacommand{case}\isamarkupfalse%
\ {\isadigit{6}}\isanewline
\ \ \isacommand{hence}\isamarkupfalse%
\ False\ \isacommand{using}\isamarkupfalse%
\ visible{\isacharunderscore}{\kern0pt}actions{\isacharunderscore}{\kern0pt}def\ assms{\isacharparenleft}{\kern0pt}{\isadigit{2}}{\isacharparenright}{\kern0pt}\ \isacommand{by}\isamarkupfalse%
\ simp\isanewline
\ \ \isacommand{thus}\isamarkupfalse%
\ {\isacharquery}{\kern0pt}case\ \isacommand{by}\isamarkupfalse%
\ simp\isanewline
\isacommand{qed}\isamarkupfalse%
%
\endisatagproof
{\isafoldproof}%
%
\isadelimproof
\isanewline
%
\endisadelimproof
\isanewline
\isacommand{lemma}\isamarkupfalse%
\ generation{\isacharunderscore}{\kern0pt}stable{\isacharunderscore}{\kern0pt}tau{\isacharcolon}{\kern0pt}\isanewline
\ \ \isakeyword{assumes}\ {\isacartoucheopen}{\isasymtheta}{\isacharbrackleft}{\kern0pt}X{\isacharbrackright}{\kern0pt}{\isacharparenleft}{\kern0pt}p{\isacharparenright}{\kern0pt}\ {\isasymlongmapsto}\isactrlsup {\isasymtheta}{\isasymtau}\ P{\isacharprime}{\kern0pt}{\isacartoucheclose}\isanewline
\ \ \isakeyword{shows}\ {\isacartoucheopen}{\isasymexists}\ p{\isacharprime}{\kern0pt}{\isachardot}{\kern0pt}\ P{\isacharprime}{\kern0pt}\ {\isacharequal}{\kern0pt}\ {\isasymtheta}{\isacharbrackleft}{\kern0pt}X{\isacharbrackright}{\kern0pt}{\isacharparenleft}{\kern0pt}p{\isacharprime}{\kern0pt}{\isacharparenright}{\kern0pt}\ {\isasymand}\ p\ {\isasymlongmapsto}{\isasymtau}\ p{\isacharprime}{\kern0pt}{\isacartoucheclose}\isanewline
%
\isadelimproof
\ \ %
\endisadelimproof
%
\isatagproof
\isacommand{using}\isamarkupfalse%
\ iffD{\isadigit{1}}{\isacharbrackleft}{\kern0pt}OF\ tran{\isacharunderscore}{\kern0pt}theta{\isachardot}{\kern0pt}simps\ assms{\isacharbrackright}{\kern0pt}\isanewline
\isacommand{proof}\isamarkupfalse%
\ {\isacharparenleft}{\kern0pt}elim\ disjE{\isacharcomma}{\kern0pt}\ goal{\isacharunderscore}{\kern0pt}cases{\isacharparenright}{\kern0pt}\isanewline
\ \ \isacommand{case}\isamarkupfalse%
\ {\isadigit{1}}\isanewline
\ \ \isacommand{hence}\isamarkupfalse%
\ False\ \isacommand{using}\isamarkupfalse%
\ distinctness{\isacharunderscore}{\kern0pt}special{\isacharunderscore}{\kern0pt}actions\ \isacommand{by}\isamarkupfalse%
\ blast\isanewline
\ \ \isacommand{thus}\isamarkupfalse%
\ {\isacharquery}{\kern0pt}case\ \isacommand{by}\isamarkupfalse%
\ simp\isanewline
\isacommand{next}\isamarkupfalse%
\isanewline
\ \ \isacommand{case}\isamarkupfalse%
\ {\isadigit{2}}\isanewline
\ \ \isacommand{then}\isamarkupfalse%
\ \isacommand{obtain}\isamarkupfalse%
\ p{\isacharprime}{\kern0pt}\ \isakeyword{where}\ {\isacartoucheopen}{\isasymtheta}{\isacharbrackleft}{\kern0pt}X{\isacharbrackright}{\kern0pt}{\isacharparenleft}{\kern0pt}p{\isacharparenright}{\kern0pt}\ {\isacharequal}{\kern0pt}\ {\isasymtheta}{\isacharparenleft}{\kern0pt}p{\isacharprime}{\kern0pt}{\isacharparenright}{\kern0pt}{\isacartoucheclose}\ \isacommand{by}\isamarkupfalse%
\ blast\isanewline
\ \ \isacommand{hence}\isamarkupfalse%
\ False\ \isacommand{using}\isamarkupfalse%
\ injectivity{\isacharunderscore}{\kern0pt}theta{\isacharparenleft}{\kern0pt}{\isadigit{1}}{\isacharparenright}{\kern0pt}\ \isacommand{by}\isamarkupfalse%
\ metis\isanewline
\ \ \isacommand{thus}\isamarkupfalse%
\ {\isacharquery}{\kern0pt}case\ \isacommand{by}\isamarkupfalse%
\ simp\isanewline
\isacommand{next}\isamarkupfalse%
\isanewline
\ \ \isacommand{case}\isamarkupfalse%
\ {\isadigit{3}}\isanewline
\ \ \isacommand{hence}\isamarkupfalse%
\ False\ \isacommand{using}\isamarkupfalse%
\ distinctness{\isacharunderscore}{\kern0pt}special{\isacharunderscore}{\kern0pt}actions\ \isacommand{by}\isamarkupfalse%
\ blast\isanewline
\ \ \isacommand{thus}\isamarkupfalse%
\ {\isacharquery}{\kern0pt}case\ \isacommand{by}\isamarkupfalse%
\ simp\isanewline
\isacommand{next}\isamarkupfalse%
\isanewline
\ \ \isacommand{case}\isamarkupfalse%
\ {\isadigit{4}}\isanewline
\ \ \isacommand{then}\isamarkupfalse%
\ \isacommand{show}\isamarkupfalse%
\ {\isacharquery}{\kern0pt}case\ \isacommand{using}\isamarkupfalse%
\ visible{\isacharunderscore}{\kern0pt}actions{\isacharunderscore}{\kern0pt}def\ \isacommand{by}\isamarkupfalse%
\ fastforce\isanewline
\isacommand{next}\isamarkupfalse%
\isanewline
\ \ \isacommand{case}\isamarkupfalse%
\ {\isadigit{5}}\isanewline
\ \ \isacommand{hence}\isamarkupfalse%
\ {\isacartoucheopen}X\ {\isasymsubseteq}\ visible{\isacharunderscore}{\kern0pt}actions{\isacartoucheclose}\ \isacommand{by}\isamarkupfalse%
\ {\isacharparenleft}{\kern0pt}metis\ injectivity{\isacharunderscore}{\kern0pt}theta{\isacharparenleft}{\kern0pt}{\isadigit{3}}{\isacharparenright}{\kern0pt}{\isacharparenright}{\kern0pt}\isanewline
\ \ \isacommand{with}\isamarkupfalse%
\ {\isadigit{5}}\ \isacommand{show}\isamarkupfalse%
\ {\isacharquery}{\kern0pt}case\ \isacommand{using}\isamarkupfalse%
\ injectivity{\isacharunderscore}{\kern0pt}theta{\isacharparenleft}{\kern0pt}{\isadigit{3}}{\isacharparenright}{\kern0pt}\ \isacommand{by}\isamarkupfalse%
\ blast\isanewline
\isacommand{next}\isamarkupfalse%
\isanewline
\ \ \isacommand{case}\isamarkupfalse%
\ {\isadigit{6}}\isanewline
\ \ \isacommand{hence}\isamarkupfalse%
\ False\ \isacommand{using}\isamarkupfalse%
\ lts{\isacharunderscore}{\kern0pt}timeout{\isacharunderscore}{\kern0pt}axioms\ lts{\isacharunderscore}{\kern0pt}timeout{\isacharunderscore}{\kern0pt}def\ \isacommand{by}\isamarkupfalse%
\ force\isanewline
\ \ \isacommand{thus}\isamarkupfalse%
\ {\isacharquery}{\kern0pt}case\ \isacommand{by}\isamarkupfalse%
\ simp\isanewline
\isacommand{qed}\isamarkupfalse%
%
\endisatagproof
{\isafoldproof}%
%
\isadelimproof
\isanewline
%
\endisadelimproof
\isanewline
\isacommand{lemma}\isamarkupfalse%
\ generation{\isacharunderscore}{\kern0pt}sys{\isacharunderscore}{\kern0pt}timeout{\isacharcolon}{\kern0pt}\isanewline
\ \ \isakeyword{assumes}\ {\isacartoucheopen}{\isasymtheta}{\isacharbrackleft}{\kern0pt}X{\isacharbrackright}{\kern0pt}{\isacharparenleft}{\kern0pt}p{\isacharparenright}{\kern0pt}\ {\isasymlongmapsto}\isactrlsup {\isasymtheta}t\ P{\isacharprime}{\kern0pt}{\isacartoucheclose}\isanewline
\ \ \isakeyword{shows}\ {\isacartoucheopen}{\isasymexists}\ p{\isacharprime}{\kern0pt}{\isachardot}{\kern0pt}\ P{\isacharprime}{\kern0pt}\ {\isacharequal}{\kern0pt}\ {\isasymtheta}{\isacharbrackleft}{\kern0pt}X{\isacharbrackright}{\kern0pt}{\isacharparenleft}{\kern0pt}p{\isacharprime}{\kern0pt}{\isacharparenright}{\kern0pt}\ {\isasymand}\ idle\ p\ X\ {\isasymand}\ p\ {\isasymlongmapsto}t\ p{\isacharprime}{\kern0pt}{\isacartoucheclose}\isanewline
%
\isadelimproof
\ \ %
\endisadelimproof
%
\isatagproof
\isacommand{using}\isamarkupfalse%
\ iffD{\isadigit{1}}{\isacharbrackleft}{\kern0pt}OF\ tran{\isacharunderscore}{\kern0pt}theta{\isachardot}{\kern0pt}simps\ assms{\isacharbrackright}{\kern0pt}\isanewline
\isacommand{proof}\isamarkupfalse%
\ {\isacharparenleft}{\kern0pt}elim\ disjE{\isacharcomma}{\kern0pt}\ goal{\isacharunderscore}{\kern0pt}cases{\isacharparenright}{\kern0pt}\isanewline
\ \ \isacommand{case}\isamarkupfalse%
\ {\isadigit{1}}\isanewline
\ \ \isacommand{hence}\isamarkupfalse%
\ False\ \isacommand{using}\isamarkupfalse%
\ distinctness{\isacharunderscore}{\kern0pt}special{\isacharunderscore}{\kern0pt}actions\ \isacommand{by}\isamarkupfalse%
\ blast\isanewline
\ \ \isacommand{thus}\isamarkupfalse%
\ {\isacharquery}{\kern0pt}case\ \isacommand{{\isachardot}{\kern0pt}{\isachardot}{\kern0pt}}\isamarkupfalse%
\isanewline
\isacommand{next}\isamarkupfalse%
\isanewline
\ \ \isacommand{case}\isamarkupfalse%
\ {\isadigit{2}}\isanewline
\ \ \isacommand{then}\isamarkupfalse%
\ \isacommand{obtain}\isamarkupfalse%
\ p{\isacharprime}{\kern0pt}\ \isakeyword{where}\ {\isacartoucheopen}{\isasymtheta}{\isacharbrackleft}{\kern0pt}X{\isacharbrackright}{\kern0pt}{\isacharparenleft}{\kern0pt}p{\isacharparenright}{\kern0pt}\ {\isacharequal}{\kern0pt}\ {\isasymtheta}{\isacharparenleft}{\kern0pt}p{\isacharprime}{\kern0pt}{\isacharparenright}{\kern0pt}{\isacartoucheclose}\ \isacommand{by}\isamarkupfalse%
\ blast\isanewline
\ \ \isacommand{hence}\isamarkupfalse%
\ False\ \isacommand{using}\isamarkupfalse%
\ injectivity{\isacharunderscore}{\kern0pt}theta{\isacharparenleft}{\kern0pt}{\isadigit{1}}{\isacharparenright}{\kern0pt}\ \isacommand{by}\isamarkupfalse%
\ metis\isanewline
\ \ \isacommand{thus}\isamarkupfalse%
\ {\isacharquery}{\kern0pt}case\ \isacommand{{\isachardot}{\kern0pt}{\isachardot}{\kern0pt}}\isamarkupfalse%
\isanewline
\isacommand{next}\isamarkupfalse%
\isanewline
\ \ \isacommand{case}\isamarkupfalse%
\ {\isadigit{3}}\isanewline
\ \ \isacommand{hence}\isamarkupfalse%
\ False\ \isacommand{using}\isamarkupfalse%
\ distinctness{\isacharunderscore}{\kern0pt}special{\isacharunderscore}{\kern0pt}actions\ \isacommand{by}\isamarkupfalse%
\ blast\isanewline
\ \ \isacommand{thus}\isamarkupfalse%
\ {\isacharquery}{\kern0pt}case\ \isacommand{{\isachardot}{\kern0pt}{\isachardot}{\kern0pt}}\isamarkupfalse%
\isanewline
\isacommand{next}\isamarkupfalse%
\isanewline
\ \ \isacommand{case}\isamarkupfalse%
\ {\isadigit{4}}\isanewline
\ \ \isacommand{then}\isamarkupfalse%
\ \isacommand{show}\isamarkupfalse%
\ {\isacharquery}{\kern0pt}case\ \isacommand{using}\isamarkupfalse%
\ visible{\isacharunderscore}{\kern0pt}actions{\isacharunderscore}{\kern0pt}def\ \isacommand{by}\isamarkupfalse%
\ auto\isanewline
\isacommand{next}\isamarkupfalse%
\isanewline
\ \ \isacommand{case}\isamarkupfalse%
\ {\isadigit{5}}\isanewline
\ \ \isacommand{hence}\isamarkupfalse%
\ False\ \isacommand{using}\isamarkupfalse%
\ lts{\isacharunderscore}{\kern0pt}timeout{\isacharunderscore}{\kern0pt}axioms\ lts{\isacharunderscore}{\kern0pt}timeout{\isacharunderscore}{\kern0pt}def\ \isacommand{by}\isamarkupfalse%
\ force\isanewline
\ \ \isacommand{thus}\isamarkupfalse%
\ {\isacharquery}{\kern0pt}case\ \isacommand{{\isachardot}{\kern0pt}{\isachardot}{\kern0pt}}\isamarkupfalse%
\isanewline
\isacommand{next}\isamarkupfalse%
\isanewline
\ \ \isacommand{case}\isamarkupfalse%
\ {\isadigit{6}}\isanewline
\ \ \isacommand{hence}\isamarkupfalse%
\ {\isacartoucheopen}X\ {\isasymsubseteq}\ visible{\isacharunderscore}{\kern0pt}actions{\isacartoucheclose}\ \isacommand{by}\isamarkupfalse%
\ {\isacharparenleft}{\kern0pt}metis\ injectivity{\isacharunderscore}{\kern0pt}theta{\isacharparenleft}{\kern0pt}{\isadigit{3}}{\isacharparenright}{\kern0pt}{\isacharparenright}{\kern0pt}\isanewline
\ \ \isacommand{with}\isamarkupfalse%
\ {\isadigit{6}}\ \isacommand{show}\isamarkupfalse%
\ {\isacharquery}{\kern0pt}case\ \isacommand{using}\isamarkupfalse%
\ injectivity{\isacharunderscore}{\kern0pt}theta{\isacharparenleft}{\kern0pt}{\isadigit{3}}{\isacharparenright}{\kern0pt}\ \isacommand{by}\isamarkupfalse%
\ blast\isanewline
\isacommand{qed}\isamarkupfalse%
%
\endisatagproof
{\isafoldproof}%
%
\isadelimproof
\isanewline
%
\endisadelimproof
\isanewline
\isacommand{end}\isamarkupfalse%
\ %
\isamarkupcmt{of \isa{locale\ lts{\isacharunderscore}{\kern0pt}timeout{\isacharunderscore}{\kern0pt}mappable}%
}%
\isadelimtheory
%
\endisadelimtheory
%
\isatagtheory
%
\endisatagtheory
{\isafoldtheory}%
%
\isadelimtheory
%
\endisadelimtheory
%
\end{isabellebody}%
\endinput
%:%file=~/reactive-bisimilarity-reduction/Mapping_for_Transition_Systems.thy%:%
%:%24=8%:%
%:%36=9%:%
%:%40=11%:%
%:%41=12%:%
%:%42=13%:%
%:%43=14%:%
%:%44=15%:%
%:%45=16%:%
%:%46=17%:%
%:%47=18%:%
%:%48=19%:%
%:%49=20%:%
%:%50=21%:%
%:%51=22%:%
%:%52=23%:%
%:%53=24%:%
%:%54=25%:%
%:%55=26%:%
%:%56=27%:%
%:%57=28%:%
%:%58=29%:%
%:%59=30%:%
%:%60=31%:%
%:%61=32%:%
%:%62=33%:%
%:%63=34%:%
%:%64=35%:%
%:%65=36%:%
%:%66=37%:%
%:%67=38%:%
%:%68=39%:%
%:%69=40%:%
%:%70=41%:%
%:%71=42%:%
%:%72=43%:%
%:%73=44%:%
%:%74=45%:%
%:%75=46%:%
%:%76=47%:%
%:%77=48%:%
%:%78=49%:%
%:%79=50%:%
%:%80=51%:%
%:%81=52%:%
%:%82=53%:%
%:%83=54%:%
%:%84=55%:%
%:%85=56%:%
%:%86=57%:%
%:%87=58%:%
%:%88=59%:%
%:%89=60%:%
%:%90=61%:%
%:%91=62%:%
%:%92=63%:%
%:%93=64%:%
%:%94=65%:%
%:%95=66%:%
%:%96=67%:%
%:%97=68%:%
%:%98=69%:%
%:%99=70%:%
%:%100=71%:%
%:%101=72%:%
%:%102=73%:%
%:%103=74%:%
%:%104=75%:%
%:%105=76%:%
%:%106=77%:%
%:%107=78%:%
%:%108=79%:%
%:%109=80%:%
%:%110=81%:%
%:%111=82%:%
%:%112=83%:%
%:%113=84%:%
%:%114=85%:%
%:%115=86%:%
%:%116=87%:%
%:%117=88%:%
%:%118=89%:%
%:%119=90%:%
%:%120=91%:%
%:%121=92%:%
%:%122=93%:%
%:%123=94%:%
%:%124=95%:%
%:%125=96%:%
%:%126=97%:%
%:%127=98%:%
%:%128=99%:%
%:%129=100%:%
%:%130=101%:%
%:%131=102%:%
%:%132=103%:%
%:%133=104%:%
%:%134=105%:%
%:%135=106%:%
%:%136=107%:%
%:%137=108%:%
%:%138=109%:%
%:%139=110%:%
%:%140=111%:%
%:%141=112%:%
%:%150=115%:%
%:%154=117%:%
%:%166=119%:%
%:%167=120%:%
%:%168=121%:%
%:%169=122%:%
%:%170=123%:%
%:%171=124%:%
%:%172=125%:%
%:%173=126%:%
%:%174=127%:%
%:%175=128%:%
%:%176=129%:%
%:%177=130%:%
%:%178=131%:%
%:%179=132%:%
%:%180=133%:%
%:%182=135%:%
%:%183=135%:%
%:%184=136%:%
%:%185=137%:%
%:%186=138%:%
%:%187=139%:%
%:%188=140%:%
%:%189=141%:%
%:%190=142%:%
%:%191=143%:%
%:%192=144%:%
%:%193=145%:%
%:%194=146%:%
%:%195=147%:%
%:%196=148%:%
%:%197=149%:%
%:%198=150%:%
%:%199=151%:%
%:%200=152%:%
%:%201=153%:%
%:%202=154%:%
%:%203=155%:%
%:%204=156%:%
%:%205=157%:%
%:%206=158%:%
%:%207=159%:%
%:%208=160%:%
%:%209=161%:%
%:%210=162%:%
%:%212=164%:%
%:%214=166%:%
%:%215=166%:%
%:%216=167%:%
%:%217=168%:%
%:%218=169%:%
%:%219=169%:%
%:%220=170%:%
%:%221=171%:%
%:%223=173%:%
%:%225=175%:%
%:%226=175%:%
%:%227=176%:%
%:%228=177%:%
%:%229=178%:%
%:%250=187%:%
%:%262=188%:%
%:%263=189%:%
%:%264=190%:%
%:%265=191%:%
%:%266=192%:%
%:%268=195%:%
%:%269=195%:%
%:%270=196%:%
%:%271=197%:%
%:%272=198%:%
%:%273=199%:%
%:%274=200%:%
%:%275=201%:%
%:%276=202%:%
%:%277=203%:%
%:%278=204%:%
%:%279=205%:%
%:%280=206%:%
%:%281=207%:%
%:%282=208%:%
%:%290=212%:%
%:%302=214%:%
%:%304=216%:%
%:%305=216%:%
%:%306=217%:%
%:%307=218%:%
%:%308=219%:%
%:%311=220%:%
%:%315=220%:%
%:%316=220%:%
%:%317=221%:%
%:%318=221%:%
%:%323=221%:%
%:%326=222%:%
%:%327=223%:%
%:%328=223%:%
%:%329=224%:%
%:%330=225%:%
%:%333=226%:%
%:%337=226%:%
%:%338=226%:%
%:%339=227%:%
%:%340=227%:%
%:%345=227%:%
%:%348=228%:%
%:%349=229%:%
%:%350=229%:%
%:%351=230%:%
%:%352=231%:%
%:%355=232%:%
%:%359=232%:%
%:%360=232%:%
%:%361=233%:%
%:%362=233%:%
%:%367=233%:%
%:%370=234%:%
%:%371=235%:%
%:%372=235%:%
%:%373=236%:%
%:%374=237%:%
%:%377=238%:%
%:%381=238%:%
%:%382=238%:%
%:%383=239%:%
%:%384=239%:%
%:%385=239%:%
%:%390=239%:%
%:%393=240%:%
%:%394=241%:%
%:%395=241%:%
%:%396=242%:%
%:%397=243%:%
%:%400=244%:%
%:%404=244%:%
%:%405=244%:%
%:%406=245%:%
%:%407=245%:%
%:%412=245%:%
%:%415=246%:%
%:%416=247%:%
%:%417=247%:%
%:%418=248%:%
%:%419=249%:%
%:%422=250%:%
%:%426=250%:%
%:%427=250%:%
%:%428=251%:%
%:%429=251%:%
%:%430=252%:%
%:%431=252%:%
%:%432=253%:%
%:%433=253%:%
%:%434=253%:%
%:%435=253%:%
%:%436=254%:%
%:%437=254%:%
%:%438=255%:%
%:%439=255%:%
%:%440=256%:%
%:%441=256%:%
%:%442=256%:%
%:%443=256%:%
%:%444=257%:%
%:%445=257%:%
%:%446=257%:%
%:%447=257%:%
%:%448=258%:%
%:%449=258%:%
%:%450=258%:%
%:%451=259%:%
%:%452=259%:%
%:%453=260%:%
%:%454=260%:%
%:%455=261%:%
%:%456=261%:%
%:%457=261%:%
%:%458=261%:%
%:%459=261%:%
%:%460=262%:%
%:%461=262%:%
%:%462=263%:%
%:%463=263%:%
%:%464=264%:%
%:%465=264%:%
%:%466=264%:%
%:%467=265%:%
%:%468=265%:%
%:%469=265%:%
%:%470=265%:%
%:%471=265%:%
%:%472=266%:%
%:%473=266%:%
%:%474=267%:%
%:%475=267%:%
%:%476=268%:%
%:%477=268%:%
%:%478=268%:%
%:%479=268%:%
%:%480=269%:%
%:%481=269%:%
%:%482=269%:%
%:%483=270%:%
%:%484=270%:%
%:%485=271%:%
%:%486=271%:%
%:%487=272%:%
%:%488=272%:%
%:%489=272%:%
%:%490=272%:%
%:%491=273%:%
%:%492=273%:%
%:%493=273%:%
%:%494=274%:%
%:%500=274%:%
%:%503=275%:%
%:%504=276%:%
%:%505=276%:%
%:%506=277%:%
%:%507=278%:%
%:%510=279%:%
%:%514=279%:%
%:%515=279%:%
%:%516=280%:%
%:%517=280%:%
%:%518=281%:%
%:%519=281%:%
%:%520=282%:%
%:%521=282%:%
%:%522=282%:%
%:%523=282%:%
%:%524=283%:%
%:%525=283%:%
%:%526=283%:%
%:%527=284%:%
%:%528=284%:%
%:%529=285%:%
%:%530=285%:%
%:%531=286%:%
%:%532=286%:%
%:%533=286%:%
%:%534=286%:%
%:%535=287%:%
%:%536=287%:%
%:%537=287%:%
%:%538=287%:%
%:%539=288%:%
%:%540=288%:%
%:%541=288%:%
%:%542=289%:%
%:%543=289%:%
%:%544=290%:%
%:%545=290%:%
%:%546=291%:%
%:%547=291%:%
%:%548=291%:%
%:%549=291%:%
%:%550=292%:%
%:%551=292%:%
%:%552=292%:%
%:%553=293%:%
%:%554=293%:%
%:%555=294%:%
%:%556=294%:%
%:%557=295%:%
%:%558=295%:%
%:%559=295%:%
%:%560=295%:%
%:%561=295%:%
%:%562=296%:%
%:%563=296%:%
%:%564=297%:%
%:%565=297%:%
%:%566=298%:%
%:%567=298%:%
%:%568=298%:%
%:%569=299%:%
%:%570=299%:%
%:%571=299%:%
%:%572=299%:%
%:%573=299%:%
%:%574=300%:%
%:%575=300%:%
%:%576=301%:%
%:%577=301%:%
%:%578=302%:%
%:%579=302%:%
%:%580=302%:%
%:%581=302%:%
%:%582=303%:%
%:%583=303%:%
%:%584=303%:%
%:%585=304%:%
%:%591=304%:%
%:%594=305%:%
%:%595=306%:%
%:%596=306%:%
%:%597=307%:%
%:%598=308%:%
%:%601=309%:%
%:%605=309%:%
%:%606=309%:%
%:%607=310%:%
%:%608=310%:%
%:%609=311%:%
%:%610=311%:%
%:%611=312%:%
%:%612=312%:%
%:%613=312%:%
%:%614=312%:%
%:%615=313%:%
%:%616=313%:%
%:%617=313%:%
%:%618=314%:%
%:%619=314%:%
%:%620=315%:%
%:%621=315%:%
%:%622=316%:%
%:%623=316%:%
%:%624=316%:%
%:%625=316%:%
%:%626=317%:%
%:%627=317%:%
%:%628=317%:%
%:%629=317%:%
%:%630=318%:%
%:%631=318%:%
%:%632=318%:%
%:%633=319%:%
%:%634=319%:%
%:%635=320%:%
%:%636=320%:%
%:%637=321%:%
%:%638=321%:%
%:%639=321%:%
%:%640=321%:%
%:%641=322%:%
%:%642=322%:%
%:%643=322%:%
%:%644=323%:%
%:%645=323%:%
%:%646=324%:%
%:%647=324%:%
%:%648=325%:%
%:%649=325%:%
%:%650=325%:%
%:%651=325%:%
%:%652=325%:%
%:%653=326%:%
%:%654=326%:%
%:%655=327%:%
%:%656=327%:%
%:%657=328%:%
%:%658=328%:%
%:%659=328%:%
%:%660=328%:%
%:%661=329%:%
%:%662=329%:%
%:%663=329%:%
%:%664=330%:%
%:%665=330%:%
%:%666=331%:%
%:%667=331%:%
%:%668=332%:%
%:%669=332%:%
%:%670=332%:%
%:%671=333%:%
%:%672=333%:%
%:%673=333%:%
%:%674=333%:%
%:%675=333%:%
%:%676=334%:%
%:%682=334%:%
%:%685=335%:%
%:%686=336%:%
%:%687=336%:%
%:%688=336%:%
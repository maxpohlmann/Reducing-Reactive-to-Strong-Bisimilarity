%
\begin{isabellebody}%
\setisabellecontext{Reactive{\isacharunderscore}{\kern0pt}Bisimilarity}%
%
\isadelimtheory
%
\endisadelimtheory
%
\isatagtheory
%
\endisatagtheory
{\isafoldtheory}%
%
\isadelimtheory
%
\endisadelimtheory
%
\isadelimdocument
%
\endisadelimdocument
%
\isatagdocument
%
\isamarkupsection{Reactive Bisimilarity%
}
\isamarkuptrue%
%
\endisatagdocument
{\isafolddocument}%
%
\isadelimdocument
%
\endisadelimdocument
%
\begin{isamarkuptext}%
\label{sec:reactive_bisimilarity}%
\end{isamarkuptext}\isamarkuptrue%
%
\begin{isamarkuptext}%
In the examples of the previous section, we saw that there are \LTSt{}s with transitions that can never be performed or that can only be performed in certain environments. The behavioural equivalence implied hereby is defined in \cite{rbs} as \emph{strong reactive bisimilarity}.

\example{%
The processes $p$ and $q$ are behaviourally equivalent for \LTSt{} semantics, i.e.\@ strongly reactive bisimilar.

\lts{
    \node[state]    (p0)                            {$p$};
    \node[state]    (p1) [below left of=p0]         {$p_1$};
    \node[state]    (p2) [below right of=p0]        {$p_2$};
    \node[state]    (p3) [below left of=p2]         {$p_3$};
    \node[state]    (p4) [below right of=p2]        {$p_4$};
    
    \path   (p0) edge node[above left]  {$a$}   (p1)
                 edge node              {$t$}   (p2)
            (p2) edge node[above left]  {$\tau$}(p3)
                 edge node              {$a$}   (p4);
                 
    \node[state]    (q0) [right of=p0,xshift=5cm]   {$q$};
    \node[state]    (q1) [below left of=q0]         {$q_1$};
    \node[state]    (q2) [below right of=q0]        {$q_2$};
    \node[state]    (q3) [below left of=q2]         {$q_3$};
    
    \path   (q0) edge node[above left]  {$a$}   (q1)
                 edge node              {$t$}   (q2)
            (q2) edge node[above left]  {$\tau$}(q3);
}}%
\end{isamarkuptext}\isamarkuptrue%
%
\isadelimdocument
%
\endisadelimdocument
%
\isatagdocument
%
\isamarkupsubsubsection{Strong Reactive Bisimulations%
}
\isamarkuptrue%
%
\endisatagdocument
{\isafolddocument}%
%
\isadelimdocument
%
\endisadelimdocument
%
\begin{isamarkuptext}%
Van~Glabbeek introduces several characterisations of this equivalence, beginning with \emph{strong reactive bisimulation} (SRB) relations. These differ from strong bisimulations in that the relations contain not only pairs of processes $(p,q)$, but additionally triples consisting of two processes and a set of actions, $(p,X,q)$. The following definition of SRB relations is quoted, with minor adaptations, from \cite[Definition 1]{rbs}:

A \emph{strong reactive bisimulation} is a symmetric relation 
$$\mathcal{R} \subseteq (\Proc \times \mathcal{P}(A) \times \Proc) \cup (\Proc \times \Proc)$$
(meaning that $(p,X,q)\!\in\!\mathcal{R}\!\iff\!(q,X,p)\!\in\!\mathcal{R}$ and
$(p,q)\!\in\!\mathcal{R}\!\iff\!(q,p)\!\in\!\mathcal{R}$),\\
such that, for all $(p,q) \in \mathcal{R}$:
\begin{enumerate}
    \item if $p \xrightarrow{\tau} p'$, then there exists a $q'$ such that $q \xrightarrow{\tau} q'$ and $(p',q') \in \mathcal{R}$,
    \item $(p,X,q) \in \mathcal{R}$ for all $X \subseteq A$,
\end{enumerate}
and for all $(p,X,q) \in \mathcal{R}$:
\begin{enumerate}
    \setcounter{enumi}{2}
    \item if $p \xrightarrow{a} p'$ with $a \in X$, then $\exists q'$ such that $q \xrightarrow{a} q'$ and $(p',q') \in \mathcal{R}$,
    \item if $p \xrightarrow{\tau} p'$, then there exists a $q'$ such that $q \xrightarrow{\tau} q'$ and $(p',X,q') \in \mathcal{R}$,
    \item if $\mathcal{I}(p) \cap (X \cup \{\tau\}) = \emptyset$, then $(p,q) \in \mathcal{R}$, and
    \item if $\mathcal{I}(p) \cap (X \cup \{\tau\}) = \emptyset$ and $p \xrightarrow{t} p'$, then there exists a $q'$ such that $q \xrightarrow{t} q'$ and $(p',X,q') \in \mathcal{R}$.
\end{enumerate}

This definition explicitly formalises the semantics of \LTSt{}s. We can derive the following intuitions: an environment can either be stable, allowing a specific set of actions, or indeterminate. Indeterminate environments cannot facilitate any transitions, but they can stabilise into arbitrary stable environments. This is expressed by clause 2. Hence, $X$-bisimilarity is behavioural equivalence in stable environments~$X$, and reactive bisimilarity is behavioural equivalence in indeterminate environments (and thus in arbitrary stable environments).

Since only stable environments can facilitate transitions, there are no clauses involving visible action transitions for $(p,q) \in \mathcal{R}$. However, $\tau$-transitions can be performed regardless of the environment, hence clause 1.

At this point, it is important to discuss what exactly it means for an action to be visible or hidden in this context: as we saw in the last section, the environment cannot react (change its set of allowed actions) when the system performs a $\tau$- or a $t$-transition, since these are hidden actions. However, since we are talking about a \emph{strong} bisimilarity (as opposed to e.g.\@ weak bisimilarity briefly mentioned in \cref{sec:strong_bisimilarity}), the performance of $\tau$- or $t$-transitions is still relevant when examining and comparing the behavior of systems.

Therewith, we can look more closely at the remaining clauses:
in clause 3, given $(p,X,q) \in \mathcal{R}$, for $p \xrightarrow{a} p'$ with $a \in X$, we require for the \enquote{mirroring} state $q'$ that $(p',q') \in \mathcal{R}$, because $a$ is a visible action and the transition can thus trigger a change of the environment;%
\footnote{This is why van~Glabbeek talks about \emph{triggered} environments rather than indeterminate ones. I will use both terms interchangeably.}
on the other hand, in clause 4, for $p \xrightarrow{\tau} p'$, and in clause 6, for $p \xrightarrow{t} p'$, we require $(p',X,q') \in \mathcal{R}$, because these actions are hidden and cannot trigger a change of the environment.

Lastly, clause 5 formalises the possibility of the environment timing out (i.e.\@ turning into an indeterminate environment) instead of the system.

These intuitions also form the basis for the process mapping which will be presented in \cref{sec:mapping_lts}.%
\end{isamarkuptext}\isamarkuptrue%
%
\isadelimdocument
%
\endisadelimdocument
%
\isatagdocument
%
\isamarkupsubsubsection{Strong Reactive/$X$-Bisimilarity%
}
\isamarkuptrue%
%
\endisatagdocument
{\isafolddocument}%
%
\isadelimdocument
%
\endisadelimdocument
%
\begin{isamarkuptext}%
Two processes $p$ and $q$ are \emph{strongly reactive bisimilar} (written $p \leftrightarrow_r q$) iff there is an SRB containing $(p,q)$, and \emph{strongly $X$-bisimilar} (written $p \leftrightarrow_r^X q$), i.e.\@ \enquote{equivalent} in environments~$X$, when there is an SRB containing $(p,X,q)$.%
\end{isamarkuptext}\isamarkuptrue%
%
\isadelimdocument
%
\endisadelimdocument
%
\isatagdocument
%
\isamarkupsubsubsection{Generalised Strong Reactive Bisimulations%
}
\isamarkuptrue%
%
\endisatagdocument
{\isafolddocument}%
%
\isadelimdocument
%
\endisadelimdocument
%
\begin{isamarkuptext}%
Another characterisation of reactive bisimilarity uses \emph{generalised strong reactive bisimulation} (GSRB) relations, defined over the same set as SRBs, but with different clauses \cite[Definition 3]{rbs}. It is proved that both characterisations do, in fact, characterise the same equivalence.%
\end{isamarkuptext}\isamarkuptrue%
%
\isadelimdocument
%
\endisadelimdocument
%
\isatagdocument
%
\isamarkupsubsection{Isabelle%
}
\isamarkuptrue%
%
\endisatagdocument
{\isafolddocument}%
%
\isadelimdocument
%
\endisadelimdocument
%
\begin{isamarkuptext}%
I first formalise both SRB and GSRB relations (as well as strong reactive bisimilarity, defined by the existence of an SRB, as above), and then replicate the proof of their correspondence.%
\end{isamarkuptext}\isamarkuptrue%
%
\isadelimdocument
%
\endisadelimdocument
%
\isatagdocument
%
\isamarkupsubsubsection{Strong Reactive Bisimulations%
}
\isamarkuptrue%
%
\endisatagdocument
{\isafolddocument}%
%
\isadelimdocument
%
\endisadelimdocument
%
\begin{isamarkuptext}%
SRB relations are defined over the set
$$(\Proc \times \mathscr{P}(A) \times \Proc) \cup (\Proc \times \Proc).$$

As can be easily seen, this set it isomorphic to
$$(\Proc \times (\mathscr{P}(A) \cup \{\bot\}) \times \Proc),$$
which is a subset of
$$(\Proc \times (\mathscr{P}(\Act) \cup \{\bot\}) \times \Proc).$$ 

This last set can now be easily formalised in terms of a type, where we formalise
$\mathscr{P}(\Act) \cup \{\bot\}$
as \isa{{\isacharprime}{\kern0pt}a\ set\ option}.

The fact that SRBs are defined using the power set of visible actions ($A$), whereas our type uses all actions ($\Act$ / \isa{{\isacharprime}{\kern0pt}a}), is handled by the first line of the definition below. The second line formalises that symmetry is required by definition. All other lines are direct formalisations of the clauses of the original definition.%
\end{isamarkuptext}\isamarkuptrue%
\isacommand{context}\isamarkupfalse%
\ lts{\isacharunderscore}{\kern0pt}timeout\ \isakeyword{begin}\isanewline
\isanewline
%
\isamarkupcmt{strong reactive bisimulation \cite[Definition 1]{rbs}%
}\isanewline
\isacommand{definition}\isamarkupfalse%
\ SRB\ {\isacharcolon}{\kern0pt}{\isacharcolon}{\kern0pt}\ {\isacartoucheopen}{\isacharparenleft}{\kern0pt}{\isacharprime}{\kern0pt}s\ {\isasymRightarrow}\ {\isacharprime}{\kern0pt}a\ set\ option\ {\isasymRightarrow}\ {\isacharprime}{\kern0pt}s\ {\isasymRightarrow}\ bool{\isacharparenright}{\kern0pt}\ {\isasymRightarrow}\ bool{\isacartoucheclose}\isanewline
\ \ \isakeyword{where}\ {\isacartoucheopen}SRB\ R\ {\isasymequiv}\isanewline
\ \ {\isacharparenleft}{\kern0pt}{\isasymforall}\ p\ X\ q{\isachardot}{\kern0pt}\ R\ p\ {\isacharparenleft}{\kern0pt}Some\ X{\isacharparenright}{\kern0pt}\ q\ \ {\isasymlongrightarrow}\ \ X\ {\isasymsubseteq}\ visible{\isacharunderscore}{\kern0pt}actions{\isacharparenright}{\kern0pt}\ {\isasymand}\isanewline
\ \ {\isacharparenleft}{\kern0pt}{\isasymforall}\ p\ XoN\ q{\isachardot}{\kern0pt}\ R\ p\ XoN\ q\ \ {\isasymlongrightarrow}\ \ R\ q\ XoN\ p{\isacharparenright}{\kern0pt}\ {\isasymand}\isanewline
\isanewline
\ \ {\isacharparenleft}{\kern0pt}{\isasymforall}\ p\ q{\isachardot}{\kern0pt}\ R\ p\ None\ q\ {\isasymlongrightarrow}\isanewline
\ \ \ \ {\isacharparenleft}{\kern0pt}{\isasymforall}\ p{\isacharprime}{\kern0pt}{\isachardot}{\kern0pt}\ p\ {\isasymlongmapsto}{\isasymtau}\ p{\isacharprime}{\kern0pt}\ \ {\isasymlongrightarrow}\ \ {\isacharparenleft}{\kern0pt}{\isasymexists}\ q{\isacharprime}{\kern0pt}{\isachardot}{\kern0pt}\ {\isacharparenleft}{\kern0pt}q\ {\isasymlongmapsto}{\isasymtau}\ q{\isacharprime}{\kern0pt}{\isacharparenright}{\kern0pt}\ {\isasymand}\ R\ p{\isacharprime}{\kern0pt}\ None\ q{\isacharprime}{\kern0pt}{\isacharparenright}{\kern0pt}{\isacharparenright}{\kern0pt}\ {\isasymand}\isanewline
\ \ \ \ {\isacharparenleft}{\kern0pt}{\isasymforall}\ X\ {\isasymsubseteq}\ visible{\isacharunderscore}{\kern0pt}actions{\isachardot}{\kern0pt}\ {\isacharparenleft}{\kern0pt}R\ p\ {\isacharparenleft}{\kern0pt}Some\ X{\isacharparenright}{\kern0pt}\ q{\isacharparenright}{\kern0pt}{\isacharparenright}{\kern0pt}{\isacharparenright}{\kern0pt}\ {\isasymand}\isanewline
\isanewline
\ \ {\isacharparenleft}{\kern0pt}{\isasymforall}\ p\ X\ q{\isachardot}{\kern0pt}\ R\ p\ {\isacharparenleft}{\kern0pt}Some\ X{\isacharparenright}{\kern0pt}\ q\ {\isasymlongrightarrow}\isanewline
\ \ \ \ {\isacharparenleft}{\kern0pt}{\isasymforall}\ p{\isacharprime}{\kern0pt}\ a{\isachardot}{\kern0pt}\ p\ {\isasymlongmapsto}a\ p{\isacharprime}{\kern0pt}\ {\isasymand}\ a\ {\isasymin}\ X\ \ {\isasymlongrightarrow}\ \ {\isacharparenleft}{\kern0pt}{\isasymexists}\ q{\isacharprime}{\kern0pt}{\isachardot}{\kern0pt}\ {\isacharparenleft}{\kern0pt}q\ {\isasymlongmapsto}a\ q{\isacharprime}{\kern0pt}{\isacharparenright}{\kern0pt}\ {\isasymand}\ \isanewline
\ \ \ \ \ \ R\ p{\isacharprime}{\kern0pt}\ None\ q{\isacharprime}{\kern0pt}{\isacharparenright}{\kern0pt}{\isacharparenright}{\kern0pt}\ {\isasymand}\isanewline
\ \ \ \ {\isacharparenleft}{\kern0pt}{\isasymforall}\ p{\isacharprime}{\kern0pt}{\isachardot}{\kern0pt}\ p\ {\isasymlongmapsto}{\isasymtau}\ p{\isacharprime}{\kern0pt}\ \ {\isasymlongrightarrow}\ \ {\isacharparenleft}{\kern0pt}{\isasymexists}\ q{\isacharprime}{\kern0pt}{\isachardot}{\kern0pt}\ {\isacharparenleft}{\kern0pt}q\ {\isasymlongmapsto}{\isasymtau}\ q{\isacharprime}{\kern0pt}{\isacharparenright}{\kern0pt}\ {\isasymand}\ R\ p{\isacharprime}{\kern0pt}\ {\isacharparenleft}{\kern0pt}Some\ X{\isacharparenright}{\kern0pt}\ q{\isacharprime}{\kern0pt}{\isacharparenright}{\kern0pt}{\isacharparenright}{\kern0pt}\ {\isasymand}\isanewline
\ \ \ \ {\isacharparenleft}{\kern0pt}idle\ p\ X\ \ {\isasymlongrightarrow}\ \ R\ p\ None\ q{\isacharparenright}{\kern0pt}\ {\isasymand}\isanewline
\ \ \ \ {\isacharparenleft}{\kern0pt}{\isasymforall}\ p{\isacharprime}{\kern0pt}{\isachardot}{\kern0pt}\ idle\ p\ X\ {\isasymand}\ {\isacharparenleft}{\kern0pt}p\ {\isasymlongmapsto}t\ p{\isacharprime}{\kern0pt}{\isacharparenright}{\kern0pt}\ \ {\isasymlongrightarrow}\ \ {\isacharparenleft}{\kern0pt}{\isasymexists}\ q{\isacharprime}{\kern0pt}{\isachardot}{\kern0pt}\ q\ {\isasymlongmapsto}t\ q{\isacharprime}{\kern0pt}\ {\isasymand}\ \isanewline
\ \ \ \ \ \ R\ p{\isacharprime}{\kern0pt}\ {\isacharparenleft}{\kern0pt}Some\ X{\isacharparenright}{\kern0pt}\ q{\isacharprime}{\kern0pt}{\isacharparenright}{\kern0pt}{\isacharparenright}{\kern0pt}{\isacharparenright}{\kern0pt}{\isacartoucheclose}\isanewline
\isanewline
%
\isadelimunimportant
%
\endisadelimunimportant
%
\isatagunimportant
%
\endisatagunimportant
{\isafoldunimportant}%
%
\isadelimunimportant
%
\endisadelimunimportant
%
\isadelimdocument
%
\endisadelimdocument
%
\isatagdocument
%
\isamarkupsubsubsection{Strong Reactive/$X$-Bisimilarity%
}
\isamarkuptrue%
%
\endisatagdocument
{\isafolddocument}%
%
\isadelimdocument
%
\endisadelimdocument
%
\begin{isamarkuptext}%
Van~Glabbeek differentiates between strong reactive bisimilarity ($(p,q) \in \mathcal{R}$ for an SRB $\mathcal{R}$) and strong $X$-bisimilarity ($(p,X,q) \in \mathcal{R}$ for an SRB $\mathcal{R}$).%
\end{isamarkuptext}\isamarkuptrue%
\isacommand{definition}\isamarkupfalse%
\ strongly{\isacharunderscore}{\kern0pt}reactive{\isacharunderscore}{\kern0pt}bisimilar\ {\isacharcolon}{\kern0pt}{\isacharcolon}{\kern0pt}\ {\isacartoucheopen}{\isacharprime}{\kern0pt}s\ {\isasymRightarrow}\ {\isacharprime}{\kern0pt}s\ {\isasymRightarrow}\ bool{\isacartoucheclose}\ \isanewline
\ \ {\isacharparenleft}{\kern0pt}{\isacartoucheopen}{\isacharunderscore}{\kern0pt}\ {\isasymleftrightarrow}\isactrlsub r\ {\isacharunderscore}{\kern0pt}{\isacartoucheclose}\ {\isacharbrackleft}{\kern0pt}{\isadigit{7}}{\isadigit{0}}{\isacharcomma}{\kern0pt}\ {\isadigit{7}}{\isadigit{0}}{\isacharbrackright}{\kern0pt}\ {\isadigit{7}}{\isadigit{0}}{\isacharparenright}{\kern0pt}\isanewline
\ \ \isakeyword{where}\ {\isacartoucheopen}p\ {\isasymleftrightarrow}\isactrlsub r\ q\ {\isasymequiv}\ {\isasymexists}\ R{\isachardot}{\kern0pt}\ SRB\ R\ {\isasymand}\ R\ p\ None\ q{\isacartoucheclose}\isanewline
\isanewline
\isacommand{definition}\isamarkupfalse%
\ strongly{\isacharunderscore}{\kern0pt}X{\isacharunderscore}{\kern0pt}bisimilar\ {\isacharcolon}{\kern0pt}{\isacharcolon}{\kern0pt}\ {\isacartoucheopen}{\isacharprime}{\kern0pt}s\ {\isasymRightarrow}\ {\isacharprime}{\kern0pt}a\ set\ {\isasymRightarrow}\ {\isacharprime}{\kern0pt}s\ {\isasymRightarrow}\ bool{\isacartoucheclose}\ \isanewline
\ \ {\isacharparenleft}{\kern0pt}{\isacartoucheopen}{\isacharunderscore}{\kern0pt}\ {\isasymleftrightarrow}\isactrlsub r\isactrlsup {\isacharunderscore}{\kern0pt}\ {\isacharunderscore}{\kern0pt}{\isacartoucheclose}\ {\isacharbrackleft}{\kern0pt}{\isadigit{7}}{\isadigit{0}}{\isacharcomma}{\kern0pt}\ {\isadigit{7}}{\isadigit{0}}{\isacharcomma}{\kern0pt}\ {\isadigit{7}}{\isadigit{0}}{\isacharbrackright}{\kern0pt}\ {\isadigit{7}}{\isadigit{0}}{\isacharparenright}{\kern0pt}\isanewline
\ \ \isakeyword{where}\ {\isacartoucheopen}p\ {\isasymleftrightarrow}\isactrlsub r\isactrlsup X\ q\ {\isasymequiv}\ {\isasymexists}\ R{\isachardot}{\kern0pt}\ SRB\ R\ {\isasymand}\ R\ p\ {\isacharparenleft}{\kern0pt}Some\ X{\isacharparenright}{\kern0pt}\ q{\isacartoucheclose}%
\begin{isamarkuptext}%
For the upcoming proofs, it is useful to combine both reactive and $X$-bisimilarity into a single one.%
\end{isamarkuptext}\isamarkuptrue%
\isacommand{definition}\isamarkupfalse%
\ strongly{\isacharunderscore}{\kern0pt}reactive{\isacharunderscore}{\kern0pt}or{\isacharunderscore}{\kern0pt}X{\isacharunderscore}{\kern0pt}bisimilar\ \isanewline
\ \ {\isacharcolon}{\kern0pt}{\isacharcolon}{\kern0pt}\ {\isacartoucheopen}{\isacharprime}{\kern0pt}s\ {\isasymRightarrow}\ {\isacharprime}{\kern0pt}a\ set\ option\ {\isasymRightarrow}\ {\isacharprime}{\kern0pt}s\ {\isasymRightarrow}\ bool{\isacartoucheclose}\isanewline
\ \ \isakeyword{where}\ {\isacartoucheopen}strongly{\isacharunderscore}{\kern0pt}reactive{\isacharunderscore}{\kern0pt}or{\isacharunderscore}{\kern0pt}X{\isacharunderscore}{\kern0pt}bisimilar\ p\ XoN\ q\ \isanewline
\ \ \ \ {\isasymequiv}\ {\isasymexists}\ R{\isachardot}{\kern0pt}\ SRB\ R\ {\isasymand}\ R\ p\ XoN\ q{\isacartoucheclose}%
\begin{isamarkuptext}%
Obviously, then, these predicates coincide accordingly.%
\end{isamarkuptext}\isamarkuptrue%
\isacommand{corollary}\isamarkupfalse%
\ {\isacartoucheopen}p\ {\isasymleftrightarrow}\isactrlsub r\ q\ {\isasymLongleftrightarrow}\ strongly{\isacharunderscore}{\kern0pt}reactive{\isacharunderscore}{\kern0pt}or{\isacharunderscore}{\kern0pt}X{\isacharunderscore}{\kern0pt}bisimilar\ p\ None\ q{\isacartoucheclose}\isanewline
%
\isadelimproof
\ \ %
\endisadelimproof
%
\isatagproof
\isacommand{using}\isamarkupfalse%
\ strongly{\isacharunderscore}{\kern0pt}reactive{\isacharunderscore}{\kern0pt}bisimilar{\isacharunderscore}{\kern0pt}def\ strongly{\isacharunderscore}{\kern0pt}reactive{\isacharunderscore}{\kern0pt}or{\isacharunderscore}{\kern0pt}X{\isacharunderscore}{\kern0pt}bisimilar{\isacharunderscore}{\kern0pt}def\ \isacommand{by}\isamarkupfalse%
\ force%
\endisatagproof
{\isafoldproof}%
%
\isadelimproof
\isanewline
%
\endisadelimproof
\isacommand{corollary}\isamarkupfalse%
\ {\isacartoucheopen}p\ {\isasymleftrightarrow}\isactrlsub r\isactrlsup X\ q\ {\isasymLongleftrightarrow}\ strongly{\isacharunderscore}{\kern0pt}reactive{\isacharunderscore}{\kern0pt}or{\isacharunderscore}{\kern0pt}X{\isacharunderscore}{\kern0pt}bisimilar\ p\ {\isacharparenleft}{\kern0pt}Some\ X{\isacharparenright}{\kern0pt}\ q{\isacartoucheclose}\isanewline
%
\isadelimproof
\ \ %
\endisadelimproof
%
\isatagproof
\isacommand{using}\isamarkupfalse%
\ strongly{\isacharunderscore}{\kern0pt}X{\isacharunderscore}{\kern0pt}bisimilar{\isacharunderscore}{\kern0pt}def\ strongly{\isacharunderscore}{\kern0pt}reactive{\isacharunderscore}{\kern0pt}or{\isacharunderscore}{\kern0pt}X{\isacharunderscore}{\kern0pt}bisimilar{\isacharunderscore}{\kern0pt}def\ \isacommand{by}\isamarkupfalse%
\ force%
\endisatagproof
{\isafoldproof}%
%
\isadelimproof
%
\endisadelimproof
%
\isadelimdocument
%
\endisadelimdocument
%
\isatagdocument
%
\isamarkupsubsubsection{Generalised Strong Reactive Bisimulations%
}
\isamarkuptrue%
%
\endisatagdocument
{\isafolddocument}%
%
\isadelimdocument
%
\endisadelimdocument
%
\begin{isamarkuptext}%
Since GSRBs are defined over the same set as SRBs, the same considerations as above hold.%
\end{isamarkuptext}\isamarkuptrue%
%
\isamarkupcmt{generalised strong reactive bisimulation \cite[Definition 3]{rbs}%
}\isanewline
\isacommand{definition}\isamarkupfalse%
\ GSRB\ {\isacharcolon}{\kern0pt}{\isacharcolon}{\kern0pt}\ {\isacartoucheopen}{\isacharparenleft}{\kern0pt}{\isacharprime}{\kern0pt}s\ {\isasymRightarrow}\ {\isacharprime}{\kern0pt}a\ set\ option\ {\isasymRightarrow}\ {\isacharprime}{\kern0pt}s\ {\isasymRightarrow}\ bool{\isacharparenright}{\kern0pt}\ {\isasymRightarrow}\ bool{\isacartoucheclose}\isanewline
\ \ \isakeyword{where}\ {\isacartoucheopen}GSRB\ R\ {\isasymequiv}\isanewline
\ \ \ \ {\isacharparenleft}{\kern0pt}{\isasymforall}\ p\ X\ q{\isachardot}{\kern0pt}\ R\ p\ {\isacharparenleft}{\kern0pt}Some\ X{\isacharparenright}{\kern0pt}\ q\ \ {\isasymlongrightarrow}\ \ X\ {\isasymsubseteq}\ visible{\isacharunderscore}{\kern0pt}actions{\isacharparenright}{\kern0pt}\ {\isasymand}\isanewline
\ \ \ \ {\isacharparenleft}{\kern0pt}{\isasymforall}\ p\ XoN\ q{\isachardot}{\kern0pt}\ R\ p\ XoN\ q\ \ {\isasymlongrightarrow}\ \ R\ q\ XoN\ p{\isacharparenright}{\kern0pt}\ {\isasymand}\isanewline
\isanewline
\ \ \ \ {\isacharparenleft}{\kern0pt}{\isasymforall}\ p\ q{\isachardot}{\kern0pt}\ R\ p\ None\ q\ {\isasymlongrightarrow}\isanewline
\ \ \ \ \ \ {\isacharparenleft}{\kern0pt}{\isasymforall}\ p{\isacharprime}{\kern0pt}\ a{\isachardot}{\kern0pt}\ p\ {\isasymlongmapsto}a\ p{\isacharprime}{\kern0pt}\ {\isasymand}\ a\ {\isasymin}\ visible{\isacharunderscore}{\kern0pt}actions\ {\isasymunion}\ {\isacharbraceleft}{\kern0pt}{\isasymtau}{\isacharbraceright}{\kern0pt}\ \ {\isasymlongrightarrow}\isanewline
\ \ \ \ \ \ \ \ {\isacharparenleft}{\kern0pt}{\isasymexists}\ q{\isacharprime}{\kern0pt}{\isachardot}{\kern0pt}\ q\ {\isasymlongmapsto}a\ q{\isacharprime}{\kern0pt}\ {\isasymand}\ R\ p{\isacharprime}{\kern0pt}\ None\ q{\isacharprime}{\kern0pt}{\isacharparenright}{\kern0pt}{\isacharparenright}{\kern0pt}\ {\isasymand}\isanewline
\ \ \ \ \ \ {\isacharparenleft}{\kern0pt}{\isasymforall}\ X\ p{\isacharprime}{\kern0pt}{\isachardot}{\kern0pt}\ idle\ p\ X\ {\isasymand}\ X\ {\isasymsubseteq}\ visible{\isacharunderscore}{\kern0pt}actions\ {\isasymand}\ p\ {\isasymlongmapsto}t\ p{\isacharprime}{\kern0pt}\ \ {\isasymlongrightarrow}\ \ \isanewline
\ \ \ \ \ \ \ \ {\isacharparenleft}{\kern0pt}{\isasymexists}\ q{\isacharprime}{\kern0pt}{\isachardot}{\kern0pt}\ q\ {\isasymlongmapsto}t\ q{\isacharprime}{\kern0pt}\ {\isasymand}\ R\ p{\isacharprime}{\kern0pt}\ {\isacharparenleft}{\kern0pt}Some\ X{\isacharparenright}{\kern0pt}\ q{\isacharprime}{\kern0pt}{\isacharparenright}{\kern0pt}{\isacharparenright}{\kern0pt}{\isacharparenright}{\kern0pt}\ {\isasymand}\isanewline
\ \ \ \ \isanewline
\ \ \ \ {\isacharparenleft}{\kern0pt}{\isasymforall}\ p\ Y\ q{\isachardot}{\kern0pt}\ R\ p\ {\isacharparenleft}{\kern0pt}Some\ Y{\isacharparenright}{\kern0pt}\ q\ {\isasymlongrightarrow}\isanewline
\ \ \ \ \ \ {\isacharparenleft}{\kern0pt}{\isasymforall}\ p{\isacharprime}{\kern0pt}\ a{\isachardot}{\kern0pt}\ a\ {\isasymin}\ visible{\isacharunderscore}{\kern0pt}actions\ {\isasymand}\ p\ {\isasymlongmapsto}a\ p{\isacharprime}{\kern0pt}\ {\isasymand}\ {\isacharparenleft}{\kern0pt}a{\isasymin}Y\ {\isasymor}\ idle\ p\ Y{\isacharparenright}{\kern0pt}\ {\isasymlongrightarrow}\isanewline
\ \ \ \ \ \ \ \ {\isacharparenleft}{\kern0pt}{\isasymexists}\ q{\isacharprime}{\kern0pt}{\isachardot}{\kern0pt}\ q\ {\isasymlongmapsto}a\ q{\isacharprime}{\kern0pt}\ {\isasymand}\ R\ p{\isacharprime}{\kern0pt}\ None\ q{\isacharprime}{\kern0pt}{\isacharparenright}{\kern0pt}{\isacharparenright}{\kern0pt}\ {\isasymand}\isanewline
\ \ \ \ \ \ {\isacharparenleft}{\kern0pt}{\isasymforall}\ p{\isacharprime}{\kern0pt}{\isachardot}{\kern0pt}\ p\ {\isasymlongmapsto}{\isasymtau}\ p{\isacharprime}{\kern0pt}\ \ {\isasymlongrightarrow}\ \ \isanewline
\ \ \ \ \ \ \ \ {\isacharparenleft}{\kern0pt}{\isasymexists}\ q{\isacharprime}{\kern0pt}{\isachardot}{\kern0pt}\ q\ {\isasymlongmapsto}{\isasymtau}\ q{\isacharprime}{\kern0pt}\ {\isasymand}\ R\ p{\isacharprime}{\kern0pt}\ {\isacharparenleft}{\kern0pt}Some\ Y{\isacharparenright}{\kern0pt}\ q{\isacharprime}{\kern0pt}{\isacharparenright}{\kern0pt}{\isacharparenright}{\kern0pt}\ {\isasymand}\isanewline
\ \ \ \ \ \ {\isacharparenleft}{\kern0pt}{\isasymforall}\ p{\isacharprime}{\kern0pt}\ X{\isachardot}{\kern0pt}\ idle\ p\ {\isacharparenleft}{\kern0pt}X{\isasymunion}Y{\isacharparenright}{\kern0pt}\ {\isasymand}\ X\ {\isasymsubseteq}\ visible{\isacharunderscore}{\kern0pt}actions\ {\isasymand}\ p\ {\isasymlongmapsto}t\ p{\isacharprime}{\kern0pt}\ \ {\isasymlongrightarrow}\ \ \isanewline
\ \ \ \ \ \ \ \ {\isacharparenleft}{\kern0pt}{\isasymexists}\ q{\isacharprime}{\kern0pt}{\isachardot}{\kern0pt}\ q\ {\isasymlongmapsto}t\ q{\isacharprime}{\kern0pt}\ {\isasymand}\ R\ p{\isacharprime}{\kern0pt}\ {\isacharparenleft}{\kern0pt}Some\ X{\isacharparenright}{\kern0pt}\ q{\isacharprime}{\kern0pt}{\isacharparenright}{\kern0pt}{\isacharparenright}{\kern0pt}{\isacharparenright}{\kern0pt}{\isacartoucheclose}\isanewline
\isanewline
%
\isadelimunimportant
%
\endisadelimunimportant
%
\isatagunimportant
%
\endisatagunimportant
{\isafoldunimportant}%
%
\isadelimunimportant
%
\endisadelimunimportant
%
\isadelimdocument
%
\endisadelimdocument
%
\isatagdocument
%
\isamarkupsubsubsection{GSRBs characterise strong reactive/$X$-bisimilarity%
}
\isamarkuptrue%
%
\endisatagdocument
{\isafolddocument}%
%
\isadelimdocument
%
\endisadelimdocument
%
\begin{isamarkuptext}%
\cite[Proposition 4]{rbs} reads (notation adapted): \enquote{$p \leftrightarrow_r q$ iff there exists a GSRB $\mathcal{R}$ with $(p,q) \in \mathcal{R}$. Likewise, $p \leftrightarrow_r^X q$ iff there exists a GSRB $\mathcal{R}$ with $(p,X,q) \in \mathcal{R}$.} We shall now replicate the proof of this proposition. First, we prove that each SRB is a GSRB (by showing that each SRB satisfies all clauses of the definition of GSRBs).%
\end{isamarkuptext}\isamarkuptrue%
\isacommand{lemma}\isamarkupfalse%
\ SRB{\isacharunderscore}{\kern0pt}is{\isacharunderscore}{\kern0pt}GSRB{\isacharcolon}{\kern0pt}\isanewline
\ \ \isakeyword{assumes}\ {\isacartoucheopen}SRB\ R{\isacartoucheclose}\isanewline
\ \ \isakeyword{shows}\ {\isacartoucheopen}GSRB\ R{\isacartoucheclose}\isanewline
%
\isadelimproof
\ \ %
\endisadelimproof
%
\isatagproof
\isacommand{unfolding}\isamarkupfalse%
\ GSRB{\isacharunderscore}{\kern0pt}def\isanewline
\isacommand{proof}\isamarkupfalse%
\ {\isacharparenleft}{\kern0pt}safe{\isacharparenright}{\kern0pt}\isanewline
\ \ \isacommand{fix}\isamarkupfalse%
\ p\ XoN\ q\isanewline
\ \ \isacommand{assume}\isamarkupfalse%
\ {\isacartoucheopen}R\ p\ XoN\ q{\isacartoucheclose}\isanewline
\ \ \isacommand{thus}\isamarkupfalse%
\ {\isacartoucheopen}R\ q\ XoN\ p{\isacartoucheclose}\ \isanewline
\ \ \ \ \isacommand{using}\isamarkupfalse%
\ SRB{\isacharunderscore}{\kern0pt}ruleformat{\isacharbrackleft}{\kern0pt}OF\ assms{\isacharbrackright}{\kern0pt}\ \isacommand{by}\isamarkupfalse%
\ blast\isanewline
\isacommand{next}\isamarkupfalse%
\isanewline
\ \ \isacommand{fix}\isamarkupfalse%
\ p\ X\ q\ x\isanewline
\ \ \isacommand{assume}\isamarkupfalse%
\ {\isacartoucheopen}R\ p\ {\isacharparenleft}{\kern0pt}Some\ X{\isacharparenright}{\kern0pt}\ q{\isacartoucheclose}\ \isakeyword{and}\ {\isacartoucheopen}x\ {\isasymin}\ X{\isacartoucheclose}\isanewline
\ \ \isacommand{thus}\isamarkupfalse%
\ {\isacartoucheopen}x\ {\isasymin}\ visible{\isacharunderscore}{\kern0pt}actions{\isacartoucheclose}\ \isanewline
\ \ \ \ \isacommand{using}\isamarkupfalse%
\ SRB{\isacharunderscore}{\kern0pt}ruleformat{\isacharbrackleft}{\kern0pt}OF\ assms{\isacharbrackright}{\kern0pt}\ \isacommand{by}\isamarkupfalse%
\ blast\isanewline
\isacommand{next}\isamarkupfalse%
\isanewline
\ \ \isacommand{fix}\isamarkupfalse%
\ p\ q\ p{\isacharprime}{\kern0pt}\ a\isanewline
\ \ \isacommand{assume}\isamarkupfalse%
\ {\isacartoucheopen}R\ p\ None\ q{\isacartoucheclose}\ \isakeyword{and}\ {\isacartoucheopen}p\ {\isasymlongmapsto}a\ p{\isacharprime}{\kern0pt}{\isacartoucheclose}\ \isakeyword{and}\ {\isacartoucheopen}a\ {\isasymin}\ visible{\isacharunderscore}{\kern0pt}actions{\isacartoucheclose}\isanewline
\ \ \isacommand{thus}\isamarkupfalse%
\ {\isacartoucheopen}{\isasymexists}q{\isacharprime}{\kern0pt}{\isachardot}{\kern0pt}\ q\ {\isasymlongmapsto}a\ q{\isacharprime}{\kern0pt}\ {\isasymand}\ R\ p{\isacharprime}{\kern0pt}\ None\ q{\isacharprime}{\kern0pt}{\isacartoucheclose}\isanewline
\ \ \ \ \isacommand{using}\isamarkupfalse%
\ SRB{\isacharunderscore}{\kern0pt}ruleformat{\isacharparenleft}{\kern0pt}{\isadigit{4}}{\isacharcomma}{\kern0pt}\ {\isadigit{5}}{\isacharparenright}{\kern0pt}{\isacharbrackleft}{\kern0pt}OF\ assms{\isacharcomma}{\kern0pt}\ \isakeyword{where}\ {\isacharquery}{\kern0pt}X\ {\isacharequal}{\kern0pt}\ {\isacartoucheopen}{\isacharbraceleft}{\kern0pt}a{\isacharbraceright}{\kern0pt}{\isacartoucheclose}{\isacharbrackright}{\kern0pt}\ \isacommand{by}\isamarkupfalse%
\ blast\isanewline
\isacommand{next}\isamarkupfalse%
\isanewline
\ \ \isacommand{fix}\isamarkupfalse%
\ p\ q\ p{\isacharprime}{\kern0pt}\ a\isanewline
\ \ \isacommand{assume}\isamarkupfalse%
\ {\isacartoucheopen}R\ p\ None\ q{\isacartoucheclose}\ \isakeyword{and}\ {\isacartoucheopen}p\ {\isasymlongmapsto}{\isasymtau}\ \ p{\isacharprime}{\kern0pt}{\isacartoucheclose}\isanewline
\ \ \isacommand{thus}\isamarkupfalse%
\ {\isacartoucheopen}{\isasymexists}q{\isacharprime}{\kern0pt}{\isachardot}{\kern0pt}\ q\ {\isasymlongmapsto}{\isasymtau}\ \ q{\isacharprime}{\kern0pt}\ {\isasymand}\ R\ p{\isacharprime}{\kern0pt}\ None\ q{\isacharprime}{\kern0pt}{\isacartoucheclose}\isanewline
\ \ \ \ \isacommand{using}\isamarkupfalse%
\ SRB{\isacharunderscore}{\kern0pt}ruleformat{\isacharparenleft}{\kern0pt}{\isadigit{3}}{\isacharparenright}{\kern0pt}{\isacharbrackleft}{\kern0pt}OF\ assms{\isacharbrackright}{\kern0pt}\ \isacommand{by}\isamarkupfalse%
\ blast\isanewline
\isacommand{next}\isamarkupfalse%
\isanewline
\ \ \isacommand{fix}\isamarkupfalse%
\ p\ q\ X\ p{\isacharprime}{\kern0pt}\isanewline
\ \ \isacommand{assume}\isamarkupfalse%
\ {\isacartoucheopen}R\ p\ None\ q{\isacartoucheclose}\ \isakeyword{and}\ {\isacartoucheopen}idle\ p\ X{\isacartoucheclose}\ \isakeyword{and}\ {\isacartoucheopen}X\ {\isasymsubseteq}\ visible{\isacharunderscore}{\kern0pt}actions{\isacartoucheclose}\ \isakeyword{and}\ {\isacartoucheopen}p\ {\isasymlongmapsto}t\ p{\isacharprime}{\kern0pt}{\isacartoucheclose}\isanewline
\ \ \isacommand{thus}\isamarkupfalse%
\ {\isacartoucheopen}{\isasymexists}q{\isacharprime}{\kern0pt}{\isachardot}{\kern0pt}\ q\ {\isasymlongmapsto}t\ \ q{\isacharprime}{\kern0pt}\ {\isasymand}\ R\ p{\isacharprime}{\kern0pt}\ {\isacharparenleft}{\kern0pt}Some\ X{\isacharparenright}{\kern0pt}\ q{\isacharprime}{\kern0pt}{\isacartoucheclose}\isanewline
\ \ \ \ \isacommand{using}\isamarkupfalse%
\ SRB{\isacharunderscore}{\kern0pt}ruleformat{\isacharparenleft}{\kern0pt}{\isadigit{4}}{\isacharcomma}{\kern0pt}\ {\isadigit{8}}{\isacharparenright}{\kern0pt}{\isacharbrackleft}{\kern0pt}OF\ assms{\isacharbrackright}{\kern0pt}\ \isacommand{by}\isamarkupfalse%
\ blast\isanewline
\isacommand{next}\isamarkupfalse%
\isanewline
\ \ \isacommand{fix}\isamarkupfalse%
\ p\ Y\ q\ p{\isacharprime}{\kern0pt}\ a\isanewline
\ \ \isacommand{assume}\isamarkupfalse%
\ {\isacartoucheopen}R\ p\ {\isacharparenleft}{\kern0pt}Some\ Y{\isacharparenright}{\kern0pt}\ q{\isacartoucheclose}\ \isakeyword{and}\ {\isacartoucheopen}p\ {\isasymlongmapsto}a\ p{\isacharprime}{\kern0pt}{\isacartoucheclose}\ \isakeyword{and}\ {\isacartoucheopen}a\ {\isasymin}\ Y{\isacartoucheclose}\isanewline
\ \ \isacommand{thus}\isamarkupfalse%
\ {\isacartoucheopen}{\isasymexists}q{\isacharprime}{\kern0pt}{\isachardot}{\kern0pt}\ q\ {\isasymlongmapsto}a\ q{\isacharprime}{\kern0pt}\ {\isasymand}\ R\ p{\isacharprime}{\kern0pt}\ None\ q{\isacharprime}{\kern0pt}{\isacartoucheclose}\isanewline
\ \ \ \ \isacommand{using}\isamarkupfalse%
\ SRB{\isacharunderscore}{\kern0pt}ruleformat{\isacharparenleft}{\kern0pt}{\isadigit{5}}{\isacharparenright}{\kern0pt}{\isacharbrackleft}{\kern0pt}OF\ assms{\isacharbrackright}{\kern0pt}\ \isacommand{by}\isamarkupfalse%
\ blast\isanewline
\isacommand{next}\isamarkupfalse%
\isanewline
\ \ \isacommand{fix}\isamarkupfalse%
\ p\ Y\ q\ p{\isacharprime}{\kern0pt}\ a\isanewline
\ \ \isacommand{assume}\isamarkupfalse%
\ {\isacartoucheopen}R\ p\ {\isacharparenleft}{\kern0pt}Some\ Y{\isacharparenright}{\kern0pt}\ q{\isacartoucheclose}\ {\isacartoucheopen}a\ {\isasymin}\ visible{\isacharunderscore}{\kern0pt}actions{\isacartoucheclose}\ {\isacartoucheopen}p\ {\isasymlongmapsto}a\ p{\isacharprime}{\kern0pt}{\isacartoucheclose}\ \ {\isacartoucheopen}idle\ p\ Y{\isacartoucheclose}\isanewline
\ \ \isacommand{hence}\isamarkupfalse%
\ {\isacartoucheopen}R\ p\ None\ q{\isacartoucheclose}\ \isacommand{using}\isamarkupfalse%
\ SRB{\isacharunderscore}{\kern0pt}ruleformat{\isacharparenleft}{\kern0pt}{\isadigit{7}}{\isacharparenright}{\kern0pt}{\isacharbrackleft}{\kern0pt}OF\ assms{\isacharbrackright}{\kern0pt}\ \isacommand{by}\isamarkupfalse%
\ simp\isanewline
\ \ \isacommand{hence}\isamarkupfalse%
\ {\isacartoucheopen}R\ p\ {\isacharparenleft}{\kern0pt}Some\ {\isacharbraceleft}{\kern0pt}a{\isacharbraceright}{\kern0pt}{\isacharparenright}{\kern0pt}\ q{\isacartoucheclose}\ \isacommand{using}\isamarkupfalse%
\ SRB{\isacharunderscore}{\kern0pt}ruleformat{\isacharparenleft}{\kern0pt}{\isadigit{4}}{\isacharparenright}{\kern0pt}{\isacharbrackleft}{\kern0pt}OF\ assms{\isacharbrackright}{\kern0pt}\ {\isacartoucheopen}a\ {\isasymin}\ visible{\isacharunderscore}{\kern0pt}actions{\isacartoucheclose}\ \isacommand{by}\isamarkupfalse%
\ simp\isanewline
\ \ \isacommand{thus}\isamarkupfalse%
\ {\isacartoucheopen}{\isasymexists}q{\isacharprime}{\kern0pt}{\isachardot}{\kern0pt}\ q\ {\isasymlongmapsto}a\ q{\isacharprime}{\kern0pt}\ {\isasymand}\ R\ p{\isacharprime}{\kern0pt}\ None\ q{\isacharprime}{\kern0pt}{\isacartoucheclose}\ \isacommand{using}\isamarkupfalse%
\ SRB{\isacharunderscore}{\kern0pt}ruleformat{\isacharparenleft}{\kern0pt}{\isadigit{5}}{\isacharparenright}{\kern0pt}{\isacharbrackleft}{\kern0pt}OF\ assms{\isacharbrackright}{\kern0pt}\ {\isacartoucheopen}p\ {\isasymlongmapsto}a\ p{\isacharprime}{\kern0pt}{\isacartoucheclose}\ \isacommand{by}\isamarkupfalse%
\ blast\isanewline
\isacommand{next}\isamarkupfalse%
\isanewline
\ \ \isacommand{fix}\isamarkupfalse%
\ p\ Y\ q\ p{\isacharprime}{\kern0pt}\isanewline
\ \ \isacommand{assume}\isamarkupfalse%
\ {\isacartoucheopen}R\ p\ {\isacharparenleft}{\kern0pt}Some\ Y{\isacharparenright}{\kern0pt}\ q{\isacartoucheclose}\ \isakeyword{and}\ {\isacartoucheopen}p\ {\isasymlongmapsto}{\isasymtau}\ p{\isacharprime}{\kern0pt}{\isacartoucheclose}\isanewline
\ \ \isacommand{thus}\isamarkupfalse%
\ {\isacartoucheopen}{\isasymexists}q{\isacharprime}{\kern0pt}{\isachardot}{\kern0pt}\ q\ {\isasymlongmapsto}{\isasymtau}\ \ q{\isacharprime}{\kern0pt}\ {\isasymand}\ R\ p{\isacharprime}{\kern0pt}\ {\isacharparenleft}{\kern0pt}Some\ Y{\isacharparenright}{\kern0pt}\ q{\isacharprime}{\kern0pt}{\isacartoucheclose}\isanewline
\ \ \ \ \isacommand{using}\isamarkupfalse%
\ SRB{\isacharunderscore}{\kern0pt}ruleformat{\isacharparenleft}{\kern0pt}{\isadigit{6}}{\isacharparenright}{\kern0pt}{\isacharbrackleft}{\kern0pt}OF\ assms{\isacharbrackright}{\kern0pt}\ \isacommand{by}\isamarkupfalse%
\ blast\isanewline
\isacommand{next}\isamarkupfalse%
\isanewline
\ \ \isacommand{fix}\isamarkupfalse%
\ p\ Y\ q\ p{\isacharprime}{\kern0pt}\ X\isanewline
\ \ \isacommand{assume}\isamarkupfalse%
\ {\isacartoucheopen}R\ p\ {\isacharparenleft}{\kern0pt}Some\ Y{\isacharparenright}{\kern0pt}\ q{\isacartoucheclose}\ {\isacartoucheopen}idle\ p\ {\isacharparenleft}{\kern0pt}X\ {\isasymunion}\ Y{\isacharparenright}{\kern0pt}{\isacartoucheclose}\ {\isacartoucheopen}X\ {\isasymsubseteq}\ visible{\isacharunderscore}{\kern0pt}actions{\isacartoucheclose}\ {\isacartoucheopen}p\ {\isasymlongmapsto}t\ p{\isacharprime}{\kern0pt}{\isacartoucheclose}\isanewline
\ \ \isacommand{from}\isamarkupfalse%
\ {\isacartoucheopen}idle\ p\ {\isacharparenleft}{\kern0pt}X\ {\isasymunion}\ Y{\isacharparenright}{\kern0pt}{\isacartoucheclose}\ \isacommand{have}\isamarkupfalse%
\ {\isacartoucheopen}idle\ p\ Y{\isacartoucheclose}\ \isakeyword{and}\ {\isacartoucheopen}idle\ p\ X{\isacartoucheclose}\isanewline
\ \ \ \ \isacommand{by}\isamarkupfalse%
\ {\isacharparenleft}{\kern0pt}simp\ add{\isacharcolon}{\kern0pt}\ Int{\isacharunderscore}{\kern0pt}Un{\isacharunderscore}{\kern0pt}distrib{\isacharparenright}{\kern0pt}{\isacharplus}{\kern0pt}\isanewline
\ \ \isacommand{from}\isamarkupfalse%
\ {\isacartoucheopen}R\ p\ {\isacharparenleft}{\kern0pt}Some\ Y{\isacharparenright}{\kern0pt}\ q{\isacartoucheclose}\ {\isacartoucheopen}idle\ p\ Y{\isacartoucheclose}\ \isacommand{have}\isamarkupfalse%
\ {\isacartoucheopen}R\ p\ None\ q{\isacartoucheclose}\isanewline
\ \ \ \ \isacommand{using}\isamarkupfalse%
\ SRB{\isacharunderscore}{\kern0pt}ruleformat{\isacharparenleft}{\kern0pt}{\isadigit{7}}{\isacharparenright}{\kern0pt}{\isacharbrackleft}{\kern0pt}OF\ assms{\isacharbrackright}{\kern0pt}\ \isacommand{by}\isamarkupfalse%
\ blast\isanewline
\ \ \isacommand{with}\isamarkupfalse%
\ {\isacartoucheopen}X\ {\isasymsubseteq}\ visible{\isacharunderscore}{\kern0pt}actions{\isacartoucheclose}\ \isacommand{have}\isamarkupfalse%
\ {\isacartoucheopen}R\ p\ {\isacharparenleft}{\kern0pt}Some\ X{\isacharparenright}{\kern0pt}\ q{\isacartoucheclose}\ \isanewline
\ \ \ \ \isacommand{using}\isamarkupfalse%
\ SRB{\isacharunderscore}{\kern0pt}ruleformat{\isacharparenleft}{\kern0pt}{\isadigit{4}}{\isacharparenright}{\kern0pt}{\isacharbrackleft}{\kern0pt}OF\ assms{\isacharbrackright}{\kern0pt}\ \isacommand{by}\isamarkupfalse%
\ blast\isanewline
\ \ \isacommand{with}\isamarkupfalse%
\ {\isacartoucheopen}idle\ p\ X{\isacartoucheclose}\ {\isacartoucheopen}p\ {\isasymlongmapsto}t\ p{\isacharprime}{\kern0pt}{\isacartoucheclose}\ \isacommand{show}\isamarkupfalse%
\ {\isacartoucheopen}{\isasymexists}q{\isacharprime}{\kern0pt}{\isachardot}{\kern0pt}\ q\ {\isasymlongmapsto}t\ \ q{\isacharprime}{\kern0pt}\ {\isasymand}\ R\ p{\isacharprime}{\kern0pt}\ {\isacharparenleft}{\kern0pt}Some\ X{\isacharparenright}{\kern0pt}\ q{\isacharprime}{\kern0pt}{\isacartoucheclose}\isanewline
\ \ \ \ \isacommand{using}\isamarkupfalse%
\ SRB{\isacharunderscore}{\kern0pt}ruleformat{\isacharparenleft}{\kern0pt}{\isadigit{8}}{\isacharparenright}{\kern0pt}{\isacharbrackleft}{\kern0pt}OF\ assms{\isacharbrackright}{\kern0pt}\ \isacommand{by}\isamarkupfalse%
\ blast\isanewline
\isacommand{qed}\isamarkupfalse%
%
\endisatagproof
{\isafoldproof}%
%
\isadelimproof
%
\endisadelimproof
%
\begin{isamarkuptext}%
Then, we show that each GSRB can be extended to yield an SRB. First, we define this extension. Generally, GSRBs can be smaller than SRBs when proving reactive bisimilarity, because they require triples $(p,X,q)$ only after encountering $t$-transitions, whereas SRBs require these triples for all processes and environments. These triples (and also some process pairs $(p,q)$ related to environment time-outs) are re-added by this extension.%
\end{isamarkuptext}\isamarkuptrue%
\isacommand{definition}\isamarkupfalse%
\ GSRB{\isacharunderscore}{\kern0pt}extension\ \isanewline
\ \ {\isacharcolon}{\kern0pt}{\isacharcolon}{\kern0pt}\ {\isacartoucheopen}{\isacharparenleft}{\kern0pt}{\isacharprime}{\kern0pt}s{\isasymRightarrow}{\isacharprime}{\kern0pt}a\ set\ option{\isasymRightarrow}{\isacharprime}{\kern0pt}s\ {\isasymRightarrow}\ bool{\isacharparenright}{\kern0pt}{\isasymRightarrow}{\isacharparenleft}{\kern0pt}{\isacharprime}{\kern0pt}s{\isasymRightarrow}{\isacharprime}{\kern0pt}a\ set\ option{\isasymRightarrow}{\isacharprime}{\kern0pt}s\ {\isasymRightarrow}\ bool{\isacharparenright}{\kern0pt}{\isacartoucheclose}\isanewline
\ \ \isakeyword{where}\ {\isacartoucheopen}{\isacharparenleft}{\kern0pt}GSRB{\isacharunderscore}{\kern0pt}extension\ R{\isacharparenright}{\kern0pt}\ p\ XoN\ q\ {\isasymequiv}\isanewline
\ \ \ \ {\isacharparenleft}{\kern0pt}R\ p\ XoN\ q{\isacharparenright}{\kern0pt}\isanewline
\ \ \ \ {\isasymor}\ {\isacharparenleft}{\kern0pt}some{\isacharunderscore}{\kern0pt}visible{\isacharunderscore}{\kern0pt}subset\ XoN\ {\isasymand}\ R\ p\ None\ q{\isacharparenright}{\kern0pt}\isanewline
\ \ \ \ {\isasymor}\ {\isacharparenleft}{\kern0pt}{\isacharparenleft}{\kern0pt}XoN\ {\isacharequal}{\kern0pt}\ None\ {\isasymor}\ some{\isacharunderscore}{\kern0pt}visible{\isacharunderscore}{\kern0pt}subset\ XoN{\isacharparenright}{\kern0pt}\ \isanewline
\ \ \ \ \ \ {\isasymand}\ {\isacharparenleft}{\kern0pt}{\isasymexists}\ Y{\isachardot}{\kern0pt}\ R\ p\ {\isacharparenleft}{\kern0pt}Some\ Y{\isacharparenright}{\kern0pt}\ q\ {\isasymand}\ idle\ p\ Y{\isacharparenright}{\kern0pt}{\isacharparenright}{\kern0pt}{\isacartoucheclose}%
\begin{isamarkuptext}%
Now we show that this extension does, in fact, yield an SRB (again, by showing that all clauses of the definition of SRBs are satisfied).%
\end{isamarkuptext}\isamarkuptrue%
%
\isadelimunimportant
%
\endisadelimunimportant
%
\isatagunimportant
%
\endisatagunimportant
{\isafoldunimportant}%
%
\isadelimunimportant
\isanewline
%
\endisadelimunimportant
\isacommand{lemma}\isamarkupfalse%
\ GSRB{\isacharunderscore}{\kern0pt}extension{\isacharunderscore}{\kern0pt}is{\isacharunderscore}{\kern0pt}SRB{\isacharcolon}{\kern0pt}\isanewline
\ \ \isakeyword{assumes}\isanewline
\ \ \ \ {\isacartoucheopen}GSRB\ R{\isacartoucheclose}\isanewline
\ \ \isakeyword{shows}\isanewline
\ \ \ \ {\isacartoucheopen}SRB\ {\isacharparenleft}{\kern0pt}GSRB{\isacharunderscore}{\kern0pt}extension\ R{\isacharparenright}{\kern0pt}{\isacartoucheclose}\ {\isacharparenleft}{\kern0pt}\isakeyword{is}\ {\isacartoucheopen}SRB\ {\isacharquery}{\kern0pt}R{\isacharunderscore}{\kern0pt}ext{\isacartoucheclose}{\isacharparenright}{\kern0pt}\isanewline
%
\isadelimproof
\ \ %
\endisadelimproof
%
\isatagproof
\isacommand{unfolding}\isamarkupfalse%
\ SRB{\isacharunderscore}{\kern0pt}def\isanewline
\isacommand{proof}\isamarkupfalse%
\ {\isacharparenleft}{\kern0pt}safe{\isacharparenright}{\kern0pt}\isanewline
\ \ \isacommand{fix}\isamarkupfalse%
\ p\ XoN\ q\isanewline
\ \ \isacommand{assume}\isamarkupfalse%
\ {\isacartoucheopen}{\isacharquery}{\kern0pt}R{\isacharunderscore}{\kern0pt}ext\ p\ XoN\ q{\isacartoucheclose}\isanewline
\ \ \isacommand{thus}\isamarkupfalse%
\ {\isacartoucheopen}{\isacharquery}{\kern0pt}R{\isacharunderscore}{\kern0pt}ext\ q\ XoN\ p{\isacartoucheclose}\ \isanewline
\ \ \ \ \isacommand{unfolding}\isamarkupfalse%
\ GSRB{\isacharunderscore}{\kern0pt}extension{\isacharunderscore}{\kern0pt}def\isanewline
\ \ \isacommand{proof}\isamarkupfalse%
\ {\isacharparenleft}{\kern0pt}elim\ disjE{\isacharcomma}{\kern0pt}\ goal{\isacharunderscore}{\kern0pt}cases{\isacharparenright}{\kern0pt}\isanewline
\ \ \ \ \isacommand{case}\isamarkupfalse%
\ {\isadigit{1}}\isanewline
\ \ \ \ \isacommand{hence}\isamarkupfalse%
\ {\isacartoucheopen}R\ q\ XoN\ p{\isacartoucheclose}\isanewline
\ \ \ \ \ \ \isacommand{using}\isamarkupfalse%
\ assms\ GSRB{\isacharunderscore}{\kern0pt}def\ \isacommand{by}\isamarkupfalse%
\ auto\isanewline
\ \ \ \ \isacommand{thus}\isamarkupfalse%
\ {\isacharquery}{\kern0pt}case\ \isacommand{by}\isamarkupfalse%
\ simp\isanewline
\ \ \isacommand{next}\isamarkupfalse%
\isanewline
\ \ \ \ \isacommand{case}\isamarkupfalse%
\ {\isadigit{2}}\isanewline
\ \ \ \ \isacommand{hence}\isamarkupfalse%
\ {\isacartoucheopen}some{\isacharunderscore}{\kern0pt}visible{\isacharunderscore}{\kern0pt}subset\ XoN\ {\isasymand}\ R\ q\ None\ p{\isacartoucheclose}\isanewline
\ \ \ \ \ \ \isacommand{using}\isamarkupfalse%
\ assms\ GSRB{\isacharunderscore}{\kern0pt}def\ \isacommand{by}\isamarkupfalse%
\ auto\isanewline
\ \ \ \ \isacommand{thus}\isamarkupfalse%
\ {\isacharquery}{\kern0pt}case\ \isacommand{by}\isamarkupfalse%
\ simp\isanewline
\ \ \isacommand{next}\isamarkupfalse%
\isanewline
\ \ \ \ \isacommand{case}\isamarkupfalse%
\ {\isadigit{3}}\isanewline
\ \ \ \ \isacommand{then}\isamarkupfalse%
\ \isacommand{obtain}\isamarkupfalse%
\ Y\ \isakeyword{where}\ {\isacartoucheopen}R\ p\ {\isacharparenleft}{\kern0pt}Some\ Y{\isacharparenright}{\kern0pt}\ q{\isacartoucheclose}\ {\isacartoucheopen}idle\ p\ Y{\isacartoucheclose}\ \isacommand{by}\isamarkupfalse%
\ auto\isanewline
\ \ \ \ \isacommand{hence}\isamarkupfalse%
\ {\isacartoucheopen}R\ q\ {\isacharparenleft}{\kern0pt}Some\ Y{\isacharparenright}{\kern0pt}\ p{\isacartoucheclose}\isanewline
\ \ \ \ \ \ \isacommand{using}\isamarkupfalse%
\ assms\ GSRB{\isacharunderscore}{\kern0pt}def\ \isacommand{by}\isamarkupfalse%
\ auto\isanewline
\ \ \ \ \isacommand{have}\isamarkupfalse%
\ {\isacartoucheopen}idle\ q\ Y{\isacartoucheclose}\isanewline
\ \ \ \ \ \ \isacommand{using}\isamarkupfalse%
\ GSRB{\isacharunderscore}{\kern0pt}preserves{\isacharunderscore}{\kern0pt}idleness{\isacharbrackleft}{\kern0pt}OF\ assms{\isacharbrackright}{\kern0pt}\ {\isacartoucheopen}R\ p\ {\isacharparenleft}{\kern0pt}Some\ Y{\isacharparenright}{\kern0pt}\ q{\isacartoucheclose}\ {\isacartoucheopen}idle\ p\ Y{\isacartoucheclose}\ \isacommand{{\isachardot}{\kern0pt}}\isamarkupfalse%
\isanewline
\ \ \ \ \isacommand{from}\isamarkupfalse%
\ {\isadigit{3}}\ {\isacartoucheopen}R\ q\ {\isacharparenleft}{\kern0pt}Some\ Y{\isacharparenright}{\kern0pt}\ p{\isacartoucheclose}\ {\isacartoucheopen}idle\ q\ Y{\isacartoucheclose}\ \isacommand{show}\isamarkupfalse%
\ {\isacharquery}{\kern0pt}case\ \isacommand{by}\isamarkupfalse%
\ blast\isanewline
\ \ \isacommand{qed}\isamarkupfalse%
\isanewline
\isacommand{next}\isamarkupfalse%
\isanewline
\ \ \isacommand{fix}\isamarkupfalse%
\ p\ X\ q\ x\isanewline
\ \ \isacommand{assume}\isamarkupfalse%
\ {\isacartoucheopen}{\isacharquery}{\kern0pt}R{\isacharunderscore}{\kern0pt}ext\ p\ {\isacharparenleft}{\kern0pt}Some\ X{\isacharparenright}{\kern0pt}\ q{\isacartoucheclose}\ {\isacartoucheopen}x\ {\isasymin}\ X{\isacartoucheclose}\isanewline
\ \ \isacommand{thus}\isamarkupfalse%
\ {\isacartoucheopen}x\ {\isasymin}\ visible{\isacharunderscore}{\kern0pt}actions{\isacartoucheclose}\ \isanewline
\ \ \ \ \isacommand{unfolding}\isamarkupfalse%
\ GSRB{\isacharunderscore}{\kern0pt}extension{\isacharunderscore}{\kern0pt}def\isanewline
\ \ \isacommand{proof}\isamarkupfalse%
\ {\isacharparenleft}{\kern0pt}elim\ disjE{\isacharcomma}{\kern0pt}\ goal{\isacharunderscore}{\kern0pt}cases{\isacharparenright}{\kern0pt}\isanewline
\ \ \ \ \isacommand{case}\isamarkupfalse%
\ {\isadigit{1}}\isanewline
\ \ \ \ \isacommand{thus}\isamarkupfalse%
\ {\isacharquery}{\kern0pt}case\ \isacommand{using}\isamarkupfalse%
\ GSRB{\isacharunderscore}{\kern0pt}ruleformat{\isacharparenleft}{\kern0pt}{\isadigit{1}}{\isacharparenright}{\kern0pt}{\isacharbrackleft}{\kern0pt}OF\ assms{\isacharbrackright}{\kern0pt}\ \isacommand{by}\isamarkupfalse%
\ blast\isanewline
\ \ \isacommand{next}\isamarkupfalse%
\isanewline
\ \ \ \ \isacommand{case}\isamarkupfalse%
\ {\isadigit{2}}\isanewline
\ \ \ \ \isacommand{thus}\isamarkupfalse%
\ {\isacharquery}{\kern0pt}case\ \isacommand{by}\isamarkupfalse%
\ auto\isanewline
\ \ \isacommand{next}\isamarkupfalse%
\isanewline
\ \ \ \ \isacommand{case}\isamarkupfalse%
\ {\isadigit{3}}\isanewline
\ \ \ \ \isacommand{thus}\isamarkupfalse%
\ {\isacharquery}{\kern0pt}case\ \isacommand{by}\isamarkupfalse%
\ auto\isanewline
\ \ \isacommand{qed}\isamarkupfalse%
\isanewline
\isacommand{next}\isamarkupfalse%
\isanewline
\ \ \isacommand{fix}\isamarkupfalse%
\ p\ q\ p{\isacharprime}{\kern0pt}\isanewline
\ \ \isacommand{assume}\isamarkupfalse%
\ {\isacartoucheopen}{\isacharquery}{\kern0pt}R{\isacharunderscore}{\kern0pt}ext\ p\ None\ q{\isacartoucheclose}\ {\isacartoucheopen}p\ {\isasymlongmapsto}{\isasymtau}\ p{\isacharprime}{\kern0pt}{\isacartoucheclose}\isanewline
\ \ \isacommand{thus}\isamarkupfalse%
\ {\isacartoucheopen}{\isasymexists}q{\isacharprime}{\kern0pt}{\isachardot}{\kern0pt}\ q\ {\isasymlongmapsto}{\isasymtau}\ \ q{\isacharprime}{\kern0pt}\ {\isasymand}\ {\isacharquery}{\kern0pt}R{\isacharunderscore}{\kern0pt}ext\ p{\isacharprime}{\kern0pt}\ None\ q{\isacharprime}{\kern0pt}{\isacartoucheclose}\ \isanewline
\ \ \ \ \isacommand{unfolding}\isamarkupfalse%
\ GSRB{\isacharunderscore}{\kern0pt}extension{\isacharunderscore}{\kern0pt}def\isanewline
\ \ \isacommand{proof}\isamarkupfalse%
\ {\isacharparenleft}{\kern0pt}elim\ disjE{\isacharcomma}{\kern0pt}\ goal{\isacharunderscore}{\kern0pt}cases{\isacharparenright}{\kern0pt}\isanewline
\ \ \ \ \isacommand{case}\isamarkupfalse%
\ {\isadigit{1}}\isanewline
\ \ \ \ \isacommand{then}\isamarkupfalse%
\ \isacommand{obtain}\isamarkupfalse%
\ q{\isacharprime}{\kern0pt}\ \isakeyword{where}\ {\isacartoucheopen}q\ {\isasymlongmapsto}{\isasymtau}\ q{\isacharprime}{\kern0pt}{\isacartoucheclose}\ {\isacartoucheopen}R\ p{\isacharprime}{\kern0pt}\ None\ q{\isacharprime}{\kern0pt}{\isacartoucheclose}\isanewline
\ \ \ \ \ \ \isacommand{using}\isamarkupfalse%
\ GSRB{\isacharunderscore}{\kern0pt}ruleformat{\isacharparenleft}{\kern0pt}{\isadigit{3}}{\isacharparenright}{\kern0pt}{\isacharbrackleft}{\kern0pt}OF\ assms{\isacharbrackright}{\kern0pt}\ lts{\isacharunderscore}{\kern0pt}timeout{\isacharunderscore}{\kern0pt}axioms\ \isacommand{by}\isamarkupfalse%
\ fastforce\isanewline
\ \ \ \ \isacommand{thus}\isamarkupfalse%
\ {\isacharquery}{\kern0pt}case\ \isacommand{by}\isamarkupfalse%
\ auto\isanewline
\ \ \isacommand{next}\isamarkupfalse%
\isanewline
\ \ \ \ \isacommand{case}\isamarkupfalse%
\ {\isadigit{2}}\isanewline
\ \ \ \ \isacommand{hence}\isamarkupfalse%
\ False\ \isacommand{by}\isamarkupfalse%
\ auto\isanewline
\ \ \ \ \isacommand{thus}\isamarkupfalse%
\ {\isacharquery}{\kern0pt}case\ \isacommand{by}\isamarkupfalse%
\ simp\isanewline
\ \ \isacommand{next}\isamarkupfalse%
\isanewline
\ \ \ \ \isacommand{thm}\isamarkupfalse%
\ GSRB{\isacharunderscore}{\kern0pt}ruleformat{\isacharparenleft}{\kern0pt}{\isadigit{5}}{\isacharparenright}{\kern0pt}{\isacharbrackleft}{\kern0pt}OF\ assms{\isacharcomma}{\kern0pt}\ \isakeyword{where}\ {\isacharquery}{\kern0pt}a{\isacharequal}{\kern0pt}{\isasymtau}{\isacharbrackright}{\kern0pt}\isanewline
\ \ \ \ \isacommand{case}\isamarkupfalse%
\ {\isadigit{3}}\isanewline
\ \ \ \ \isacommand{hence}\isamarkupfalse%
\ {\isacartoucheopen}{\isasymexists}q{\isacharprime}{\kern0pt}{\isachardot}{\kern0pt}\ q\ {\isasymlongmapsto}{\isasymtau}\ \ q{\isacharprime}{\kern0pt}\ {\isasymand}\ R\ p{\isacharprime}{\kern0pt}\ None\ q{\isacharprime}{\kern0pt}{\isacartoucheclose}\isanewline
\ \ \ \ \ \ \isacommand{using}\isamarkupfalse%
\ initial{\isacharunderscore}{\kern0pt}actions{\isacharunderscore}{\kern0pt}def\ \isacommand{by}\isamarkupfalse%
\ fastforce\isanewline
\ \ \ \ \isacommand{thus}\isamarkupfalse%
\ {\isacharquery}{\kern0pt}case\ \isacommand{by}\isamarkupfalse%
\ auto\isanewline
\ \ \isacommand{qed}\isamarkupfalse%
\isanewline
\isacommand{next}\isamarkupfalse%
\isanewline
\ \ \isacommand{fix}\isamarkupfalse%
\ p\ q\ X\isanewline
\ \ \isacommand{assume}\isamarkupfalse%
\ {\isacartoucheopen}{\isacharquery}{\kern0pt}R{\isacharunderscore}{\kern0pt}ext\ p\ None\ q{\isacartoucheclose}\ {\isacartoucheopen}X\ {\isasymsubseteq}\ visible{\isacharunderscore}{\kern0pt}actions{\isacartoucheclose}\isanewline
\ \ \isacommand{thus}\isamarkupfalse%
\ {\isacartoucheopen}{\isacharquery}{\kern0pt}R{\isacharunderscore}{\kern0pt}ext\ p\ {\isacharparenleft}{\kern0pt}Some\ X{\isacharparenright}{\kern0pt}\ q{\isacartoucheclose}\ \isanewline
\ \ \ \ \isacommand{unfolding}\isamarkupfalse%
\ GSRB{\isacharunderscore}{\kern0pt}extension{\isacharunderscore}{\kern0pt}def\isanewline
\ \ \isacommand{proof}\isamarkupfalse%
\ {\isacharparenleft}{\kern0pt}elim\ disjE{\isacharcomma}{\kern0pt}\ goal{\isacharunderscore}{\kern0pt}cases{\isacharparenright}{\kern0pt}\isanewline
\ \ \ \ \isacommand{case}\isamarkupfalse%
\ {\isadigit{1}}\isanewline
\ \ \ \ \isacommand{thus}\isamarkupfalse%
\ {\isacharquery}{\kern0pt}case\ \isacommand{by}\isamarkupfalse%
\ auto\isanewline
\ \ \isacommand{next}\isamarkupfalse%
\isanewline
\ \ \ \ \isacommand{case}\isamarkupfalse%
\ {\isadigit{2}}\isanewline
\ \ \ \ \isacommand{hence}\isamarkupfalse%
\ False\ \isacommand{by}\isamarkupfalse%
\ auto\isanewline
\ \ \ \ \isacommand{thus}\isamarkupfalse%
\ {\isacharquery}{\kern0pt}case\ \isacommand{by}\isamarkupfalse%
\ simp\isanewline
\ \ \isacommand{next}\isamarkupfalse%
\isanewline
\ \ \ \ \isacommand{case}\isamarkupfalse%
\ {\isadigit{3}}\isanewline
\ \ \ \ \isacommand{hence}\isamarkupfalse%
\ {\isacartoucheopen}some{\isacharunderscore}{\kern0pt}visible{\isacharunderscore}{\kern0pt}subset\ {\isacharparenleft}{\kern0pt}Some\ X{\isacharparenright}{\kern0pt}{\isacartoucheclose}\ \isacommand{by}\isamarkupfalse%
\ simp\isanewline
\ \ \ \ \isacommand{with}\isamarkupfalse%
\ {\isadigit{3}}\ \isacommand{show}\isamarkupfalse%
\ {\isacharquery}{\kern0pt}case\ \isacommand{by}\isamarkupfalse%
\ simp\isanewline
\ \ \isacommand{qed}\isamarkupfalse%
\isanewline
\isacommand{next}\isamarkupfalse%
\isanewline
\ \ \isacommand{fix}\isamarkupfalse%
\ p\ X\ q\ p{\isacharprime}{\kern0pt}\ a\isanewline
\ \ \isacommand{assume}\isamarkupfalse%
\ {\isacartoucheopen}{\isacharquery}{\kern0pt}R{\isacharunderscore}{\kern0pt}ext\ p\ {\isacharparenleft}{\kern0pt}Some\ X{\isacharparenright}{\kern0pt}\ q{\isacartoucheclose}\ {\isacartoucheopen}p\ {\isasymlongmapsto}a\ p{\isacharprime}{\kern0pt}{\isacartoucheclose}\ {\isacartoucheopen}a\ {\isasymin}\ X{\isacartoucheclose}\isanewline
\ \ \isacommand{thus}\isamarkupfalse%
\ {\isacartoucheopen}{\isasymexists}q{\isacharprime}{\kern0pt}{\isachardot}{\kern0pt}\ q\ {\isasymlongmapsto}a\ q{\isacharprime}{\kern0pt}\ {\isasymand}\ {\isacharquery}{\kern0pt}R{\isacharunderscore}{\kern0pt}ext\ p{\isacharprime}{\kern0pt}\ None\ q{\isacharprime}{\kern0pt}{\isacartoucheclose}\ \isanewline
\ \ \ \ \isacommand{unfolding}\isamarkupfalse%
\ GSRB{\isacharunderscore}{\kern0pt}extension{\isacharunderscore}{\kern0pt}def\isanewline
\ \ \isacommand{proof}\isamarkupfalse%
\ {\isacharparenleft}{\kern0pt}elim\ disjE{\isacharcomma}{\kern0pt}\ goal{\isacharunderscore}{\kern0pt}cases{\isacharparenright}{\kern0pt}\isanewline
\ \ \ \ \isacommand{case}\isamarkupfalse%
\ {\isadigit{1}}\isanewline
\ \ \ \ \isacommand{thus}\isamarkupfalse%
\ {\isacharquery}{\kern0pt}case\ \isanewline
\ \ \ \ \ \ \isacommand{using}\isamarkupfalse%
\ GSRB{\isacharunderscore}{\kern0pt}ruleformat{\isacharparenleft}{\kern0pt}{\isadigit{1}}{\isacharcomma}{\kern0pt}{\isadigit{5}}{\isacharparenright}{\kern0pt}{\isacharbrackleft}{\kern0pt}OF\ assms{\isacharbrackright}{\kern0pt}\ \isacommand{by}\isamarkupfalse%
\ blast\isanewline
\ \ \isacommand{next}\isamarkupfalse%
\isanewline
\ \ \ \ \isacommand{case}\isamarkupfalse%
\ {\isadigit{2}}\isanewline
\ \ \ \ \isacommand{thus}\isamarkupfalse%
\ {\isacharquery}{\kern0pt}case\ \isanewline
\ \ \ \ \ \ \isacommand{using}\isamarkupfalse%
\ GSRB{\isacharunderscore}{\kern0pt}ruleformat{\isacharparenleft}{\kern0pt}{\isadigit{3}}{\isacharparenright}{\kern0pt}{\isacharbrackleft}{\kern0pt}OF\ assms{\isacharbrackright}{\kern0pt}\ \isacommand{by}\isamarkupfalse%
\ blast\isanewline
\ \ \isacommand{next}\isamarkupfalse%
\isanewline
\ \ \ \ \isacommand{case}\isamarkupfalse%
\ {\isadigit{3}}\isanewline
\ \ \ \ \isacommand{then}\isamarkupfalse%
\ \isacommand{obtain}\isamarkupfalse%
\ Y\ \isakeyword{where}\ {\isacartoucheopen}R\ p\ {\isacharparenleft}{\kern0pt}Some\ Y{\isacharparenright}{\kern0pt}\ q{\isacartoucheclose}\ {\isacartoucheopen}idle\ p\ Y{\isacartoucheclose}\ \isacommand{by}\isamarkupfalse%
\ blast\isanewline
\ \ \ \ \isacommand{with}\isamarkupfalse%
\ {\isadigit{3}}\ \isacommand{have}\isamarkupfalse%
\ {\isacartoucheopen}a\ {\isasymin}\ visible{\isacharunderscore}{\kern0pt}actions{\isacartoucheclose}\isanewline
\ \ \ \ \ \ \isacommand{using}\isamarkupfalse%
\ GSRB{\isacharunderscore}{\kern0pt}ruleformat{\isacharparenleft}{\kern0pt}{\isadigit{2}}{\isacharparenright}{\kern0pt}{\isacharbrackleft}{\kern0pt}OF\ assms{\isacharbrackright}{\kern0pt}\ \isacommand{by}\isamarkupfalse%
\ blast\isanewline
\ \ \ \ \isacommand{from}\isamarkupfalse%
\ {\isadigit{3}}\ {\isacartoucheopen}idle\ p\ Y{\isacartoucheclose}\ \isacommand{show}\isamarkupfalse%
\ {\isacharquery}{\kern0pt}case\ \isanewline
\ \ \ \ \ \ \isacommand{using}\isamarkupfalse%
\ GSRB{\isacharunderscore}{\kern0pt}ruleformat{\isacharparenleft}{\kern0pt}{\isadigit{5}}{\isacharparenright}{\kern0pt}{\isacharbrackleft}{\kern0pt}OF\ assms\ {\isacartoucheopen}R\ p\ {\isacharparenleft}{\kern0pt}Some\ Y{\isacharparenright}{\kern0pt}\ q{\isacartoucheclose}\ {\isacartoucheopen}a\ {\isasymin}\ visible{\isacharunderscore}{\kern0pt}actions{\isacartoucheclose}{\isacharbrackright}{\kern0pt}\ \isacommand{by}\isamarkupfalse%
\ metis\isanewline
\ \ \isacommand{qed}\isamarkupfalse%
\isanewline
\isacommand{next}\isamarkupfalse%
\isanewline
\ \ \isacommand{fix}\isamarkupfalse%
\ p\ X\ q\ p{\isacharprime}{\kern0pt}\isanewline
\ \ \isacommand{assume}\isamarkupfalse%
\ {\isacartoucheopen}{\isacharquery}{\kern0pt}R{\isacharunderscore}{\kern0pt}ext\ p\ {\isacharparenleft}{\kern0pt}Some\ X{\isacharparenright}{\kern0pt}\ q{\isacartoucheclose}\ {\isacartoucheopen}p\ {\isasymlongmapsto}{\isasymtau}\ p{\isacharprime}{\kern0pt}{\isacartoucheclose}\isanewline
\ \ \isacommand{thus}\isamarkupfalse%
\ {\isacartoucheopen}{\isasymexists}q{\isacharprime}{\kern0pt}{\isachardot}{\kern0pt}\ q\ {\isasymlongmapsto}{\isasymtau}\ \ q{\isacharprime}{\kern0pt}\ {\isasymand}\ {\isacharquery}{\kern0pt}R{\isacharunderscore}{\kern0pt}ext\ p{\isacharprime}{\kern0pt}\ {\isacharparenleft}{\kern0pt}Some\ X{\isacharparenright}{\kern0pt}\ q{\isacharprime}{\kern0pt}{\isacartoucheclose}\ \isanewline
\ \ \ \ \isacommand{unfolding}\isamarkupfalse%
\ GSRB{\isacharunderscore}{\kern0pt}extension{\isacharunderscore}{\kern0pt}def\isanewline
\ \ \isacommand{proof}\isamarkupfalse%
\ {\isacharparenleft}{\kern0pt}elim\ disjE{\isacharcomma}{\kern0pt}\ goal{\isacharunderscore}{\kern0pt}cases{\isacharparenright}{\kern0pt}\isanewline
\ \ \ \ \isacommand{case}\isamarkupfalse%
\ {\isadigit{1}}\isanewline
\ \ \ \ \isacommand{thus}\isamarkupfalse%
\ {\isacharquery}{\kern0pt}case\ \isanewline
\ \ \ \ \ \ \isacommand{using}\isamarkupfalse%
\ GSRB{\isacharunderscore}{\kern0pt}ruleformat{\isacharparenleft}{\kern0pt}{\isadigit{6}}{\isacharparenright}{\kern0pt}{\isacharbrackleft}{\kern0pt}OF\ assms{\isacharbrackright}{\kern0pt}\ \isacommand{by}\isamarkupfalse%
\ meson\isanewline
\ \ \isacommand{next}\isamarkupfalse%
\isanewline
\ \ \ \ \isacommand{case}\isamarkupfalse%
\ {\isadigit{2}}\isanewline
\ \ \ \ \isacommand{thus}\isamarkupfalse%
\ {\isacharquery}{\kern0pt}case\ \isanewline
\ \ \ \ \ \ \isacommand{using}\isamarkupfalse%
\ GSRB{\isacharunderscore}{\kern0pt}ruleformat{\isacharparenleft}{\kern0pt}{\isadigit{3}}{\isacharparenright}{\kern0pt}{\isacharbrackleft}{\kern0pt}OF\ assms{\isacharbrackright}{\kern0pt}\ \isacommand{by}\isamarkupfalse%
\ blast\isanewline
\ \ \isacommand{next}\isamarkupfalse%
\isanewline
\ \ \ \ \isacommand{case}\isamarkupfalse%
\ {\isadigit{3}}\isanewline
\ \ \ \ \isacommand{then}\isamarkupfalse%
\ \isacommand{obtain}\isamarkupfalse%
\ Y\ \isakeyword{where}\ {\isacartoucheopen}idle\ p\ Y{\isacartoucheclose}\ \isacommand{by}\isamarkupfalse%
\ blast\isanewline
\ \ \ \ \isacommand{with}\isamarkupfalse%
\ {\isadigit{3}}{\isacharparenleft}{\kern0pt}{\isadigit{1}}{\isacharparenright}{\kern0pt}\ \isacommand{have}\isamarkupfalse%
\ False\ \isanewline
\ \ \ \ \ \ \isacommand{using}\isamarkupfalse%
\ initial{\isacharunderscore}{\kern0pt}actions{\isacharunderscore}{\kern0pt}def\ \isacommand{by}\isamarkupfalse%
\ auto\isanewline
\ \ \ \ \isacommand{thus}\isamarkupfalse%
\ {\isacharquery}{\kern0pt}case\ \isacommand{by}\isamarkupfalse%
\ simp\isanewline
\ \ \isacommand{qed}\isamarkupfalse%
\isanewline
\isacommand{next}\isamarkupfalse%
\isanewline
\ \ \isacommand{fix}\isamarkupfalse%
\ p\ X\ q\isanewline
\ \ \isacommand{assume}\isamarkupfalse%
\ {\isacartoucheopen}{\isacharquery}{\kern0pt}R{\isacharunderscore}{\kern0pt}ext\ p\ {\isacharparenleft}{\kern0pt}Some\ X{\isacharparenright}{\kern0pt}\ q{\isacartoucheclose}\ {\isacartoucheopen}idle\ p\ X{\isacartoucheclose}\isanewline
\ \ \isacommand{thus}\isamarkupfalse%
\ {\isacartoucheopen}{\isacharquery}{\kern0pt}R{\isacharunderscore}{\kern0pt}ext\ p\ None\ q{\isacartoucheclose}\ \isanewline
\ \ \ \ \isacommand{unfolding}\isamarkupfalse%
\ GSRB{\isacharunderscore}{\kern0pt}extension{\isacharunderscore}{\kern0pt}def\ \isacommand{by}\isamarkupfalse%
\ auto\isanewline
\isacommand{next}\isamarkupfalse%
\isanewline
\ \ \isacommand{fix}\isamarkupfalse%
\ p\ X\ q\ p{\isacharprime}{\kern0pt}\isanewline
\ \ \isacommand{assume}\isamarkupfalse%
\ {\isacartoucheopen}{\isacharquery}{\kern0pt}R{\isacharunderscore}{\kern0pt}ext\ p\ {\isacharparenleft}{\kern0pt}Some\ X{\isacharparenright}{\kern0pt}\ q{\isacartoucheclose}\ {\isacartoucheopen}idle\ p\ X{\isacartoucheclose}\ {\isacartoucheopen}p\ {\isasymlongmapsto}t\ p{\isacharprime}{\kern0pt}{\isacartoucheclose}\isanewline
\ \ \isacommand{thus}\isamarkupfalse%
\ {\isacartoucheopen}{\isasymexists}q{\isacharprime}{\kern0pt}{\isachardot}{\kern0pt}\ q\ {\isasymlongmapsto}t\ \ q{\isacharprime}{\kern0pt}\ {\isasymand}\ {\isacharquery}{\kern0pt}R{\isacharunderscore}{\kern0pt}ext\ p{\isacharprime}{\kern0pt}\ {\isacharparenleft}{\kern0pt}Some\ X{\isacharparenright}{\kern0pt}\ q{\isacharprime}{\kern0pt}{\isacartoucheclose}\ \isanewline
\ \ \ \ \isacommand{unfolding}\isamarkupfalse%
\ GSRB{\isacharunderscore}{\kern0pt}extension{\isacharunderscore}{\kern0pt}def\isanewline
\ \ \isacommand{proof}\isamarkupfalse%
\ {\isacharparenleft}{\kern0pt}elim\ disjE{\isacharcomma}{\kern0pt}\ goal{\isacharunderscore}{\kern0pt}cases{\isacharparenright}{\kern0pt}\isanewline
\ \ \ \ \isacommand{case}\isamarkupfalse%
\ {\isadigit{1}}\isanewline
\ \ \ \ \isacommand{from}\isamarkupfalse%
\ {\isadigit{1}}{\isacharparenleft}{\kern0pt}{\isadigit{1}}{\isacharparenright}{\kern0pt}\ \isacommand{have}\isamarkupfalse%
\ {\isacartoucheopen}idle\ p\ {\isacharparenleft}{\kern0pt}X\ {\isasymunion}\ X{\isacharparenright}{\kern0pt}{\isacartoucheclose}\ \isacommand{by}\isamarkupfalse%
\ simp\isanewline
\ \ \ \ \isacommand{from}\isamarkupfalse%
\ GSRB{\isacharunderscore}{\kern0pt}ruleformat{\isacharparenleft}{\kern0pt}{\isadigit{1}}{\isacharparenright}{\kern0pt}{\isacharbrackleft}{\kern0pt}OF\ assms\ {\isadigit{1}}{\isacharparenleft}{\kern0pt}{\isadigit{3}}{\isacharparenright}{\kern0pt}{\isacharbrackright}{\kern0pt}\ \isacommand{have}\isamarkupfalse%
\ {\isacartoucheopen}X\ {\isasymsubseteq}\ visible{\isacharunderscore}{\kern0pt}actions{\isacartoucheclose}\ \isacommand{{\isachardot}{\kern0pt}}\isamarkupfalse%
\isanewline
\ \ \ \ \isacommand{from}\isamarkupfalse%
\ GSRB{\isacharunderscore}{\kern0pt}ruleformat{\isacharparenleft}{\kern0pt}{\isadigit{7}}{\isacharparenright}{\kern0pt}{\isacharbrackleft}{\kern0pt}OF\ assms\ {\isadigit{1}}{\isacharparenleft}{\kern0pt}{\isadigit{3}}{\isacharparenright}{\kern0pt}\ {\isacartoucheopen}idle\ p\ {\isacharparenleft}{\kern0pt}X\ {\isasymunion}\ X{\isacharparenright}{\kern0pt}{\isacartoucheclose}\ {\isacartoucheopen}X\ {\isasymsubseteq}\ visible{\isacharunderscore}{\kern0pt}actions{\isacartoucheclose}\ {\isadigit{1}}{\isacharparenleft}{\kern0pt}{\isadigit{2}}{\isacharparenright}{\kern0pt}{\isacharbrackright}{\kern0pt}\isanewline
\ \ \ \ \isacommand{show}\isamarkupfalse%
\ {\isacharquery}{\kern0pt}case\ \isacommand{by}\isamarkupfalse%
\ auto\isanewline
\ \ \isacommand{next}\isamarkupfalse%
\isanewline
\ \ \ \ \isacommand{case}\isamarkupfalse%
\ {\isadigit{2}}\isanewline
\ \ \ \ \isacommand{thus}\isamarkupfalse%
\ {\isacharquery}{\kern0pt}case\isanewline
\ \ \ \ \ \ \isacommand{using}\isamarkupfalse%
\ GSRB{\isacharunderscore}{\kern0pt}ruleformat{\isacharparenleft}{\kern0pt}{\isadigit{4}}{\isacharparenright}{\kern0pt}{\isacharbrackleft}{\kern0pt}OF\ assms{\isacharbrackright}{\kern0pt}\isanewline
\ \ \ \ \ \ \isacommand{by}\isamarkupfalse%
\ {\isacharparenleft}{\kern0pt}metis\ option{\isachardot}{\kern0pt}inject{\isacharparenright}{\kern0pt}\isanewline
\ \ \isacommand{next}\isamarkupfalse%
\isanewline
\ \ \ \ \isacommand{case}\isamarkupfalse%
\ {\isadigit{3}}\isanewline
\ \ \ \ \isacommand{then}\isamarkupfalse%
\ \isacommand{obtain}\isamarkupfalse%
\ Y\ \isakeyword{where}\ {\isacartoucheopen}R\ p\ {\isacharparenleft}{\kern0pt}Some\ Y{\isacharparenright}{\kern0pt}\ q{\isacartoucheclose}\ {\isacartoucheopen}idle\ p\ Y{\isacartoucheclose}\ \isacommand{by}\isamarkupfalse%
\ blast\isanewline
\ \ \ \ \isacommand{from}\isamarkupfalse%
\ {\isacartoucheopen}idle\ p\ X{\isacartoucheclose}\ {\isacartoucheopen}idle\ p\ Y{\isacartoucheclose}\ \isacommand{have}\isamarkupfalse%
\ {\isacartoucheopen}idle\ p\ {\isacharparenleft}{\kern0pt}X\ {\isasymunion}\ Y{\isacharparenright}{\kern0pt}{\isacartoucheclose}\isanewline
\ \ \ \ \ \ \isacommand{by}\isamarkupfalse%
\ {\isacharparenleft}{\kern0pt}smt\ bot{\isacharunderscore}{\kern0pt}eq{\isacharunderscore}{\kern0pt}sup{\isacharunderscore}{\kern0pt}iff\ inf{\isacharunderscore}{\kern0pt}sup{\isacharunderscore}{\kern0pt}distrib{\isadigit{1}}{\isacharparenright}{\kern0pt}\isanewline
\ \ \ \ \isacommand{from}\isamarkupfalse%
\ {\isadigit{3}}{\isacharparenleft}{\kern0pt}{\isadigit{3}}{\isacharparenright}{\kern0pt}\ \isacommand{have}\isamarkupfalse%
\ {\isacartoucheopen}X\ {\isasymsubseteq}\ visible{\isacharunderscore}{\kern0pt}actions{\isacartoucheclose}\ \isacommand{by}\isamarkupfalse%
\ blast\isanewline
\ \ \ \ \isacommand{from}\isamarkupfalse%
\ GSRB{\isacharunderscore}{\kern0pt}ruleformat{\isacharparenleft}{\kern0pt}{\isadigit{7}}{\isacharparenright}{\kern0pt}{\isacharbrackleft}{\kern0pt}OF\ assms\ {\isacartoucheopen}R\ p\ {\isacharparenleft}{\kern0pt}Some\ Y{\isacharparenright}{\kern0pt}\ q{\isacartoucheclose}\ {\isacartoucheopen}idle\ p\ {\isacharparenleft}{\kern0pt}X\ {\isasymunion}\ Y{\isacharparenright}{\kern0pt}{\isacartoucheclose}\ {\isacartoucheopen}X\ {\isasymsubseteq}\ visible{\isacharunderscore}{\kern0pt}actions{\isacartoucheclose}\ {\isadigit{3}}{\isacharparenleft}{\kern0pt}{\isadigit{2}}{\isacharparenright}{\kern0pt}{\isacharbrackright}{\kern0pt}\isanewline
\ \ \ \ \isacommand{show}\isamarkupfalse%
\ {\isacharquery}{\kern0pt}case\ \isacommand{by}\isamarkupfalse%
\ auto\isanewline
\ \ \isacommand{qed}\isamarkupfalse%
\isanewline
\isacommand{qed}\isamarkupfalse%
%
\endisatagproof
{\isafoldproof}%
%
\isadelimproof
%
\endisadelimproof
%
\begin{isamarkuptext}%
Finally, we can conclude the following:%
\end{isamarkuptext}\isamarkuptrue%
\isacommand{lemma}\isamarkupfalse%
\ GSRB{\isacharunderscore}{\kern0pt}whenever{\isacharunderscore}{\kern0pt}SRB{\isacharcolon}{\kern0pt}\isanewline
\ \ \isakeyword{shows}\ {\isacartoucheopen}{\isacharparenleft}{\kern0pt}{\isasymexists}\ R{\isachardot}{\kern0pt}\ GSRB\ R\ {\isasymand}\ R\ p\ XoN\ q{\isacharparenright}{\kern0pt}\ \ {\isasymLongleftrightarrow}\ \ {\isacharparenleft}{\kern0pt}{\isasymexists}\ R{\isachardot}{\kern0pt}\ SRB\ R\ {\isasymand}\ R\ p\ XoN\ q{\isacharparenright}{\kern0pt}{\isacartoucheclose}\isanewline
%
\isadelimproof
\ \ %
\endisadelimproof
%
\isatagproof
\isacommand{by}\isamarkupfalse%
\ {\isacharparenleft}{\kern0pt}metis\ GSRB{\isacharunderscore}{\kern0pt}extension{\isacharunderscore}{\kern0pt}def\ GSRB{\isacharunderscore}{\kern0pt}extension{\isacharunderscore}{\kern0pt}is{\isacharunderscore}{\kern0pt}SRB\ SRB{\isacharunderscore}{\kern0pt}is{\isacharunderscore}{\kern0pt}GSRB{\isacharparenright}{\kern0pt}%
\endisatagproof
{\isafoldproof}%
%
\isadelimproof
%
\endisadelimproof
%
\begin{isamarkuptext}%
This, now, directly implies that GSRBs do charactarise strong reactive/$X$-bisimilarity.%
\end{isamarkuptext}\isamarkuptrue%
%
\isadelimvisible
%
\endisadelimvisible
%
\isatagvisible
\isacommand{proposition}\isamarkupfalse%
\ GSRBs{\isacharunderscore}{\kern0pt}characterise{\isacharunderscore}{\kern0pt}strong{\isacharunderscore}{\kern0pt}reactive{\isacharunderscore}{\kern0pt}bisimilarity{\isacharcolon}{\kern0pt}\isanewline
\ \ {\isacartoucheopen}p\ {\isasymleftrightarrow}\isactrlsub r\ q\ {\isasymLongleftrightarrow}\ {\isacharparenleft}{\kern0pt}{\isasymexists}\ R{\isachardot}{\kern0pt}\ GSRB\ R\ {\isasymand}\ R\ p\ None\ q{\isacharparenright}{\kern0pt}{\isacartoucheclose}\isanewline
\ \ \isacommand{using}\isamarkupfalse%
\ GSRB{\isacharunderscore}{\kern0pt}whenever{\isacharunderscore}{\kern0pt}SRB\ strongly{\isacharunderscore}{\kern0pt}reactive{\isacharunderscore}{\kern0pt}bisimilar{\isacharunderscore}{\kern0pt}def\ \isacommand{by}\isamarkupfalse%
\ blast\isanewline
\isanewline
\isacommand{proposition}\isamarkupfalse%
\ GSRBs{\isacharunderscore}{\kern0pt}characterise{\isacharunderscore}{\kern0pt}strong{\isacharunderscore}{\kern0pt}X{\isacharunderscore}{\kern0pt}bisimilarity{\isacharcolon}{\kern0pt}\isanewline
\ \ {\isacartoucheopen}p\ {\isasymleftrightarrow}\isactrlsub r\isactrlsup X\ q\ {\isasymLongleftrightarrow}\ {\isacharparenleft}{\kern0pt}{\isasymexists}\ R{\isachardot}{\kern0pt}\ GSRB\ R\ {\isasymand}\ R\ p\ {\isacharparenleft}{\kern0pt}Some\ X{\isacharparenright}{\kern0pt}\ q{\isacharparenright}{\kern0pt}{\isacartoucheclose}\isanewline
\ \ \isacommand{using}\isamarkupfalse%
\ GSRB{\isacharunderscore}{\kern0pt}whenever{\isacharunderscore}{\kern0pt}SRB\ strongly{\isacharunderscore}{\kern0pt}X{\isacharunderscore}{\kern0pt}bisimilar{\isacharunderscore}{\kern0pt}def\ \isacommand{by}\isamarkupfalse%
\ blast%
\endisatagvisible
{\isafoldvisible}%
%
\isadelimvisible
%
\endisadelimvisible
\isanewline
\isanewline
\isacommand{end}\isamarkupfalse%
\ %
\isamarkupcmt{of \isa{context\ lts{\isacharunderscore}{\kern0pt}timeout}%
}%
\begin{isamarkuptext}%
As a little meta-comment, I would like to point out that van~Glabbeek's proof spans a total of five lines (\enquote{Clearly, \textelp{}}, \enquote{It is straightforward to check \textelp{}}), whereas the Isabelle proof takes up around 250 lines of code. This just goes to show that for things which are clear and straightforward for humans, it might require quite some effort to \enquote{explain} them to a computer.%
\end{isamarkuptext}\isamarkuptrue%
%
\isadelimtheory
%
\endisadelimtheory
%
\isatagtheory
%
\endisatagtheory
{\isafoldtheory}%
%
\isadelimtheory
%
\endisadelimtheory
%
\end{isabellebody}%
\endinput
%:%file=~/reactive-bisimilarity-reduction/Reactive_Bisimilarity.thy%:%
%:%24=8%:%
%:%36=9%:%
%:%40=11%:%
%:%41=12%:%
%:%42=13%:%
%:%43=14%:%
%:%44=15%:%
%:%45=16%:%
%:%46=17%:%
%:%47=18%:%
%:%48=19%:%
%:%49=20%:%
%:%50=21%:%
%:%51=22%:%
%:%52=23%:%
%:%53=24%:%
%:%54=25%:%
%:%55=26%:%
%:%56=27%:%
%:%57=28%:%
%:%58=29%:%
%:%59=30%:%
%:%60=31%:%
%:%61=32%:%
%:%62=33%:%
%:%63=34%:%
%:%64=35%:%
%:%65=36%:%
%:%74=38%:%
%:%86=39%:%
%:%87=40%:%
%:%88=41%:%
%:%89=42%:%
%:%90=43%:%
%:%91=44%:%
%:%92=45%:%
%:%93=46%:%
%:%94=47%:%
%:%95=48%:%
%:%96=49%:%
%:%97=50%:%
%:%98=51%:%
%:%99=52%:%
%:%100=53%:%
%:%101=54%:%
%:%102=55%:%
%:%103=56%:%
%:%104=57%:%
%:%105=58%:%
%:%106=59%:%
%:%107=60%:%
%:%108=61%:%
%:%109=62%:%
%:%110=63%:%
%:%111=64%:%
%:%112=65%:%
%:%113=66%:%
%:%114=67%:%
%:%115=68%:%
%:%116=69%:%
%:%117=70%:%
%:%118=71%:%
%:%119=72%:%
%:%128=75%:%
%:%140=76%:%
%:%149=79%:%
%:%161=80%:%
%:%170=83%:%
%:%182=85%:%
%:%191=88%:%
%:%203=90%:%
%:%204=91%:%
%:%205=92%:%
%:%206=93%:%
%:%207=94%:%
%:%208=95%:%
%:%209=96%:%
%:%210=97%:%
%:%211=98%:%
%:%212=99%:%
%:%213=100%:%
%:%214=101%:%
%:%215=102%:%
%:%217=104%:%
%:%218=104%:%
%:%219=105%:%
%:%221=106%:%
%:%222=106%:%
%:%223=107%:%
%:%224=107%:%
%:%225=108%:%
%:%239=122%:%
%:%240=123%:%
%:%261=140%:%
%:%273=142%:%
%:%275=144%:%
%:%276=144%:%
%:%277=145%:%
%:%278=146%:%
%:%279=147%:%
%:%280=148%:%
%:%281=148%:%
%:%282=149%:%
%:%283=150%:%
%:%285=152%:%
%:%287=154%:%
%:%288=154%:%
%:%289=155%:%
%:%290=156%:%
%:%293=159%:%
%:%295=161%:%
%:%296=161%:%
%:%299=162%:%
%:%303=162%:%
%:%304=162%:%
%:%305=162%:%
%:%310=162%:%
%:%313=163%:%
%:%314=163%:%
%:%317=164%:%
%:%321=164%:%
%:%322=164%:%
%:%323=164%:%
%:%337=167%:%
%:%349=169%:%
%:%352=171%:%
%:%353=171%:%
%:%354=172%:%
%:%355=172%:%
%:%356=173%:%
%:%372=189%:%
%:%373=190%:%
%:%394=207%:%
%:%406=209%:%
%:%408=211%:%
%:%409=211%:%
%:%410=212%:%
%:%411=213%:%
%:%414=214%:%
%:%418=214%:%
%:%419=214%:%
%:%420=215%:%
%:%421=215%:%
%:%422=216%:%
%:%423=216%:%
%:%424=217%:%
%:%425=217%:%
%:%426=218%:%
%:%427=218%:%
%:%428=219%:%
%:%429=219%:%
%:%430=219%:%
%:%431=220%:%
%:%432=220%:%
%:%433=221%:%
%:%434=221%:%
%:%435=222%:%
%:%436=222%:%
%:%437=223%:%
%:%438=223%:%
%:%439=224%:%
%:%440=224%:%
%:%441=224%:%
%:%442=225%:%
%:%443=225%:%
%:%444=226%:%
%:%445=226%:%
%:%446=227%:%
%:%447=227%:%
%:%448=228%:%
%:%449=228%:%
%:%450=229%:%
%:%451=229%:%
%:%452=229%:%
%:%453=230%:%
%:%454=230%:%
%:%455=231%:%
%:%456=231%:%
%:%457=232%:%
%:%458=232%:%
%:%459=233%:%
%:%460=233%:%
%:%461=234%:%
%:%462=234%:%
%:%463=234%:%
%:%464=235%:%
%:%465=235%:%
%:%466=236%:%
%:%467=236%:%
%:%468=237%:%
%:%469=237%:%
%:%470=238%:%
%:%471=238%:%
%:%472=239%:%
%:%473=239%:%
%:%474=239%:%
%:%475=240%:%
%:%476=240%:%
%:%477=241%:%
%:%478=241%:%
%:%479=242%:%
%:%480=242%:%
%:%481=243%:%
%:%482=243%:%
%:%483=244%:%
%:%484=244%:%
%:%485=244%:%
%:%486=245%:%
%:%487=245%:%
%:%488=246%:%
%:%489=246%:%
%:%490=247%:%
%:%491=247%:%
%:%492=248%:%
%:%493=248%:%
%:%494=248%:%
%:%495=248%:%
%:%496=249%:%
%:%497=249%:%
%:%498=249%:%
%:%499=249%:%
%:%500=250%:%
%:%501=250%:%
%:%502=250%:%
%:%503=250%:%
%:%504=251%:%
%:%505=251%:%
%:%506=252%:%
%:%507=252%:%
%:%508=253%:%
%:%509=253%:%
%:%510=254%:%
%:%511=254%:%
%:%512=255%:%
%:%513=255%:%
%:%514=255%:%
%:%515=256%:%
%:%516=256%:%
%:%517=257%:%
%:%518=257%:%
%:%519=258%:%
%:%520=258%:%
%:%521=259%:%
%:%522=259%:%
%:%523=259%:%
%:%524=260%:%
%:%525=260%:%
%:%526=261%:%
%:%527=261%:%
%:%528=261%:%
%:%529=262%:%
%:%530=262%:%
%:%531=262%:%
%:%532=263%:%
%:%533=263%:%
%:%534=263%:%
%:%535=264%:%
%:%536=264%:%
%:%537=264%:%
%:%538=265%:%
%:%539=265%:%
%:%540=265%:%
%:%541=266%:%
%:%542=266%:%
%:%543=266%:%
%:%544=267%:%
%:%554=269%:%
%:%556=271%:%
%:%557=271%:%
%:%558=272%:%
%:%559=273%:%
%:%565=279%:%
%:%578=317%:%
%:%581=318%:%
%:%582=318%:%
%:%583=319%:%
%:%584=320%:%
%:%585=321%:%
%:%586=322%:%
%:%589=323%:%
%:%593=323%:%
%:%594=323%:%
%:%595=324%:%
%:%596=324%:%
%:%597=325%:%
%:%598=325%:%
%:%599=326%:%
%:%600=326%:%
%:%601=327%:%
%:%602=327%:%
%:%603=328%:%
%:%604=328%:%
%:%605=329%:%
%:%606=329%:%
%:%607=330%:%
%:%608=330%:%
%:%609=331%:%
%:%610=331%:%
%:%611=332%:%
%:%612=332%:%
%:%613=332%:%
%:%614=333%:%
%:%615=333%:%
%:%616=333%:%
%:%617=334%:%
%:%618=334%:%
%:%619=335%:%
%:%620=335%:%
%:%621=336%:%
%:%622=336%:%
%:%623=337%:%
%:%624=337%:%
%:%625=337%:%
%:%626=338%:%
%:%627=338%:%
%:%628=338%:%
%:%629=339%:%
%:%630=339%:%
%:%631=340%:%
%:%632=340%:%
%:%633=341%:%
%:%634=341%:%
%:%635=341%:%
%:%636=341%:%
%:%637=342%:%
%:%638=342%:%
%:%639=343%:%
%:%640=343%:%
%:%641=343%:%
%:%642=344%:%
%:%643=344%:%
%:%644=345%:%
%:%645=345%:%
%:%646=345%:%
%:%647=346%:%
%:%648=346%:%
%:%649=346%:%
%:%650=346%:%
%:%651=347%:%
%:%652=347%:%
%:%653=348%:%
%:%654=348%:%
%:%655=349%:%
%:%656=349%:%
%:%657=350%:%
%:%658=350%:%
%:%659=351%:%
%:%660=351%:%
%:%661=352%:%
%:%662=352%:%
%:%663=353%:%
%:%664=353%:%
%:%665=354%:%
%:%666=354%:%
%:%667=355%:%
%:%668=355%:%
%:%669=355%:%
%:%670=355%:%
%:%671=356%:%
%:%672=356%:%
%:%673=357%:%
%:%674=357%:%
%:%675=358%:%
%:%676=358%:%
%:%677=358%:%
%:%678=359%:%
%:%679=359%:%
%:%680=360%:%
%:%681=360%:%
%:%682=361%:%
%:%683=361%:%
%:%684=361%:%
%:%685=362%:%
%:%686=362%:%
%:%687=363%:%
%:%688=363%:%
%:%689=364%:%
%:%690=364%:%
%:%691=365%:%
%:%692=365%:%
%:%693=366%:%
%:%694=366%:%
%:%695=367%:%
%:%696=367%:%
%:%697=368%:%
%:%698=368%:%
%:%699=369%:%
%:%700=369%:%
%:%701=370%:%
%:%702=370%:%
%:%703=370%:%
%:%704=371%:%
%:%705=371%:%
%:%706=371%:%
%:%707=372%:%
%:%708=372%:%
%:%709=372%:%
%:%710=373%:%
%:%711=373%:%
%:%712=374%:%
%:%713=374%:%
%:%714=375%:%
%:%715=375%:%
%:%716=375%:%
%:%717=376%:%
%:%718=376%:%
%:%719=376%:%
%:%720=377%:%
%:%721=377%:%
%:%722=378%:%
%:%723=378%:%
%:%724=379%:%
%:%725=379%:%
%:%726=380%:%
%:%727=380%:%
%:%728=381%:%
%:%729=381%:%
%:%730=381%:%
%:%731=382%:%
%:%732=382%:%
%:%733=382%:%
%:%734=383%:%
%:%735=383%:%
%:%736=384%:%
%:%737=384%:%
%:%738=385%:%
%:%739=385%:%
%:%740=386%:%
%:%741=386%:%
%:%742=387%:%
%:%743=387%:%
%:%744=388%:%
%:%745=388%:%
%:%746=389%:%
%:%747=389%:%
%:%748=390%:%
%:%749=390%:%
%:%750=391%:%
%:%751=391%:%
%:%752=391%:%
%:%753=392%:%
%:%754=392%:%
%:%755=393%:%
%:%756=393%:%
%:%757=394%:%
%:%758=394%:%
%:%759=394%:%
%:%760=395%:%
%:%761=395%:%
%:%762=395%:%
%:%763=396%:%
%:%764=396%:%
%:%765=397%:%
%:%766=397%:%
%:%767=398%:%
%:%768=398%:%
%:%769=398%:%
%:%770=399%:%
%:%771=399%:%
%:%772=399%:%
%:%773=399%:%
%:%774=400%:%
%:%775=400%:%
%:%776=401%:%
%:%777=401%:%
%:%778=402%:%
%:%779=402%:%
%:%780=403%:%
%:%781=403%:%
%:%782=404%:%
%:%783=404%:%
%:%784=405%:%
%:%785=405%:%
%:%786=406%:%
%:%787=406%:%
%:%788=407%:%
%:%789=407%:%
%:%790=408%:%
%:%791=408%:%
%:%792=409%:%
%:%793=409%:%
%:%794=409%:%
%:%795=410%:%
%:%796=410%:%
%:%797=411%:%
%:%798=411%:%
%:%799=412%:%
%:%800=412%:%
%:%801=413%:%
%:%802=413%:%
%:%803=413%:%
%:%804=414%:%
%:%805=414%:%
%:%806=415%:%
%:%807=415%:%
%:%808=416%:%
%:%809=416%:%
%:%810=416%:%
%:%811=416%:%
%:%812=417%:%
%:%813=417%:%
%:%814=417%:%
%:%815=418%:%
%:%816=418%:%
%:%817=418%:%
%:%818=419%:%
%:%819=419%:%
%:%820=419%:%
%:%821=420%:%
%:%822=420%:%
%:%823=420%:%
%:%824=421%:%
%:%825=421%:%
%:%826=422%:%
%:%827=422%:%
%:%828=423%:%
%:%829=423%:%
%:%830=424%:%
%:%831=424%:%
%:%832=425%:%
%:%833=425%:%
%:%834=426%:%
%:%835=426%:%
%:%836=427%:%
%:%837=427%:%
%:%838=428%:%
%:%839=428%:%
%:%840=429%:%
%:%841=429%:%
%:%842=430%:%
%:%843=430%:%
%:%844=430%:%
%:%845=431%:%
%:%846=431%:%
%:%847=432%:%
%:%848=432%:%
%:%849=433%:%
%:%850=433%:%
%:%851=434%:%
%:%852=434%:%
%:%853=434%:%
%:%854=435%:%
%:%855=435%:%
%:%856=436%:%
%:%857=436%:%
%:%858=437%:%
%:%859=437%:%
%:%860=437%:%
%:%861=437%:%
%:%862=438%:%
%:%863=438%:%
%:%864=438%:%
%:%865=439%:%
%:%866=439%:%
%:%867=439%:%
%:%868=440%:%
%:%869=440%:%
%:%870=440%:%
%:%871=441%:%
%:%872=441%:%
%:%873=442%:%
%:%874=442%:%
%:%875=443%:%
%:%876=443%:%
%:%877=444%:%
%:%878=444%:%
%:%879=445%:%
%:%880=445%:%
%:%881=446%:%
%:%882=446%:%
%:%883=446%:%
%:%884=447%:%
%:%885=447%:%
%:%886=448%:%
%:%887=448%:%
%:%888=449%:%
%:%889=449%:%
%:%890=450%:%
%:%891=450%:%
%:%892=451%:%
%:%893=451%:%
%:%894=452%:%
%:%895=452%:%
%:%896=453%:%
%:%897=453%:%
%:%898=454%:%
%:%899=454%:%
%:%900=454%:%
%:%901=454%:%
%:%902=455%:%
%:%903=455%:%
%:%904=455%:%
%:%905=455%:%
%:%906=456%:%
%:%907=456%:%
%:%908=457%:%
%:%909=457%:%
%:%910=457%:%
%:%911=458%:%
%:%912=458%:%
%:%913=459%:%
%:%914=459%:%
%:%915=460%:%
%:%916=460%:%
%:%917=461%:%
%:%918=461%:%
%:%919=462%:%
%:%920=462%:%
%:%921=463%:%
%:%922=463%:%
%:%923=464%:%
%:%924=464%:%
%:%925=465%:%
%:%926=465%:%
%:%927=465%:%
%:%928=465%:%
%:%929=466%:%
%:%930=466%:%
%:%931=466%:%
%:%932=467%:%
%:%933=467%:%
%:%934=468%:%
%:%935=468%:%
%:%936=468%:%
%:%937=468%:%
%:%938=469%:%
%:%939=469%:%
%:%940=470%:%
%:%941=470%:%
%:%942=470%:%
%:%943=471%:%
%:%944=471%:%
%:%945=472%:%
%:%955=474%:%
%:%957=476%:%
%:%958=476%:%
%:%959=477%:%
%:%962=478%:%
%:%966=478%:%
%:%967=478%:%
%:%976=480%:%
%:%984=482%:%
%:%985=482%:%
%:%986=483%:%
%:%987=484%:%
%:%988=484%:%
%:%989=484%:%
%:%990=485%:%
%:%991=486%:%
%:%992=486%:%
%:%993=487%:%
%:%994=488%:%
%:%995=488%:%
%:%996=488%:%
%:%1003=488%:%
%:%1004=489%:%
%:%1005=490%:%
%:%1006=490%:%
%:%1007=490%:%
%:%1010=493%:%